\input amssym
\magnification1200
\language2  %Ymp�rist�st� riippuva ratkaisu!!

\font\tlustazupa=cmbx10 scaled\magstep2
\font\zupa=cmr10 scaled\magstep1
\font\upa=msbm10 scaled\magstep0
\def\R{\hbox{\upa \char'122}}     % definition of reals
\def\Q{\hbox{\upa \char'121}}     % rationals
\def\Z{\hbox{\upa \char'132}}     % integers
\def\C{\hbox{\upa \char'103}}     % complex numbers
\def\N{\hbox{\upa \char'116}}     % natural numbers
\def\frac#1#2{{#1\over #2}}
\abovedisplayskip=4pt plus 2pt minus 2pt
\belowdisplayskip=4pt plus 2pt minus 3pt
\hsize=15cm
\vsize=21cm
\voffset=-1cm
\hoffset=-1cm
\overfullrule=0pt
\let\angle=\measuredangle

\centerline{\tlustazupa 9. Baltian Tie %joukkuekilpailu 1998
}

\medskip
\centerline{\zupa Varsova, 8. marraskuuta 1998}

\bigskip\noindent
\line{Version: {\bf Finnish}\hfill Kilpailuaika:
$4{1\over2}$ tuntia.}

\bigskip
\item{\bf 1.}
Etsi kaikki kahden muuttujan funktiot $f$,
joiden muuttujat $x$, $y$
ja arvot  $f(x,y)$ ovat positiivisia kokonaislukuja ja 
jotka toteuttavat seuraavat ehdot
(kaikilla positiivisilla kokonaisluvuilla $x$ ja~$y$):
$$\eqalign{f(x,x)&=x,\cr f(x,y)&=f(y,x),\cr
(x+y)f(x,y)&=yf(x,x{+}y).\cr}$$

\bigskip
\item{\bf 2.}
%\def\q{quasi-Pythagorean triple}%
%\def\Q{Quasi-Pythagorean triple}%
Positiivisten kokonaislukujen kolmikko $(a,b,c)$
on {\it kvasipythagoralainen\/},
jos on olemassa kolmio, jonka sivujen pituudet
ovat $a$, $b$, $c$ 
ja jonka sivun $c$ vastainen kulma on $120^\circ$.
Todista, ett\"a jos $(a,b,c)$  on kvasipythagoralainen
kolmikko, niin luvulla $c$ on lukua $5$ suurempi alkutekij\"a.

\bigskip
\item{\bf 3.}
Etsi kaikki positiivisten kokonaislukuparit
$x,\,y$, jotka toteuttavat yht\"al\"on
$$2x^2+5y^2=11(xy-11).$$

\bigskip
\item{\bf 4.}
Olkoon $P$ kokonaiskertoiminen polynomi.
Oletetaan, ett\"a jokaisella $n=1,2,3,\ldots,1998$
luku $P(n)$ on kolminumeroinen positiivinen kokonaisluku.
Todista, ett\"a polynomilla $P$ ei ole koko\-nais\-lukujuuria.


\bigskip
\item{\bf 5.}
Olkoon $a$ pariton ja  $b$ parillinen numero.
Todista, ett\"a jokaista positiivista kokonaislukua~$n$
kohti on olemassa luvulla $2^n$ jaollinen
positiivinen kokonaisluku,
jonka kymmenj\"arjestelm\"aesityksess\"a esiintyy
vain numeroita $a$~and~$b$.

\bigskip
\item{\bf 6.}
Olkoot $P$ kuudennen asteen polynomi ja $a$ ja $b$ reaalilukuja,
joille $0<a<b$. Oletetaan, ett\"a $P(a)=P(-a)$,
$P(b)=P(-b)$ ja $P'(0)=0$. Osoita, ett\"a $P(x)=P(-x)$ kaikille
reaaliluvuille $x$.

\bigskip
\item{\bf 7.}
Olkoon $\R$ reaalilukujen joukko. Etsi kaikki funktiot
${f\colon\R\to\R}$, joille kaikilla ${x,y\in\R}$ p\"atee
$\;f(x)+f(y)=f(f(x)f(y)).$

\bigskip
\item{\bf 8.} Olkoon ${P_k(x)=1+x+x^2+\cdots+x^{k-1}}$. Osoita,
ett\"a
$$\;\displaystyle{\sum_{k=1}^n{n\choose k}P_k(x)
=2^{n-1}P_n\Bigl({1+x\over2}\Bigr)}\;$$
kaikilla reaaliluvuilla $x$ ja kaikilla positiivisilla
kokonaisluvuilla~$n$.

\bigskip
\item{\bf 9.}
Reaaliluvuille $\alpha$, $\beta$ p\"atee $0<\alpha<\beta<\pi/2$
ja luvut $\gamma$ ja $\delta$ m\"a\"aritell\"a\"an ehdoilla:
\hfill\break (i) \ $0<\gamma<\pi/2$, ja $\tan\gamma$
on lukujen $\tan\alpha$ ja
$\tan\beta$ aritmeettinen keskiarvo;
\hfill\break (ii) \ $0<\delta<\pi/2$, ja
$\displaystyle{1\over\cos\delta}$ on lukujen
$\displaystyle{1\over\cos\alpha}$ ja
$\displaystyle{1\over\cos\beta}$ aritmeettinen keskiarvo.
\hfill\break Osoita, ett\"a
$\gamma<\delta$.

\bigskip
\item{\bf 10.}
Olkoon ${n\ge4}$ parillinen kokonaisluku. S\"a\"ann\"ollinen
$n$-kulmio ja s\"a\"ann\"ollinen
${(n{-}1)}$-kulmio on piirretty yksikk\"oympyr\"an sis\"a\"an.
Jokaisesta $n$-kulmion k\"arjest\"a mitataan et\"aisyys
l\"ahimp\"a\"an ${(n{-}1)}$-kulmion k\"arkeen ympyr\"an keh\"a\"a
pitkin. Olkoon $S$ n\"aiden $n$ et\"aisyyden summa. Osoita,
ett\"a $S$ riippuu vain luvusta~$n$, ei monikulmioiden
keskin\"aisest\"a sijainnista.

\bigskip
\item{\bf 11.}
Olkoot $a$, $b$ ja $c$ kolmion sivujen pituudet.
Olkoon $R$ kolmion
ymp\"aripiirretyn ympyr\"an s\"ade. Osoita, ett\"a
$$R\geq{{a^{2}+b^{2}}\over {2\,\sqrt{2a^{2}+2b^{2}-c^{2}}}}\ .$$
Milloin yht\"asuuruus on voimassa$\,$?

\bigskip
\item{\bf 12.}
Kolmiolle $ABC$ p\"atee $\angle BAC=90^\circ$. Piste $D$
on sivulla $BC$ ja toteuttaa ehdon $\angle BDA=2\angle BAD$.
Osoita, ett\"a
$$\frac{1}{AD}=\frac{1}{2}\left(\frac{1}{BD}+\frac{1}{CD}\right).$$

\bigskip
\item{\bf 13.}
Kuperan viisikulmion $ABCDE$ sivut $AE$ ja $BC$ ovat
yhdensuuntaisia ja $\angle ADE=\angle BDC$. L\"avist\"aj\"at $AC$
ja $BE$ leikkaavat pisteess\"a $P$. Osoita, ett\"a
$\angle EAD=\angle
BDP$ ja $\angle CBD=\angle ADP$.

\bigskip
\item{\bf 14.}
Kolmiolle $ABC$ p\"atee $AB<AC$. Pisteen
$B$ kautta kulkeva sivun $AC$ suuntainen suora
leikkaa kulman $\angle BAC$ vieruskulman puolittajan
pisteess\"a $D$. Pisteen $C$ kautta kulkeva sivun $AB$
suuntainen suora kohtaa saman kulmanpuolittajan
pisteess\"a $E$. Piste $F$ on sivulla $AC$ ja toteuttaa ehdon
$FC=AB$. Osoita, ett\"a $DF=FE$.

\bigskip
\item{\bf 15.}
Ter\"av\"akulmaisessa
kolmiossa $ABC$ piste $D$ on pisteest\"a $A$ sivulle $BC$
piirretyn korkeusjanan kantapiste. Piste $E$ on janalla $AD$
ja toteuttaa ehdon $${AE\over ED}={CD\over DB}.$$ Piste
$F$ on pisteest\"a $D$ sivulle $BE$ piirretyn korkeusjanan
kantapiste. Osoita, ett\"a
$\angle AFC=90^\circ$.

\bigskip
\item{\bf 16.}
Voiko $13\times 13$-shakkilaudan peitt\"a\"a
nelj\"all\"akymmenell\"akahdella $4\times 1$-nappulalla niin,
ett\"a vain shakkilaudan keskiruutu j\"a\"a peitt\"am\"att\"a?
(Oletetaan, ett\"a jokainen nappula peitt\"a\"a
t\"asm\"alleen nelj\"a shakkilaudan ruutua.)

\bigskip
\item{\bf 17.}
Olkoot $n$ ja $k$ positiivisia kokonaislukuja.
K\"ayt\"oss\"a on $nk$ (samankokoista) esinett\"a
ja $k$ laatikkoa, joista kuhunkin mahtuu $n$ esinett\"a.
Jokainen esineist\"a v\"aritet\"a\"an yhdell\"a
$k$:sta eri v\"arist\"a. Osoita, ett\"a esineet voidaan pakata
laatikoihin niin, ett\"a jokaiseen laatikoista tulee enint\"a\"an
kahden eri v\"arin esineit\"a.

\bigskip
\item{\bf 18.}
M\"a\"arit\"a kaikki sellaiset positiiviset kokonaisluvut $n$,
ett\"a on olemassa joukko $S$, jolla on seuraavat ominaisuudet:
\hfill\break (i) $S$ koostuu
$n$ positiivisesta kokonaisluvusta, jotka kaikki ovat
pienempi\"a kuin $2^{n-1}$;
\hfill\break
(ii) Jos $A$ ja $B$ ovat joukon $S$ eri osajoukkoja, niin
joukon $A$ alkioiden summa on eri kuin joukon $B$
alkioiden summa.


\bigskip
\item{\bf 19.}
Tarkastellaan kahden joukkueen v\"alist\"a
p\"oyt\"atennisottelua; kummassakin joukkueessa oli
1000 pelaajaa.  Jokainen pelaaja pelasi t\"asm\"alleen
yhden pelin kutakin toisen joukkueen pelaajaa vastaan
(p\"oyt\"atenniksess\"a ei ole tasapelej\"a).
Todista, ett\"a on olemassa sellaiset kymmenen saman joukkueen
j\"asent\"a, ett\"a jokainen toisen joukkueen j\"asenist\"a
h\"avisi ainakin yhden pelin n\"ait\"a kymment\"a pelaajaa
vastaan.


\bigskip
\item{\bf 20.}
Positiivisen kokonaisluvun $m$ sanotaan {\it peitt\"av\"an\/}
luvun 1998, jos 1, 9, 9, 8 esiintyv\"at, t\"ass\"a
j\"arjestyksess\"a, luvun $m$ numeroina. (Esimerkiksi
2{\bf 1}59{\bf 9}36{\bf 98} peitt\"a\"a luvun 1998,
mutta \hfill\break 213326798 ei.) Olkoon $k(n)$ niiden
positiivisten kokonaislukujen lukum\"a\"ar\"a, jotka
peitt\"av\"at luvun 1998 ja joissa on tasan $n$
nollasta poikkevaa numeroa ($n\geq 5$).
Mik\"a on jakoj\"a\"ann\"os,
kun luku $k(n)$ jaetaan luvulla $8$?

\bye
