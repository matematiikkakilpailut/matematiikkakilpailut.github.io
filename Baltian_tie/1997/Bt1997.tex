\documentclass[12pt]{article}

\usepackage[T1]{fontenc}
%\usepackage{A4}
\usepackage{a4}
\usepackage{amsmath,amsfonts,amssymb}
\usepackage{epic}
\usepackage{floatflt}
%\usepackage[finnish]{babel}

\newcommand{\abs}[1]{\lvert#1\rvert}
%\newcommand{\pr}[2]{\noindent{\bf#1. }\input{#2}\par\smallskip}

\let\angle=\measuredangle

\begin{document}
{\centering\section*{Baltian tie 1997}
K\"o\"openhamina, 9.\thinspace11.\thinspace1997, klo 10.00.\par\bigskip
Aika: $4\textstyle{\frac{1}{2}}$ tuntia.\quad Jokaisesta teht\"av\"ast\"a
voi saada 5~pistett\"a.\par\bigskip}

\setlength{\parindent}{0pt}
\setlength{\parskip}{\smallskipamount}



%\pr{1}{a3}

\textbf{1.\ } M\"a\"arit\"a kaikki reaaliarvoiset reaalilukufunktiot $f$,
jotka eiv\"at ole nollafunktioita ja joille
$$f(x)f(y)=f(x-y)$$ kaikilla reaaliluvuilla $x$ ja $y$.

%\pr{2}{a1}

\textbf{2.\ }
Olkoon $a_1,a_2,a_3,\dots$ jono positiivisia koko\-nais\-lukuja, jossa
jokainen posi\-tii\-vinen koko\-nais\-luku esiintyy t\"asm\"alleen kerran.
Osoita, ett\"a on olemassa koko\-nais\-luvut $\ell$ ja $m$, 
$1<\ell<m$, joille $a_1+a_m=2a_{\ell}$.

%\pr{3}{no1}

\textbf{3.\ }
Olkoon $x_1=1$ ja
$x_{n+1}=x_n+\left\lfloor\frac{x_n}{n}\right\rfloor+2$,
kun $n=1,2,3,\ldots$,
miss\"a $\lfloor x\rfloor$ tarkoittaa suurinta $x$:\"a\"a pienemp\"a\"a
koko\-naislukua. M\"a\"arit\"a $x_{1997}$.

%\pr{4}{a10}
\textbf{4.\ }
Todista, ett\"a lukujen $x_1,\ldots,x_n$ aritmeettiselle keskiarvolle $a$ 
p\"atee 
$$(x_1-a)^2+\cdots+(x_n-a)^2\le{1\over2}(|x_1-a|+\cdots+|x_n-a|)^2\,.$$

%\pr{5}{a8}
\textbf{5.\ }
Positiivisten koko\-naislukujen jonossa $u_0,u_1,\ldots$ luku $u_0$ on
mielivaltainen ja jokaisella ei-negatiivisella kokonaisluvulla $n$ p\"atee
$$u_{n+1}=\left\{\begin{array}{ll}
\textstyle\frac{1}{2}u_n&\mbox{kun $u_n$\,on parillinen,}\\
a+u_n&\mbox{kun $u_n$\,on pariton,}\end{array}\right.$$
miss\"a $a$ on pariton kiinte\"a positiivinen koko\-naisluku. 
Todista, ett\"a jono on jostain j\"asenest\"a\"an alkaen jaksollinen.

%\pr{6}{n1}
\textbf{6.\ }
Etsi kaikki ei-negatiivisten lukujen kolmikot $(a,b,c)$, 
joille p\"atee $a\ge b\ge c$ ja
\[ 1\cdot a^3 \,+\, 9\cdot b^2 \,+\, 9\cdot c \,+\, 7 ~=~ 1997. \]

%\pr{7}{n5}
\textbf{7.\ }
Olkoot $P$ ja $Q$ koko\-naislukukertoimisia polynomeja.
Oletetaan, ett\"a koko\-naisluvut $a$ ja $a+1997$ ovat polynomin
$P$ juuria ja  $Q(1998)=2000$. Osoita, ett\"a yht\"al\"oll\"a
$Q(P(x))=1$ ei ole koko\-naislukuratkaisuja.

%\pr{8}{n6}
\textbf{8.\ }
Kun lukuun $1996$ lis\"at\"a\"an  $1997$, ensin numerot  $6$
ja $7$ lasketaan yhteen. Kun tulokseksi saadaan $13$, numero $3$ 
kirjoitetaan yhteenlaskurivin alle ja numero $1$
kirjoitetaan seuraavaan sarakkeeseen muistiluvuksi.  N\"ain jatkaen huomataan, 
ett\"a kaikkiaan tarvitaan kolme muistilukua: \arraycolsep=1pt
$$
\begin{array}{ccccc}
&\mbox{\scriptsize $1$}&\mbox{\scriptsize $1$}
&\mbox{\scriptsize $1$}&\\[-.5ex]
&1&9&9&6\\[-.5ex]
+\;&1&9&9&7\\
\hline
&3&9&9&3
\end{array}
$$
Onko olemassa sellainen positiivinen koko\-nais\-luku
$k$, ett\"a lukujen  $1996{\cdot}k$ ja $1997{\cdot}k$ 
yhteenlaskussa ei tarvita muistilukuja?

%\pr{9}{newn4}
\textbf{9.\ }
Maailmanpallon maailmat on numeroitu 1, 2, 3, \dots
ja jokaisella positiivisella koko\-naisluvulla  $n\geq 1$, 
taikuri Gandalf voi siirty\"a mihin tahansa 
suuntaan maailmojen  $n$, $2n$ ja $3n+1$ v\"alill\"a. 
Voiko Gandalf siirty\"a mist\"a tahansa maailmasta mihin
tahansa maailmaan?

%\pr{10}{n8}
\textbf{10.\ }
Osoita, ett\"a mist\"a tahansa 79 per\"akk\"aisen positiivisen
koko\-naisluvun jonosta l\"oytyy luku, jonka kymmenj\"arjestelm\"aesityksen
numeroiden summa on jaollinen luvulla 13.

%\pr{11}{g2}
\begin{floatingfigure}[r]{4cm}
  \setlength{\unitlength}{1pt}
  \begin{picture}(50,50)(0,0)
    \multiput(5,10)(10,0){10}{\circle*{2}}
    \multiput(10,40)(20,0){5}{\circle*{2}}
    \put(0,10){\line(1,0){100}}
    \put(0,40){\line(1,0){100}}
    \drawline(5,10)(10,40)(15,10)(30,40)(25,10)(50,40)(35,10)(70,40)(45,10)(90,40)
    (55,10)(101,34) \put(0,0){\scriptsize
      $A_1$}\put(10,0){\scriptsize $A_2$}
    \put(20,0){\scriptsize $A_3$}\put(30,0){\scriptsize
      $A_4$} \put(40,0){\scriptsize
      $A_5$}\put(50,0){\scriptsize $A_6$}
    \put(5,44){\scriptsize $B_1$}\put(25,44){\scriptsize
      $B_2$} \put(45,44){\scriptsize
      $B_3$}\put(65,44){\scriptsize $B_4$}
    \put(85,44){\scriptsize $B_5$}%\put(105,44){\scriptsize $B_6$}
  \end{picture}
\end{floatingfigure}
\textbf{11.\ }
Kahdella yhdensuuntaisella suoralla on annettuna pisteet $A_1$,~$A_2$,
$A_3$, \dots ja $B_1$,~$B_2$, $B_3$, \dots (kuten kuvassa), joille
p\"atee $\abs{A_iA_{i+1}}=1$ ja
$\abs{B_iB_{i+1}}=2$ kaikilla $i=1,2,\ldots$.
Olkoon $\angle A_1A_2B_1=\alpha$. Laske
$$\angle A_1B_1A_2 + \angle A_2B_2A_3 + \angle A_3B_3A_4
+ \cdots.$$

%\pr{12}{g3}
\textbf{12.\ }
Kaksi ympyr\"a\"a  ${\cal C}_1$ ja ${\cal C}_2$ leikkaavat pisteiss\"a
$P$
ja $Q$. Pisteen $P$ kautta kulkeva jana leikkaa ympyr\"an ${\cal C}_1$ 
pisteess\"a $A$ ja ympyr\"an 
${\cal C}_2$ pisteess\"a $B$. Olkoon  $X$ janan
$AB$ keskipiste. Suora $QX$ leikkaa ympyr\"an 
${\cal C}_1$ pisteess\"a $Y$ ja ympyr\"an ${\cal C}_2$ pisteess\"a $Z$.
Osoita, ett\"a $X$ on janan $YZ$ keskipiste.

%\pr{13}{g6}
\parbox[b]{0.7\textwidth}{%
  \vspace{3pt}\textbf{13.\ } Samalla suoralla oleville pisteille $A$,~$B$, 
  $C$, $D$
  ja~$E$ p\"atee $\abs{AB} = \abs{BC} = \abs{CD}
  = \abs{DE}$. Piste~$F$ ei ole t\"all\"a suoralla. Olkoon $G$
  kolmion~$ADF$ ymp\"aripiirretyn ympyr\"an keskipiste ja $H$
  kolmion~$BEF$ ymp\"aripiirretyn ympyr\"an keskipiste. Osoita, ett\"a
  suorat~$GH$ ja~$FC$ ovat kohtisuorassa toisiaan vastaan.}
{\setlength{\unitlength}{1pt}
  \smash{\begin{picture}(50,60)(-10,-10)
%\put(-10,0){\vector(1,0){100}} \put(0,-10){\vector(0,1){40}}
\drawline(0,0)(80,0)
\drawline(0,0)(55,25)(20,0)
\drawline(55,25)(80,0)\drawline(40,0)(55,25)(60,0)
\put(-10,-10){\small $A$}\put(16,-10){\small $B$}\put(36,-10){\small $C$}
\put(56,-10){\small $D$}\put(76,-10){\small $E$}\put(52,30){\small $F$}
\end{picture}}}

%\pr{14}{g9}
\textbf{14.\ }
Kolmiossa $ABC$ p\"atee, ett\"a $|AC|^2$ on
$|BC|^2$:n ja $|AB|^2$:n aritmeettinen keskiarvo. Osoita, ett\"a 
$\cot^2 B\ge\cot A\cot C$. (Huom! $\cot v = \cos v/\sin v$.)

%\pr{15}{g12}
\textbf{15.\ }
Ter\"av\"akulmaisessa kolmiossa $ABC$ kulmien $\angle A$,
$\angle B$ ja $\angle C$ puolittajat leikkaavat kolmion ymp\"aripiirretyn
ympyr\"an pisteiss\"a
$A_1$, $B_1$ ja $C_1$, t\"ass\"a j\"arjestyksess\"a. Piste $M$ on
suorien $AB$ ja $B_1C_1$ leikkauspiste, ja $N$ on
suorien $BC$ ja $A_1B_1$ leikkauspiste. Osoita, ett\"a 
$MN$ kulkee kolmion $\triangle ABC$ sis\"a\"anpiirretyn ympyr\"an kautta.

%\pr{16}{d4}
\textbf{16.\ }
Kaksi pelaajaa pelaa  $5\times 5$-shakkilaudalla seuraavaa peli\"a:
Ensimm\"ainen pelaaja asettaa ratsun jollekin ruudulle.
T\"am\"an j\"alkeen pelaajat siirt\"av\"at ratsua shakin s\"a\"ant\"ojen
mukaisesti (j\"alkimm\"ainen pelaaja aloittaa siirtelyn). 
Ratsua ei saa siirt\"a\"a ruutuun, jossa se jo oli.
Pelaaja, joka ei voi en\"a\"a siirt\"a\"a, h\"avi\"a\"a pelin. 
Kummalla pelaajista on voittostrategia?

%\pr{17}{d5}
\textbf{17.\ }
Suorakaiteen voi jakaa $n$ samankokoiseen neli\"o\"on.
Saman suorakaiteen voi my\"os jakaa $n+76$ pienemp\"a\"an
samankokoiseen neli\"o\"on. M\"a\"arit\"a $n$.

%\pr{18}{d9}
\textbf{18.\ }
(\emph{i}) Osoita, ett\"a on olemassa kaksi, toisiaan mahdollisesti
leikkaavaa, \"a\"are\-t\"ont\"a joukkoa $A$ ja $B$, joille p\"atee,
ett\"a jokainen ep\"anegatiivinen koko\-naisluku $n$
voidaan esitt\"a\"a yksik\"asitteisell\"a tavalla muodossa $n=a+b$,
miss\"a  $a\in A$ ja $b\in B$.

\thinspace (\emph{ii})
Osoita, ett\"a jokaisesta sellaisesta parista $(A,B)$ joko joukko $A$
tai joukko $B$ sis\"alt\"a\"a ainoastaan jonkin koko\-naisluvun
$k>1$ monikertoja.

%\pr{19}{d6}
\textbf{19.\ }
Mets\"an $n$ el\"aint\"a $(n\geq3)$ el\"a\"a kukin omassa luolassaan, ja
kustakin luolasta on polku kuhunkin toiseen. Ennen mets\"an kuninkaan
vaalia jotkin el\"aimet kampanjoivat ehdokkuutensa puolesta.
Jokainen kampanjoiva el\"ain vierailee kunkin toisen el\"aimen luona t�sm�lleen 
kerran, k\"aytt\"a\"a aina polkua siirtyess\"a\"an luolasta toiseen, ei
koskaan poikkea polulta toiselle siirtyess\"a\"an luolasta toiseen ja
palaa kampanjansa lopuksi omaan luolaansa.  Tiedet\"a\"an my\"os,
ett\"a kutakin polkua k\"aytt\"a\"a vain yksi kampanjoiva el\"ain.
\begin{itemize}
\item[a)]
  Osoita, ett\"a jokaisella alku\-luvulla $n$ suurin mahdollinen
  kampanjoivien el\"ain\-ten m\"a\"ar\"a on $\frac{n-1}{2}$;
\item[b)]
  Etsi suurin mahdollinen kampanjoivien el\"ainten lukum\"a\"ar\"a,
  kun $n=9$.
\end{itemize}

%\pr{20}{d7}
\textbf{20.\ }
Riviss\"a on kaksitoista korttia, joita on kolmenlaisia:
joko molemmat puolet ovat valkoisia, molemmat ovat mustia
tai toinen on valkoinen ja toinen on musta.
Aluksi yhdeks\"an korteista on musta puoli yl\"osp\"ain.
Sitten kortit~1--6 k\"a\"annet\"a\"an, jonka j\"alkeen
nelj\"ast\"a kortista on musta puoli yl\"osp\"ain. 
Sitten kortit~4--9 k\"a\"annet\"a\"an  ja kuudesta kortista
on musta puoli yl\"osp\"ain.
Lopuksi kortit 1--3 ja~10--12 k\"a\"annet\"a\"an, mink\"a j\"alkeen
viidest\"a kortista on musta puoli yl\"osp\"ain.
Kuinka monta kunkin lajin korttia on p\"oyd\"all\"a? 

\end{document}
