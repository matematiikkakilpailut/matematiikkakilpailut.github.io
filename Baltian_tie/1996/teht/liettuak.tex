\input bw96

 
\def\ctg{{\rm ctg}}

\logoandtitle{         Baltijos kelias 1996 }

\kieli{Lithuanian}

\prob %VI5
Sakykim ,kad   $\a$  ir   $\b$ =0 yra bet kurie du per taisyklingojo
1996 -kampio istri{\u z}ines i{\u s}vestuju tiesiu  sankirtos kampai.
Irodykite,kad tokiu atveju  $\a/\b$ yra racionalusis skaicius.    
                 





\prob %RU2
  Zemiau parodytoje figuroje skritulys   $C$ liecia du pusskritulius,
bei atkarpa   PQ kuri yra statmena skersmeniui AB. Taskuotos srities
 plotas yra    $39\pi$  o skritulio  $C$    $9\pi$.
                 
%\vskip 5cm
  $$\beginpicture
\setcoordinatesystem units <1truemm,1truemm> point at 0 23
\circulararc  180  degrees  from  23  0  center  at  0 0
\circulararc  180  degrees  from  23  0  center  at  17 0
\circulararc  180  degrees  from  11  0  center  at  -6 0
\circulararc  360  degrees  from  11  10.3 center  at  15.4  10.3  
% 24r=y^2
% r=403/92
\putrule from -23 0 to 23 0
\putrule from  11 0 to 11 20.2
\put {A} [t] at -23 -1      \put {B} [t] at  23 -1 
\put {P} [t] at  11 -1      \put {Q} [b] at  11 21
\put {C}     at  15 10
\put {D}     at   0 19
\setshadegrid span <1.5pt>
\setquadratic
\vshade -21 9 11  -5.5 17 22  11 5.5 20 /
\vshade  11 11 20  15 15 17  18 15 15 /
\vshade  11  0 10.3   13  4.5 6.6   16  6  6 /
\vshade  16  6  6     17.5  5.5  6.3     19  5.5  11.5  /
\vshade  19  5.5 12.7  20 5  11    23  0  0 /
  \endpicture$$





\prob %VI6
                                           

                                 
    $ABCD$   yra vienetinis kvadratas,o $P$ ir $Q$  yra tokie 
plokstumos taskai,kad taskas $Q$ yra apie trikampi $BPC$ apibreztojo,
o  $D$ - apie trikampi  $PQA$ apibreztojo apskritimo centras.Raskite
visas galimas atkarpos  $PQ$ ilgio reiksmes. 

                         
               
       
           
 




\prob  %PI3

Keturkampis  $ABCD$ yra trapecija  ($AD{\l}ygiagreti BC$).Tieses  AB
taskas  $P$ yra paimtas taip,kad  $\ang CPD$ didumas yra pats
didziausias.Taskas  $Q$  yra paimtas tieseje  $CD$ tokiu budu,kad 
$\ang BQA$ didumas yra pats didziausias. Zinodami, kad  taskas $P$
yra  atkarpos $AB$ taskas,irodykite, jog $\ang CPD=\ang BQA$.
              




\prob %LA6
Iskilasis keturkampis  $ABCD$    yra ibreztas i apskritima, o $r_a$, $r_b$,
 $r_c$, $r_d$ yra atitinkamai i trikampius  $BCD$, $ACD$, $ABD$, $ABC$
ibreztuju apskritimu spinduliai.Irodykite, kad $r_a+r_c=r_b+r_d$.



\prob   %NO4 (Found on sci.math) 
Jeigu  $a,b,c,d$ yra teigiami sveikieji skaiciai ir $ab=cd$ , tai
irodykite, jog skaicius  $a+b+c+d$ negali buti pirminis skaicius.             





\prob %LA1
Sveikuju skaiciu seka   $a_1,a_2\3,$             sudaroma pagal tokia
taisykle: $a_1=1$, $a_2=2$ ir jei  $n\ge 1$, tai
   $$a_{n+2}=\cases{
      5a_{n+1}-3a_n,&jei $a_n\cdot a_{n+1}$ yra lyginis\cr
      a_{n+1}-a_n,&jei $a_n\cdot a_{n+1}$ yra nelyginis\cr
}$$
  Irodykite,kad koks bebutu $n$, $a_n\neq 0$.


\def\BMD{{\rm BMD}}
\prob  %PI1
Nagrinejame tokia seka :                            
$$
\eqalign{x_1=19,\quad x_2=95,\cr ir
      x_{n+2}=\BMD(x_{n+1}, x_n)+x_n\cr}
$$
      $\BMD(a,b)$    zymime maziausiaji bendraji skaiciu  $a$ ir  $b$.
dalikli. Raskite didziausiaji bendraji skaiciu $x_{1995}$ ir $x_{1996}$   dalikli.
             




\prob  %SU2
    $n$ ir  $k$ yra sveikieji skaiciai ir $1 < k \leq n$.Raskite tokia
n sveikuju skaiciu turincia aibe  $A$ ir toki sveikaji skaiciu $b$,kad
butu patenkintos tokios salygos :
            \item{1.} Jokia skirtingu $k-1$ aibes $A$ elemento
sandauga nesidalija is $b$.
\item{2.} Kiekviena $k$ skirtingu aibes $A$ elementu sandauga dalijasi
 is $b$.
\item{3.} Jei $a,a'\in A$ yra skirtingi aibes $A$ elementai,
tai a nedalo  $a'$.




\prob %VI2
Simboliu  $d(n)$ pazymekime visu skirtingu teigiamu teigiamojo
skaiciaus $n$ dalikliu skaiciu ( iskaitant $1$ ir $n$).Tarkime, kad
$a>1$ ir $n>0$ yra tokie sveikieji skaiciai,kad  $a^n+1$ yra pirminis
skaicius. Irodykite,kad  $$d(a^n-1)\geq n\;.$$







\prob  %PU4
Realieji skaiciai  $x_{1},x_{2},\ldots ,x_{1996}$ pasizymi tokia
savybe, kad imant bet kuri antrojo laipsnio daugianari $W$ ,maziausiai
trys is skaiciu $W(x_{1}),W(x_{2}),\ldots ,W(x_{1996})$  yra lygus.
Irodykite, kad tada ir maziausiai trys is skaiciu
 $x_{1},x_{2},\ldots ,x_{1996}$  irgi yra lygus.         




\prob  %PU1
    $S$ yra sveikuju skaiciu aibe, i kuria ieina  $0 \in S$ ir $1996
\in S$ .Yra zinoma, jog bet kuri sveikoji bet kurio nenulinio
daugianario su $S$ priklausanciais koeficientais saknis irgi priklauso
$S$ .Irodykite, kad tada $-2 \in S.$





\prob %LI2 
 Nagrinekime funkcijas f, apibreztas sveikuju skaiciu aibeje ir
 tenkinancias salyga
                 $f(x)=f(x^2+x+1)$ ,
 koks bebutu sveikasis skaicius x. Raskite 
            a) visas tokias lygines funkcijas;
            b) visas tokias nelygines funkcijas.
                                                                            
     



\prob  %RU1
 Funkcijos  $f(x)=x^n+a_{n-1}x^{n-1}+\cdots+a_1x+a_0$      
$n>1$  grafikas kerta tiese $y=b$ taskuose $B_1$, $B_2$, \dots,~$B_n$,
 ( imant is kaires i desine ), o tiese $y=c$ ($c\ne b$) taskuose
  $C_1$,~$C_2$, \dots,~$C_n$ ( imant is kaires i desine).     
 Jeigu  $P$  yra desiniau tasko $C_n$ esantis tieses  $y=c$ taskas,
tai raskite suma
        $\ctg(\ang P_1C_1P)+\ctg\alpha_2+\cdots+\ctg(\ang P_nC_nP)$.




\prob   %NO2
  Su kuriomis teigiamomis realiomis $a ir b>0$ reiksmemis nelygybe 
   $$x_1\cdot x_2+x_2\cdot x_3\3+x_{n-1}\cdot x_n+x_n\cdot x_1
    \ge x_1^a\cdot x_2^b\cdot x_3^a+x_2^a\cdot x_3^b\cdot 
        x_4^a\3+x_n^a\cdot x_1^b\cdot x_2^a$$
  yra teisinga su visais sveikaisiais $n>2$ ir su visais teigiamais
realiaisiais  skaiciais $x_1\3,x_n>0$?



\prob %LA4
 Begalineje langeliais suliniuotoje lentoje du zaidejai
 paeiliui zenklina langelius.Vienas is ju i langelius irasineja                                                                                $\times$            $\circ$              
 $2\times 2$ , o kitas  - 0 zenklus. Laimi tas, kuris pirmasis savo
simboliais sugeba pazyneti 2 x 2 matmenu kvadrata . Ar pradedantis
zaidejas visada gali laimeti ? 


                                                
\prob %RU5 
Naudodami kiekviena  is astuoniu skaitmenu ~1,~3, 4, 5, 6, 7, 8 ir ~9
po viena karta, sudarome viena trizenkli skaiciu A , du dvizenklius
skaicius $B$~ir~$C$, $B<C$ ir viena vienazenkli skaiciu ~$D$ .
Tie skaiciai yra tokie, kad  $A+D=B+C=143$. Keliais budais yra imanoma
tai padaryti?



\prob %PI2
 Pradzioje olimpiados ziuri komisija sudare 30 nariu. Kiekvienas ziuri
komisijos narys vienus komisijos narius laiko kompetetingais, o visus
kitus likusius (be saves) - nekompetetingais ir savo nuomones toliau
nebekeicia. Kiekvieno posedzio pradzioje yra balsuojama ir tie nariai,
kurie daugiau negu puses komisijos nariu nuomone yra nekompetetingi,
yra pasalinami is ziuri komisijos ligi olimpiados pabaigos. Irodykite,
kad po daugiausiai 15 posedziu nauju pasalinimu jau nebebus.(
Atkreipkite demesi, kad ne vienas neturi teises vertinti saves). 
                               
                                                                      
                                      
                                                              
                                                          
                                                                  
     
                                           
                                                               
                         
                                                                  
                           




\prob %PI4
Keturiose degtuku dezutese yra atitinkamai 38, 45, 61, 70 degtuku. Du
zaidejai paeiliui ima dvi bet kurias degtuku dezutes ir is ju turi
paimti  is kiekvienos nenulini  degtuku skaiciu. Nebegalintis paimti
degtuku zaidejas pralaimi.Kuris zaidejas turi islosimo strategija?                                   
                                                                      
                                                                      
                                             
            




\prob %LA2
                                                               
Ar imanoma visus naturaliuosius skaicius suskirstyti i tokias dvi
klases A ir $B$ ,kad 
       
\item{a)} jokie trys $A$ elementai nesudarytu aritmetines progresijos
                       
\item{b)} Is $B$ elementu nebutu galima isrinkti jokios turincios
nesutampanciu elementu begalines aritmetines progresijos?


\bye

