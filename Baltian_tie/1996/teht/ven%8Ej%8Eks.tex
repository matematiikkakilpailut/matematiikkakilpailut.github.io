\input bw96

\font\cyr=wncyr9

\def\ang{\measuredangle}

\logoandtitle{Baltic Way 1996}

\cyr
\prob %VI5
$\a$ i  $\b \ne 0$ -- lyubye dva ugla mezhdu pryamymi, soderzhawimi
diagonali pravil'nogo 
                       $1996$-ugol'nika. Dokazhite, chto chislo  $\a/\b$     
racional'noe.
%
%i okruzhnost'yu $C$).

\prob %RU2 
Okruzhnost' $C$ (sm. risunok vnizu) kasaet{$ $}sya dvuh polukrugov i 
otrezka          $PQ$, perpendikulyarnogo diametru               
   $AB$. Izvestno, chto plowad' zakrashennoj oblasti ravna  $39\pi$,
a plowad' kruga  $C$ ravna   $9\pi$.
Najdite dlinu diametra                   $AB$.
(Zakrashennaya oblast' ogranichena tremya poluokruzhnostyami
i okruzhnost'yu $C$).

\rm
%\vskip 5cm
  $$\beginpicture
\setcoordinatesystem units <1truemm,1truemm> point at 0 23
\circulararc  180  degrees  from  23  0  center  at  0 0
\circulararc  180  degrees  from  23  0  center  at  17 0
\circulararc  180  degrees  from  11  0  center  at  -6 0
\circulararc  360  degrees  from  11  10.3 center  at  15.4  10.3  
% 24r=y^2
% r=403/92
\putrule from -23 0 to 23 0
\putrule from  11 0 to 11 20.2
\put {A} [t] at -23 -1      \put {B} [t] at  23 -1 
\put {P} [t] at  11 -1      \put {Q} [b] at  11 21
\put {C}     at  15 10
%\put {D}     at   0 19
\setshadegrid span <1.5pt>
\setquadratic
\vshade -21 9 11  -5.5 17 22  11 5.5 20 /
\vshade  11 11 20  15 15 17  18 15 15 /
\vshade  11  0 10.3   13  4.5 6.6   16  6  6 /
\vshade  16  6  6     17.5  5.5  6.3     19  5.5  11.5  /
\vshade  19  5.5 12.7  20 5  11    23  0  0 /
  \endpicture$$

\cyr
\prob %VI6
%                                
%
%                                 
    $ABCD$ -- edinichnyj kvadrat.  $P$ i  $Q$ -- tochki na 
ploskosti, takie chto                   
$Q$ -- centr opisannoj okruzhnosti  $\triangle BPC$,
     $D$ -- centr opisannoj okruzhnosti  $\triangle PQA$.
Najdite vsevozmozhnye znacheniya dliny otrezka  $PQ$. 

                         
%    $ABCD$                          $P$     $Q$                       
%          
%$P$                                 $AQB$     $Q$                       
%         $BPC$                                                    $PQ$.
%
%




\prob  %PI3
Dana trapeciya $ABCD$ ($AD\parallel BC$). Na pryamoj $AB$ vybrana
tochka
$P$, takaya chto 
          $\ang CPD$ imeet naibol'shee vozmozhnoe znachenie.
Na pryamoj                        
$CD$ vybrana tochka $Q$, takaya chto           $\ang BQA$
imeet naibol'shee vozmozhnoe znachenie. Izvestno, chto
                               $P$ lezhit na otrezke
                    $AB$. Dokazhite, chto    $\ang CPD=\ang BQA$.
              




\prob %LA6
%
    $ABCD$ -- vpisannyj vypuklyj chetyrehugol'nik.
                                 $r_a$, $r_b$,
$r_c$, $r_d$ -- radiusy vpisannyh okruzhnostej treugol'nikov
%                                       
$BCD$, $ACD$, $ABD$, $ABC$ sootvet${}$stvenno.                          
Dokazhite, chto $r_a+r_c=r_b+r_d$.



\prob   %NO4 (Found on sci.math) 
  $a,b,c,d$ -- natural'nye chisla,
                     $ab=cd$. Dokazhite, chto            
chislo $a+b+c+d$              --- ne prostoe.





\prob %LA1
Posledovatel'nost' celyh chisel zadana rekurrentnym sootnosheniem:
%$$a_1,a_2\3,\qquad              
$$a_1=1,\qquad
a_2=2, 
\qquad
a_{n+2}=\cases{
      5a_{n+1}-3a_n,&\hbox{\cyr esli $a_n\cdot a_{n+1}$ \cyr
chetnoe}\cr
      a_{n+1}-a_n,&\hbox{\cyr esli $a_n\cdot a_{n+1}$ \cyr nechetnoe}.\cr
}$$
Dokazhite, chto dlya vseh                    $n$\ \ \  $a_n\neq 0$.



\prob  %PI1
Rassmotrim posledovatel'nost'                            
$$
x_1=19,\qquad x_2=95,\qquad
      x_{n+2}=\hbox{\cyr NOK}(x_{n+1}, x_n)+x_n\,,
$$
gde   $\hbox{\cyr NOK}(a,b)$ -- naimen'shee obwee kratnoe chisel  $a$ i    $b$.
Naidite naibol'shij obwij delitel'  chisel    $x_{1995}$ i    $x_{1996}$.
             
\goodbreak



\prob  %SU2
 $n$ i  $k$ -- celye chisla,      $1 < k \leq n$.  Najdite mnozhestvo     $A$,
sostoyawee iz   $n$ celyh chisel, i celoe chislo  $b$,
udovletvoryayuwie sleduyuwim usloviyam:                                      
\item{1.} Ni odno iz proizvedenij    $k-1$ razlichnyh elementov mnozhestva 
$A$
ne delit${}$sya na~$b$.
\item{2.} Proizvedenie lyubyh                  $k$ razlichnyh
elementov
mnozhestva           $A$ delit${}$sya na               $b$.
\nobreak
\item{3.}                  $\forall a,a'\in A$\ \ \ $a$ ne delit${}$sya
na
               $a'$.
\goodbreak



\prob %VI2
Oboznachim cherez          $d(n)$ kolichestvo razlichnyh polozhitel'nyh
delitelej natural'\-nogo chisla                                                      
      \  $n$ \ (vklyu\-chaya  $1$ i $n$). Pust'   $a>1$ i  $n>0$~-- takie
celye chisla, chto                 
     $a^n+1$~prostoe chislo.                         
Dokazhite, chto    $$d(a^n-1)\geq n\;.$$






\prob  %PU4
Vewestvennye chisla             $x_{1},x_{2},\ldots ,x_{1996}$         
takovy, chto dlya lyubogo mnogochlena                  
                   $W$ vtoroj stepeni sredi chisel
%                               
$W(x_{1}),W(x_{2}),\ldots ,W(x_{1996})$ najdut${}$sya po krajnej
mere tri ravnyh. Dokazhite, chto
sredi chisel                          
               $x_{1},x_{2},\ldots ,x_{1996}$           
najdut${}$sya
po krajnej mere tri ravnyh.


\prob  %PU1
Pust'     $S$~-- nekotoroe mnozhestvo celyh chisel, soderzhawee $0$ i
$1996$.
%                               $0 \in S$    
%$1996 \in S$                                              
Izvestno, chto lyuboj celyj koren' nenulevogo mnogochlena s 
koefficientami iz                          
                       $S$ takzhe prinadlezhit mnozhestvu
                 $S$. Dokazhite, chto             $-2 \in S$.





\prob %LI2 
Funkcija $f$ opredelena na mnozhestve celyh chisel
i udovletvoryaet funkcional'\-nomu uravneniyu
                       $$f(x)=f(x^2+x+1) \qquad  x\in {\Bbb Z}\,.$$   
Najdite vse takie a) chetnye b) nechetnye  funkcii.                          
     
\def\ctg{\hbox{\rm ctg}}
\prob  %RU1
Grafik funkcii         $f(x)=x^n+a_{n-1}x^{n-1}+\cdots+a_1x+a_0$       
(gde $n>1$) peresekaet pryamuyu                     
     $y=b$        v tochkah       $B_1$, $B_2$, \dots,~$B_n$ (sleva
napravo),
i pryamuyu    $y=c$ ($c\ne b$) v tochkah     $C_1$,~$C_2$, \dots,~$C_n$      
(sleva napravo).
%$B_1<B_2<\dots<B_n$    
%$C_1<C_2<\dots<C_n$.     $\alpha_i$                               $B_iC_i$
%        $x$-axis ($i=1$,~2, \dots,~$n$).
%             $\cot\alpha_1+\cot\alpha_2+\cdots+\cot\alpha_n$.
Pust' $P$~-- tochka na pryamoj $y=c$, raspolozhennaya sprava ot
$C_n$. Najdite summu
$$
\ctg(\ang B_1C_1P)+\dots+\ctg(\ang B_nC_nP)\,.
$$



\prob   %NO2
Najdite vse vewestvennye polozhitel'nye  $a$ i $b$, takie chto
pri vseh celyh $n>2$ i lyubyh polozhitel'nyh vewestvennyh
$x_1, \dots, x_n$
vypolneno neravenstvo                    
  $$x_1\cdot x_2+x_2\cdot x_3\3+x_{n-1}\cdot x_n+x_n\cdot x_1
    \ge x_1^a\cdot x_2^b\cdot x_3^a+x_2^a\cdot x_3^b\cdot 
        x_4^a\3+x_n^a\cdot x_1^b\cdot x_2^a$$
%                             $n>2$           $x_1\3,x_n>0$?



\prob %LA4
Na beskonechnoj shahmatnoj doske dva igroka po ocheredi
otmechajyut po odnoj (ne pomechennoj ranee) kletke:
odin iz nih stavit krestiki,
drugoj~-- noliki. Vyigryvaet tot, kto pervym zapolnit svoimi
simvolami kvadrat                                                            
                 $2\times 2$.                        
Pravda li, chto nachinayuwij imeet vyigryshnuyu strategiyu?                    
           



\prob   %NO5
%                                                                     
%     $131$, $2662$, $62726$       $\Rev(x)$                                   
%                         $\Rev(179)=971$, $\Rev(9364)=4639$                 
%                 $x$           $\Rev(x)=x$             $x$     $x+\Rev(x)$
Ispol'zuya po odnomu razu kazhduyu iz vos'mi cifr            
$1$, $3$, $4$, $5$, $6$, $7$, $8$, $9$
sostavili odno trehznachnoe chislo $A$, dva dvuznachnyh chisla $B$
i $C$\ \ ($B<C$) i odnoznachnoe chislo~$D$. Pri etom 
$A+D=B+C=143$. Skol'kimi sposobami eto mozhno bylo sdelat'?

\prob %PI2
Pervonachal'no zhyuri olimpiady sostoyalo iz    30 chelovek.
Kazhdyj iz chlenov zhyuri schitaet nekotoryh iz svoih kolleg
kompetentnymi, a ostal'nyh ~--nekompetentnymi, i eti mneniya
ostayut${}$sya neizmennymi. V nachale kazhdogo zasedaniya
prois${}$hodit golosovanie, i te chleny zhyuri, kotorye nekompetentny
s tochki zreniya bolee chem poloviny golosuyuwih navsegda
isklyuchayut${}$sya iz sostava zhyuri. Dokazhite, chto ne bolee chem
cherez $15$ zasedanij sostav zhyuri stabiliziruet${}$sya.
(Podrazumevaet${}$sya,
chto nikto ne golosuet po povodu svoej kompetentnosti).
                           
                                                                      
                                      
                                                              
                                                          
                                                                  
%                                   15         
                                           
                                                               
                         
                                                                  
                           




\prob %PI4
 Est' $4$ kuchi spichek, soderzhawie sootvet${}$stvenno  $ 38, 45, 61,
70$ spichek. Dva igroka po ocheredi vybirayut proizvol'nye dve kuchi
i berut nekotoroe (nenulevoe) kolichestvo spichek iz odnoj kuchi
i nekotoroe (nenulevoe) kolichestvo spichek iz drugoj. Proigryvaet
tot, kto ne mozhet sdelat' hod. Kto iz igrokov imeet vyigryshnuyu strategiyu?
                                     
                                                                      
                                                                      
                                             
            




\prob %LA2
Mozhno li razbit' mnozhestvo vseh natural'nyh chisel na dva
mnozhestva                                                 $A$
i    $B$, 
takih, chto:
\item{a)} nikakie tri chisla iz mnozhestva                     $A$                            
ne obrazuyut arifmeticheskuyu progressiyu.
\item{b)} Mnozhestvo                                                           
   $B$ ne soderzhit ni odnoj beskonechnoj nepostoyannoj
arifmeticheskoj progressii?


\bye

