\input bw96

\logoandtitle{Baltic Way 1996}


\kieli{Polish}

\let\ang=\measuredangle

\prob %VI5
Niech $\a$ i $\b\neq 0$ b\c ed\c a dowolnymi dwoma k\c atami utworzonymi
przez pary prostych zawieraj\c acych przek\c atne $1996$-k\c ata foremnego.
Udowodni\'c, \.ze $\a/\b$ jest liczb\c a wymiern\c a.
                 





\prob %RU2
Na rysunku poni\.zej przedstawione s\c a trzy p\'o\l kola i ko\l o
$C$. Ko\l o $C$ jest styczne do dw\'och p\'o\l koli oraz do prostej
$PQ$, prostopad\l ej do \'srednicy $AB$. Pole powierzchni
zacieniowanego obszaru $D$ jest r\'owne  $39\pi$, a pole ko\l a $C$
wynosi $9\pi$. Obliczy\'c d\l ugo\'s\'c \'srednicy $AB$.
%\vskip 5cm
  $$\beginpicture
\setcoordinatesystem units <1truemm,1truemm> point at 0 23
\circulararc  180  degrees  from  23  0  center  at  0 0
\circulararc  180  degrees  from  23  0  center  at  17 0
\circulararc  180  degrees  from  11  0  center  at  -6 0
\circulararc  360  degrees  from  11  10.3 center  at  15.4  10.3  
% 24r=y^2
% r=403/92
\putrule from -23 0 to 23 0
\putrule from  11 0 to 11 20.2
\put {A} [t] at -23 -1      \put {B} [t] at  23 -1 
\put {P} [t] at  11 -1      \put {Q} [b] at  11 21
\put {C}     at  15 10
\put {D}     at   0 19
\setshadegrid span <1.5pt>
\setquadratic
\vshade -21 9 11  -5.5 17 22  11 5.5 20 /
\vshade  11 11 20  15 15 17  18 15 15 /
\vshade  11  0 10.3   13  4.5 6.6   16  6  6 /
\vshade  16  6  6     17.5  5.5  6.3     19  5.5  11.5  /
\vshade  19  5.5 12.7  20 5  11    23  0  0 /
  \endpicture$$





\prob %VI6
Niech $ABCD$ b\c edzie kwadratem o boku d\l ugo\'sci 1. Niech
$P$ i $Q$  b\c ed\c a takimi punktami na p\l aszczy\'znie, \.ze 
punkt $Q$ jest \'srodkiem okr\c egu opisanego na tr\'ojk\c acie
$BPC$, za\'s  $D$  jest \'srodkiem okr\c egu opisanego na tr\'ojk\c acie
$PQA$. Wyznacz wszystkie mo\.zliwe d\l ugo\'sci odcinka $PQ$. 

                         

\prob  %PI3
Czworok\c at $ABCD$ jest trapezem ($AD\parallel BC$). Niech $P$ b\c
edzie tym z punkt\'ow prostej $AB$, dla kt\'orego miara k\c ata 
$\ang CPD$ jest najwi\c eksza. Podobnie, niech $Q$
b\c edzie tym z punkt\'ow prostej                       
$CD$, dla kt\'orego miara k\c ata  $\ang BQA$
jest najwi\c eksza. Zak\l adaj\c ac, \.ze punkt $P$ nale\.zy do odcinka
$AB$, udowodni\'c \.ze  $\ang CPD=\ang BQA$.
              




\prob %LA6
Niech  $ABCD$ b\c edzie wypuk\l ym czworok\c atem wpisanym w okr\c ag i
niech $r_a$, $r_b$, $r_c$, $r_d$  b\c ed\c a promieniami okr\c eg\'ow
wpisanych odpowiednio w tr\'ojk\c aty $BCD$, $ACD$, $ABD$, $ABC$.
Udowodni\'c, \.ze $r_a+r_c=r_b+r_d$.



\prob   %NO4 (Found on sci.math) 
Udowodni\'c, \.ze je\'sli  $a,b,c,d$ s\c a dodatnimi liczbami ca\l
kowitymi, takimi i\.z $ab=cd$, to $a+b+c+d$ nie jest liczb\c a
pierwsz\c a.





\prob %LA1
Dany jest ci\c ag liczb ca\l kowitych  $a_1,a_2\3,$ taki, \.ze  $a_1=1$,
$a_2=2$, za\'s dla $n\ge 1$ 
   $$a_{n+2}=\cases{
      5a_{n+1}-3a_n,&je\'sli $a_n\cdot a_{n+1}$ jest liczb\c a
parzyst\c a\cr
      a_{n+1}-a_n,&je\'sli $a_n\cdot a_{n+1}$ jest liczb\c a
nieparzyst\c a.\cr
}$$
Udowodni\'c, \.ze dla wszystkich liczb naturalnych $n$ mamy $a_n\neq 0$.

\vfill\eject

\prob  %PI1
Dany jest ci\c ag                             
$$
\eqalign{x_1=19,\quad x_2=95,\cr
      x_{n+2}=NWW(x_{n+1}, x_n)+x_n\cr}
$$
dla $n>1$, gdzie  $NWW(a,b)$ oznacza najmniejsz\c a wsp\'oln\c a
wielokrotno\'s\'c liczb  $a$ i $b$.
Znale\'z\'c najwi\c ekszy wsp\'olny dzielnik liczb
$x_{1995}$ i $x_{1996}$.
             

\goodbreak
%\eject

\prob  %SU2
Niech  $n$ i $k$ b\c ed\c a liczbami ca\l kowitymi, takimi \.ze  
$1 < k \leq n$. Wskaza\'c zbi\'or $A$, sk\l adaj\c acy si\c e z 
$n$ liczb ca\l kowitych oraz liczb\c e ca\l kowit\c a $b$,
spe\l niaj\c ace nast\c epuj\c ace warunki:
                                      
\item{1.} \.Zaden iloczyn  $k-1$ r\'o\.znych element\'ow zbioru $A$ nie jest
podzielny przez $b$.
\item{2.} Ka\.zdy iloczyn $k$ r\'o\.znych element\'ow zbioru  $A$
jest podzielny przez $b$.
\item{3.} Dla dowolnych r\'o\.znych  $a,a'\in A$, $a'$ nie dzieli si\c
e przez  $a$.




\prob %VI2
Dla dodatniej liczby ca\l kowitej $n$ niech  $d(n)$  oznacza 
liczb\c e jej dodatnich dzielnik\'ow (wraz z 1 i $n$).
Niech   $a>1$ i  $n>0$   b\c ed\c a takimi liczbami ca\l kowitymi, \.ze
$a^n+1$   jest liczb\c a pierwsz\c a. Udowodni\'c, \.ze
   $$d(a^n-1)\geq n\;.$$






\prob  %PU4
Liczby rzeczywiste $x_{1},x_{2},\ldots ,x_{1996}$         
maj\c a nast\c epuj\c ac\c a w\l asno\'s\'c: dla dowolnego wielomianu 
$W$  stopnia 2 co najmniej trzy spo\'sr\'od liczb
$W(x_{1}),W(x_{2}),\ldots ,W(x_{1996})$  s\c a r\'owne.
Udowodni\'c, \.ze co najmniej trzy spo\'sr\'od liczb                           
$x_{1},x_{2},\ldots ,x_{1996}$   s\c a r\'owne.




\prob  %PU1
Niech $S$ b\c edzie zbiorem liczb ca\l kowitych, do kt\'orego
nale\.z\c a liczby $0$ i $1996$.   
Po\-nadto ka\.zdy pierwiastek ca\l kowity dowolnego niezerowego
wielomianu o wsp\'o\l czyn\-ni\-kach ze zbioru
$S$ r\'ownie\.z nale\.zy do $S$. Udowodni\'c, \.ze  $-2 \in S.$





\prob %LI2 
Rozwa\.zmy funkcje $f$, okre\'slone na zbiorze liczb ca\l kowitych,
takie \.ze  r\'owno\'s\'c $f(x)=f(x^2+x+1)$ zachodzi dla wszystkich 
liczb ca\l kowitych $x$. Wyznaczy\'c a) wszystkie funkcje parzyste, b)
wszystkie funkcje nieparzyste, spe\l niaj\c ace powy\.zsze warunki.  
                                                                            
     



\prob  %RU1
Wykres funkcji $f(x)=x^n+a_{n-1}x^{n-1}+\cdots+a_1x+a_0$  (gdzie     
$n>1$) przecina prost\c a o r\'ownaniu
$y=b$ w $n$ punktach  $B_1$, $B_2$, \dots,~$B_n$ (od lewej do prawej),
a prost\c a $y=c$ ($c\ne b$) w $n$ punktach $C_1$,~$C_2$, \dots,~$C_n$      
(od lewej do prawej). 
Niech punkt $P$ le\.zy na prostej $y=c$, na prawo od punktu $C_n$.
Znale\'z\'c sum\c e ctg$\ang B_1C_1P\3+\hbox{ctg}\ang B_nC_nP$.



\prob   %NO2
Dla jakich dodatnich liczb rzeczywistych $a,b$ nier\'owno\'s\'c
                    
  $$x_1\cdot x_2+x_2\cdot x_3\3+x_{n-1}\cdot x_n+x_n\cdot x_1
    \ge x_1^a\cdot x_2^b\cdot x_3^a+x_2^a\cdot x_3^b\cdot 
        x_4^a\3+x_n^a\cdot x_1^b\cdot x_2^a$$
jest spe\l niona dla wszystkich liczb ca\l kowitych $n>2$ i dla
wszystkich uk\l ad\'ow dodatnich liczb rzeczywistych  $x_1\3,x_n$?



\prob %LA4
Na niesko\'nczonej szachownicy dwaj gracze na przemian wpisuj\c a na
wolne jeszcze pola po jednym znaku. Jeden gracz u\.zywa 
znaku  $\times$, a drugi znaku  $\circ$. Na pocz\c atku gry
szachownica jest pusta. Wygrywa ten, kto pierwszy
zape\l ni swoimi znakami jaki\'s kwadrat o wymiarach                      
$2\times 2$. Czy gracz rozpoczynaj\c acy rozgrywk\c e mo\.ze
zapewni\'c sobie zwyci\c estwo?
           



\prob   %NO5
U\.zywaj\c ac ka\.zdej z o\'smiu cyfr $1,3,4,5,6,7,8,9$ dok\l adnie
raz, tworzymy 3-cyfrow\c a liczb\c e $A$, dwie 2-cyfrowe liczby $B$ i
$C$, przy czym $B<C$, i jednocyfrow\c a liczb\c e $D$, takie \.ze
$A+D=B+C=143$. Na ile sposob\'ow mo\.zemy to uczyni\'c?

            

%\eject


\prob %PI2
Jury olimpiady sk\l ada si\c e na pocz\c atku z 30 cz\l onk\'ow. 
Ka\.zdy cz\l onek jury uwa\.za niekt\'orych swych koleg\'ow za
kompetentnych, a wszystkich pozosta\l ych za niekompetentnych. 
Opinie w tej sprawie nie ulegaj\c a zmianom. Na pocz\c atku ka\.zdego
zebrania odbywa si\c e g\l osowanie lustracyjne. Po tym 
g\l osowaniu ci cz\l onkowie, kt\'orych za niekompetentnych
uzna\l a wi\c ecej ni\.z po\l owa g\l osuj\c acych w ich sprawie, s\c a usuwani z
jury. 
Udowodni\'c, \.ze po co najwy\.zej $15$ zebraniach
sk\l ad jury przestanie si\c e zmienia\'c. Uwaga: nikt nie g\l osuje
we w\l asnej sprawie, natomiast g\l osuje w sprawach wszystkich
pozosta\l ych cz\l onk\'ow jury.



\prob %PI4
W czterech kupkach le\.zy odpowiednio  38, 45, 61 i 70  zapa\l ek.
Dwaj gracze na przemian wykonuj\c a ruchy polegaj\c ace na wyborze 
dowolnych dw\'och  kupek i usuni\c eciu dowolnej niezerowej liczby
zapa\l ek z pierwszej z nich i dowolnej niezerowej liczby zapa\l ek z
drugiej z nich. Przegrywa ten z graczy, kt\'ory nie mo\.ze wykona\'c
przypadaj\c acego na niego ruchu. Kt\'ory z graczy mo\.ze zapewni\'c
sobie zawsze zwyci\c estwo?


                                                                      
                                                                      
                                             
            




\prob %LA2
Czy mo\.zna podzieli\'c zbi\'or wszystkich liczb ca\l kowitych
dodatnich na dwa roz\l\c a\-czne podzbiory $A$ i $B$, takie \.ze
          
\item{a)} \.zadne trzy elementy zbioru $A$ nie 
tworz\c a ci\c agu arytmetycznego,

\item{b)} z element\'ow zbioru $B$ nie mo\.zna utworzy\'c
niesko\'nczonego i niesta\l ego ci\c agu arytmetycznego?
                                         


\bye

