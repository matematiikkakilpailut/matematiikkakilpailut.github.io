\input bw96

%\def\d{\hbox{$\raise3,8pt\hbox to -6pt{-\hss}\delta$}}
%\def\d{\hbox{$\delta$}}
%\def\th{\hbox{$\theta$}}

\def\d{\hbox{$\raise3,8pt\hbox to -0,65pt{-\hss}\delta$}}
\def\th{\hbox{\sser\hbox to 0pt{p\hss}b}}


\logoandtitle{Eystrasaltskeppnin 1996}

\kieli{Icelandic}


\prob %VI5
  L{\'a}tum $\a$ og $\b\neq 0$ vera horn {\'a} milli einhverra tveggja
l{\'\i}na sem innihalda hornal{\'\i}nur {\'\i} reglulegum $1996$-hyrningi. Sanni{\d} a{\d}
$\a/\b$ s{\'e} r{\ae}{\d} tala.     
                 





\prob %RU2
Hringurinn $C$ {\'a} myndinni h{\'e}r a{\d} ne{\d}an snertir tvo h{\'a}lfhringjanna og l{\'\i}nuna $PQ$ sem
er hornr{\'e}tt {\'a} {\th}verm{\'a}li{\d} $AB$. Flatarm{\'a}l skygg{\d}a sv{\ae}{\d}isins er $39\pi$ og flatarm{\'a}l hringsins $C$ er $9\pi$.
{\'A}kvar{\d}i{\d} {\th}verm{\'a}li{\d} $AB$.
%\vskip 5cm
  $$\beginpicture
\setcoordinatesystem units <1truemm,1truemm> point at 0 23
\circulararc  180  degrees  from  23  0  center  at  0 0
\circulararc  180  degrees  from  23  0  center  at  17 0
\circulararc  180  degrees  from  11  0  center  at  -6 0
\circulararc  360  degrees  from  11  10.3 center  at  15.4  10.3  
% 24r=y^2
% r=403/92
\putrule from -23 0 to 23 0
\putrule from  11 0 to 11 20.2
\put {A} [t] at -23 -1      \put {B} [t] at  23 -1 
\put {P} [t] at  11 -1      \put {Q} [b] at  11 21
\put {C}     at  15 10
%\put {D}     at   0 19
\setshadegrid span <1.5pt>
\setquadratic
\vshade -21 9 11  -5.5 17 22  11 5.5 20 /
\vshade  11 11 20  15 15 17  18 15 15 /
\vshade  11  0 10.3   13  4.5 6.6   16  6  6 /
\vshade  16  6  6     17.5  5.5  6.3     19  5.5  11.5  /
\vshade  19  5.5 12.7  20 5  11    23  0  0 /
  \endpicture$$





\prob %VI6
L{\'a}tum $ABCD$ vera einingarferning og l{\'a}tum $P$ og $Q$ vera
punkta {\'\i} fletinum {\th}annig a{\d} $Q$ s{\'e} mi{\d}punktur
umrita{\d}s hrings {\th}r{\'\i}hyrningsins $BPC$ og $D$ s{\'e}
mi{\d}punktur umrita{\d}s hrings {\th}r{\'\i}hyrningsins $PQA$.
{\'A}kvar{\d}i{\d} �ll m�guleg gildi {\'a} lengd striksins $PQ$. 
        
                

\prob  %PI3
$ABCD$ er trapisa ($AD\parallel BC$). $P$ er punktur {\'a} l{\'\i}nunni
$AB$ {\th}annig a{\d} $\ang CPD$ er eins st{\'o}rt og h{\ae}gt er. $Q$
er punktur {\'a} l{\'\i}nunni $CD$ {\th}annig a{\d} $\ang BQA$ er eins
st{\'o}rt og h{\ae}gt er. N{\'u} vill svo til a{\d} $P$ liggur {\'a} strikinu
$AB$. Sanni{\d} a{\d} $\ang CPD=\ang BQA$.
              




\prob %LA6
L{\'a}tum $ABCD$ vera innrita{\d}an k{\'u}ptan ferhyrning og l{\'a}tum $r_a$, $r_b$,
$r_c$, $r_d$ vera geisla hringjanna sem eru innrita{\d}ir {\'\i}
{\th}r{\'\i}hyrningana $BCD$, $ACD$, $ABD$, $ABC$. Sanni{\d} a{\d} $r_a+r_c=r_b+r_d$.



\prob   %NO4 (Found on sci.math) 
Ef $a,b,c,d$ eru j{\'a}kv{\ae}{\d}ar heilt�lur, $ab=cd$, sanni{\d}
a{\d} {\th}{\'a} er summan $a+b+c+d$ ekki frumtala.              





\prob %LA1
R�{\d} heiltalna $a_1,a_2\3,$ er {\th}annig a{\d} 
$a_1=1$, $a_2=2$ og fyrir $n\ge 1$ gildir:
   $$a_{n+2}=\cases{
      5a_{n+1}-3a_n,&ef $a_n\cdot a_{n+1}$ er sl{\'e}tt tala\cr
      a_{n+1}-a_n,&ef $a_n\cdot a_{n+1}$ er oddatala.\cr
}$$
Sanni{\d} a{\d} fyrir �ll $n$ er $a_n\neq 0$.



\prob  %PI1
Athugum rununa                            
$$
\eqalign{x_1=19,\quad x_2=95,\cr
      x_{n+2}=\LCM(x_{n+1}, x_n)+x_n\cr}
$$
fyrir $n>1$, {\th}ar sem $\LCM(a,b)$ er minnsta sameiginlega margfeldi
$a$ og $b$ (minnsta tala sem a og b ganga upp {\'\i}). Finni{\d}
st{\ae}rsta sameiginlega deili $x_{1995}$ og $x_{1996}$.
             




\prob  %SU2
L{\'a}tum $n$ og $k$ vera heilt�lur, $1 < k \leq n$.
{\'A}kvar{\d}i{\d} mengi $A$ me{\d} $n$ heilt�lum og heilt�lu $b$ sem
uppfylla eftirfarandi skilyr{\d}i:                                      
\item{1.} Ekkert margfeldi $k-1$ {\'o}l{\'\i}kra staka {\'\i} $A$ er
deilanlegt me{\d} $b$.
\item{2.} S{\'e}rhvert margfeldi $k$ {\'o}l{\'\i}kra staka {\'\i} $A$
er deilanlegt me{\d} $b$.
\item{3.} Fyrir �ll {\'o}l{\'\i}k $a,a'$ {\'\i} $A$, gengur $a$ ekki
upp {\'\i} $a'$.




\prob %VI2
T{\'a}knum fj�lda {\'o}l{\'\i}kra j{\'a}kv{\ae}{\d}ra deila
j{\'a}kv{\ae}{\d}rar heilt�lu $n$ me{\d} $d(n)$ ($1$ og $n$ eru
innifalin). L{\'a}tum $a>1$ og $n>0$ vera heilt�lur {\th}annig a{\d}
$a^n+1$ s{\'e} frumtala. Sanni{\d} a{\d} 
   $$d(a^n-1)\geq n\;.$$






\prob  %PU4
Raunt�lurnar $x_{1},x_{2},\ldots ,x_{1996}$ hafa eftirfarandi eiginleika:
{\'I} �llum annars stigs margli{\d}um $W$ eru a.m.k. {\th}rj{\'a}r talnanna $W(x_{1}),W(x_{2}),\ldots
,W(x_{1996})$ {\th}{\ae}r s�mu. Sanni{\d} a{\d} a.m.k. {\th}rj{\'a}r talnanna $x_{1},x_{2},\ldots
,x_{1996}$ s{\'e}u {\th}{\ae}r s�mu.           




\prob  %PU1
$S$ er mengi me{\d} heilt�lum sem inniheldur t�lurnar $0$ og $1996$.
Gerum einnig r{\'a}{\d} fyrir a{\d} s{\'e}rhver heilt�lur{\'o}t
margli{\d}u me{\d} stu{\d}lum {\'u}r $S$ s{\'e} einnig {\'\i} $S$.
Sanni{\d} a{\d} $-2 \in S.$





\prob %LI2 
Athugi{\d} f�llin $f$ sem eru skilgreind fyrir mengi heiltalna
{\th}annig a{\d}                       
   $$f(x)=f(x^2+x+1),$$ 
fyrir allar heilt�lur $x$. {\'A}kvar{\d}i{\d}
   a) �ll jafnst{\ae}{\d} f�ll sem uppfylla skilyr{\d}i{\d},
   b) �ll oddst{\ae}{\d} f�ll sem uppfylla skilyr{\d}i{\d}.
  
                                                                            
     



\prob  %RU1
Graf fallsins $f(x)=x^n+a_{n-1}x^{n-1}+\cdots+a_1x+a_0$ ({\th}ar sem
$n>1$), sker l{\'\i}nuna $y=b$ {\'\i} punktunum $B_1$, $B_2$,
\dots,~$B_n$ (fr{\'a} vinstri til h{\ae}gri), og l{\'\i}nuna $y=c$ ($c\ne
b$) {\'\i} punktunum $C_1$,~$C_2$, \dots,~$C_n$ (fr{\'a} vinstri til h{\ae}gri).
L{\'a}tum $P$ vera punkt {\'a} l{\'\i}nunni $y=c$, til h{\ae}gri
vi{\d} punktinn $C_n$. Finni{\d} summuna  
$\cot(\ang B_1C_1P)\3+\cot(\ang B_nC_nP)$. 




\prob   %NO2
Fyrir hva{\d}a j{\'a}kv{\ae}{\d}ar raunt�lur $a,b$ gildir {\'o}jafnan                    
  $$x_1\cdot x_2+x_2\cdot x_3\3+x_{n-1}\cdot x_n+x_n\cdot x_1
    \ge x_1^a\cdot x_2^b\cdot x_3^a+x_2^a\cdot x_3^b\cdot 
        x_4^a\3+x_n^a\cdot x_1^b\cdot x_2^a$$
fyrir allar heilt�lur $n>2$ og j{\'a}kv{\ae}{\d}ar raunt�lur $x_1\3,x_n>0$?



\prob %LA4
{\'A} {\'o}endanlegu sk{\'a}kbor{\d}i skiptast tveir keppendur {\'a}
a{\d} setja merki {\'\i} einn {\'o}{\'u}tfylltan reit hverju sinni.
Annar notar $\times$, en hinn $\circ$. S{\'a} sem fyrst fyllir $2\times
2$ ferning me{\d} s{\'\i}num merkjum vinnur. Getur s{\'a} sem byrjar
alltaf unni{\d}?                                                    
           



\prob   %RU5
Vi{\d} h�fum {\'a}tta t�lustafi~1,~3, 4, 5, 6, 7, 8 og~9. S{\'e}rhver
{\th}eirra er nota{\d}ur n{\'a}kv{\ae}mlega einu sinni til {\th}ess
a{\d} mynda eina {\th}riggja stafa t�lu~$A$, tv{\ae}r tveggja stafa
t�lur, $B$~og~$C$, $B<C$, og eins stafs t�luna ~$D$. T�lurnar eru
{\th}annig a{\d} $A+D=B+C=143$. {\'A} hve marga vegu er {\th}etta m�gulegt?                                                               




\prob %PI2
{\'I} {\'o}lymp{\'\i}ukeppni eru 30 manns {\'\i} d{\'o}mnefnd {\'\i}
upphafi. S{\'e}rhver d{\'o}mnefndarma{\d}ur telur a{\d} sumir
samnefndarmenn hans s{\'e}u h{\ae}fir en hinir ekki og breytir ekki um
sko{\d}un. {\'I} byrjun hvers fundar er atkv{\ae}{\d}agrei{\d}sla og
{\th}eir sem ekki eru h{\ae}fir a{\d} mati meira en helmings {\th}eirra
sem grei{\d}ir atkv{\ae}{\d}i reknir {\'u}r d{\'o}mnefndinni {\th}a{\d}
sem eftir er af keppninni. Sanni{\d} a{\d} eftir {\'\i} mesta lagi 15
fundi f{\ae}kkar ekki {\'\i} nefndinni. (Athugi{\d} a{\d} enginn
grei{\d}ir atkv{\ae}{\d}i um eigi{\d} {\'a}g{\ae}ti).                                
                           




\prob %PI4
{\'I} fj{\'o}rum eldsp{\'y}tnahr{\'u}gum eru 38, 45, 61, 70
eldsp{\'y}tur. Tveir keppendur velja til skiptis tv{\ae}r hr{\'u}gur
og einhvern fj�lda (st{\ae}rri en n{\'u}ll) {\'u}r annarri hr{\'u}gunni
og einhvern fj�lda (st{\ae}rri en n{\'u}ll) {\'u}r hinni. S{\'a}
leikma{\d}ur sem getur ekki leiki{\d} tapar. Hvor keppenda getur alltaf unni{\d}?                                       


                                                                      
\prob %LA2
Er m�gulegt a{\d} skipta �llum n{\'a}tt{\'u}rulegum t�lum {\'\i} tv�
mengi $A$ og $B$ {\th}annig a{\d}         

\item{a)} Engar {\th}rj{\'a}r t�lur {\'\i} $A$ s{\'e}u samliggjandi
{\'\i} mismunarunu?                            

\item{b)} Engar {\'o}endanlegar mismunarunur s{\'e}u {\'\i} $B$?

\bye

