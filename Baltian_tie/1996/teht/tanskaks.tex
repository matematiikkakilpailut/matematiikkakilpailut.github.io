\input bw96

\logoandtitle{Baltic Way 1996}

\kieli{Danish}

\prob %VI5
Lad $\a$ og $\b$ v{\ae}re to vilk�rlige vinkler mellem linjer, der
indeholder en diagonal i en regul{\ae}r $1996$-kant. Bevis, at $\a/\b$
er et rationalt tal. (En diagonal forbinder to vilk�rlige hj{\o}rner,
der ikke er naboer).

\prob %RU2
P� figuren ses tre halvcirkler og en cirkel.
%\vskip 5cm
  $$\beginpicture
\setcoordinatesystem units <1truemm,1truemm> point at 0 23
\circulararc  180  degrees  from  23  0  center  at  0 0
\circulararc  180  degrees  from  23  0  center  at  17 0
\circulararc  180  degrees  from  11  0  center  at  -6 0
\circulararc  360  degrees  from  11  10.3 center  at  15.4  10.3  
% 24r=y^2
% r=403/92
\putrule from -23 0 to 23 0
\putrule from  11 0 to 11 20.2
\put {A} [t] at -23 -1      \put {B} [t] at  23 -1 
\put {P} [t] at  11 -1      \put {Q} [b] at  11 21
\put {C}     at  15 10
%\put {D}     at   0 19
\setshadegrid span <1.5pt>
\setquadratic
\vshade -21 9 11  -5.5 17 22  11 5.5 20 /
\vshade  11 11 20  15 15 17  18 15 15 /
\vshade  11  0 10.3   13  4.5 6.6   16  6  6 /
\vshade  16  6  6     17.5  5.5  6.3     19  5.5  11.5  /
\vshade  19  5.5 12.7  20 5  11    23  0  0 /
  \endpicture$$
Cirklen $C$ tangerer to af halvcirklerne samt linjen $PQ$, der er
vinkelret p� diameteren $AB$. Det prikkede omr�de har arealet $39\pi$
og cirklen $C$ har arealet $9\pi$. Bestem l{\ae}ngden af diameteren $AB$.


\prob %VI6
Lad $ABCD$ v{\ae}re et enhedskvadrat og lad $P$ og $Q$ v{\ae}re
punkter i planen, s� $Q$ er centrum for trekant $BPC$'s omskrevne
cirkel, og $D$ er centrum for trekant $PQA$'s omskrevne cirkel. Find
samtlige mulige l{\ae}ngder for linjestykket $PQ$. 


\prob  %PI3
$ABCD$ er et trapez, hvor $AD\parallel BC$. $P$ er et punkt p� linjen
gennem $A$ og $B$, s�ledes at $\ang CPD$ er st{\o}rst mulig.
Tilsvarende er $Q$ et punkt p� linjen gennem $C$ og$D$, s�ledes at $\ang BQA$ 
er st{\o}rst mulig. Vis, at hvis $P$ ligger mellem $A$ og$B$, s� g{\ae}lder $\ang CPD=\ang
BQA$.


\prob %LA6
Lad $ABCD$ v{\ae}re en firkant, der har en omskreven cirkel, og lad $r_a$, $r_b$,
$r_c$, $r_d$ v{\ae}re radierne i de indskrevne cirkler for trekanterne
henholdsvis $BCD$, $ACD$, $ABD$, $ABC$. Vis, at $r_a+r_c=r_b+r_d$.


\prob   %NO4 (Found on sci.math) 
Vis, at hvis de hele positive tal $a,b,c$ og $d$ opfylder $ab=cd$, s�
er $a+b+c+d$ ikke et primtal.
             

\prob %LA1
For en f{\o}lge $a_1,a_2\3,a_n,\ldots$ af hele tal g{\ae}lder, at
$a_1=1$, $a_2=2$ og for $n\ge 1$
   $$a_{n+2}=\cases{
      5a_{n+1}-3a_n,&hvis $a_n\cdot a_{n+1}$ er lige\cr
      a_{n+1}-a_n,&hvis $a_n\cdot a_{n+1}$ er ulige.\cr
}$$
Vis, at for ethvert $n$ g{\ae}lder $a_n\neq 0$.


\prob  %PI1
En talf{\o}lge er givet ved                            
$$
\eqalign{x_1=19,\quad x_2=95,\cr
      x_{n+2}=\hbox{mfm}(x_{n+1}, x_n)+x_n\cr}
$$
for $n\ge 2$, hvor mfm$(a,b)$ betyder det mindste f{\ae}lles multiplum
af $a$ og $b$. Find den st{\o}rste f{\ae}lles divisor i $x_{1995}$ og $x_{1996}$.
             

\prob  %SU2
Lad $n$ og $k$ v{\ae}re hele tal, $1 < k \leq n$. Bestem en m{\ae}ngde
$A$ af $n$ hele tal og et helt tal $b$, der opfylder f{\o}lgende betingelser:
                                    
\item{1.} Intet produkt af $k-1$ forskellige elementer fra $A$ er
deleligt med $b$.
\item{2.} Ethvert produkt af $k$ forskellige elementer fra $A$ er
deleligt med $b$.
\item{3.} For alle forskellige $a$ og $a'$ fra $A$ g{\ae}lder, at $a$
ikke g�r op i $a'$.


\prob %VI2
For det hele, positive tal $n$ angiver $d(n)$ antallet af forskellige positive
divisorer i $n$ ($1$ og $n$ selv medregnet). Lad $a>1$ og $n>0$
v{\ae}re hele tal s� $a^n+1$ er et primtal. Vis, at                       

   $$d(a^n-1)\geq n\;.$$


\prob  %PU4
De reelle tal $x_{1},x_{2},\ldots ,x_{1996}$ har f{\o}lgende egenskab:
For ethvert andengradspolynomium $W(x)$ g{\ae}lder det, at mindst 3 af
tallene $W(x_{1}),W(x_{2}),\ldots ,W(x_{1996})$ er lige store.
Bevis, at mindst 3 af tallene $x_{1},x_{2},\ldots ,x_{1996}$ er lige store.


\prob  %PU1
$S$ er en m{\ae}ngde af hele tal, der indeholder tallene $0$ og
$1996$. Antag yderligere, at enhver heltallig rod i et polynomium med
koefficienter fra $S$ (nulpolynomiet medregnes ikke) ogs� tilh{\o}rer $S$.
Vis, at $-2$ tilh{\o}rer $S$.


\prob %LI2 
En funktion $f$ der er defineret for alle hele tal, opfylder
betingelsen $f(x)=f(x^2+x+1)$. 
Find a) alle lige funktioner af denne type
     b) alle ulige funktioner af denne type.    
                                            
                          
\prob  %RU1
Grafen for funktionen $f(x)=x^n+a_{n-1}x^{n-1}+\cdots+a_1x+a_0$, (hvor
$n>1$), vides at sk{\ae}re linjen $y=b$ i punkterne $B_1$, $B_2$,
\dots,~$B_n$ (punkterne er nummereret fra venstre mod h{\o}jre) og
linjen $y=c$ i punkterne $C_1$,~$C_2$, \dots,~$C_n$ (ligeledes
nummereret fra venstre mod h{\o}jre). Lad $P$ v{\ae}re et punkt p�
linjen $y=c$ beliggende til h{\o}jre for $C_n$. Bestem summen $\cot
(\ang B_1C_1P)+\cot (\ang B_2C_2P)+\cdots+\cot (\ang B_nC_nP)$.   
       

\prob   %NO2
For hvilke positive, hele tal $a$ og $b$ er uligheden                   
  $$x_1\cdot x_2+x_2\cdot x_3\3+x_{n-1}\cdot x_n+x_n\cdot x_1
    \ge x_1^a\cdot x_2^b\cdot x_3^a+x_2^a\cdot x_3^b\cdot 
        x_4^a\3+x_n^a\cdot x_1^b\cdot x_2^a$$
sand for alle hele tal $n>2$ og alle positive reelle tal $x_1,x_2\3,x_n$?



\prob %LA4
P� et uendeligt skakbr{\ae}t markerer to spillere p� skift et
umarkeret felt. Den ene bruger $\times$, den anden $\circ$. Den
f{\o}rste, som udfylder et $2\times 2$ kvadrat med sit symbol vinder.
Kan den startende spiller altid vinde?                            
           

\prob   %NO5
Ved at bruge de 8 cifre 1,3,4,5,6,7,8 og 9 netop en gang hver, skal
der dannes et 3-cifret tal $A$, to tocifrede tal $B$ og $C$, $B < C$,
og et et-cifret tal $D$. Der skal g{\ae}lde $A+D=B+C=143$. P� hvor
mange m�der kan dette g{\o}res?                        

            
\prob %PI2
Juryen ved en olympiade har i starten 30 medlemmer. Hvert medlem i
juryen mener at nogle af hans kolleger er kompetente, mens de andre
ikke er det, og disse meninger {\ae}ndre dig ikke. Ved starten af
hvert m{\o}de er der en afstemning om hvert medlem og de medlemmmer,
som er inkompetente if{\o}lge mere end halvdelen af juryen smides ud
af juryen ved slutningen af m{\o}det og kan s�ledes ikke deltage i
nogle af de kommende m{\o}der. Bevis at efter h{\o}jst 15 m{\o}der vil
der ikke forekomme flere eksklusioner. (Bem{\ae}rk at ingen stemmer om
egen kompetence).                                 


\prob %PI4
Fire bunker indeholder henholdsvis 38,45,61 og 71 t{\ae}ndstikker. To
spillere v{\ae}lger p� skift to vilk�rlige bunker og fjerner et
vilk�rligt antal (ikke nul) t{\ae}ndstikker fra den ene bunke og et
vilk�rligt antal (ikke nul) t{\ae}ndstikker fra den anden bunke. Den
spiller, som ikke kan tr{\ae}kke, har tabt. Hvilken spiller har en
vindende strategi?                                    
                                                                      
                                                                      
\prob %LA2
Er det muligt at opdele de positive hele tal i to disjunkte
m{\ae}ngder $A$ og $B$ s�:

\item{a)} Der ikke findes tre tal i $A$, som danner en differensr{\ae}kke

\item{b)} Det er ikke muligt fra $B$ at udtage tal, der danner en
uendelig differensr{\ae}kke.

Der ses bort fra differensr{\ae}kker med differens 0.

\bye

