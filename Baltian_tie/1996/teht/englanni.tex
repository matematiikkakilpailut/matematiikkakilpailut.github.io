\input bw96

\logoandtitle{Baltic Way 1996}

\kieli{English}

\prob %VI5
Let $\a$ be the angle between two lines containing the diagonals of a regular
$1996$-gon, and let $\b\neq 0$ be another such angle. Prove that $\a/\b$ is a
rational number. 

\prob %RU2
%The shaded region~$D$ in the figure below is enclosed between half-circles.
%Moreover it is divided by the circle~$C$. 
In the figure below, you see three half-circles. The circle~$C$ is tangent to
two of the half-circles and to
the line~$PQ$ perpendicular to the diameter $AB$. 
The area of the shaded region is~$39\pi$, 
and the area of the circle~$C$ is~$9\pi$.
Find the length of the diameter~$AB$.
%\vskip 5cm
  $$\beginpicture
\setcoordinatesystem units <1truemm,1truemm> point at 0 23
\circulararc  180  degrees  from  23  0  center  at  0 0
\circulararc  180  degrees  from  23  0  center  at  17 0
\circulararc  180  degrees  from  11  0  center  at  -6 0
\circulararc  360  degrees  from  11  10.3 center  at  15.4  10.3  
% 24r=y^2
% r=403/92
\putrule from -23 0 to 23 0
\putrule from  11 0 to 11 20.2
\put {A} [t] at -23 -1      \put {B} [t] at  23 -1 
\put {P} [t] at  11 -1      \put {Q} [b] at  11 21
\put {C}     at  15 10
%\put {D}     at   0 19
\setshadegrid span <1.5pt>
\setquadratic
\vshade -21 9 11  -5.5 17 22  11 5.5 20 /
\vshade  11 11 20  15 15 17  18 15 15 /
\vshade  11  0 10.3   13  4.5 6.6   16  6  6 /
\vshade  16  6  6     17.5  5.5  6.3     19  5.5  11.5  /
\vshade  19  5.5 12.7  20 5  11    23  0  0 /
  \endpicture$$





\prob %VI6
Let $ABCD$ be a unit square and let $P$ and~$Q$ be points in the plane
such that 
$Q$ is the circumcentre of triangle $BPC$ and $D$ is the circumcentre of
triangle $PQA$. Find all possible values of the length of segment $PQ$. 



\prob  %PI3
$ABCD$ is a trapezium ($AD\parallel BC$). $P$ is the point on the line $AB$
such that $\ang CPD$ is maximal. $Q$ is the point on the line
$CD$ such that $\ang BQA$ is maximal. Given that $P$
lies on the segment $AB$, prove that $\ang CPD=\ang BQA$.





\prob %LA6
Let $ABCD$ be a cyclic convex %inscribed 
quadrilateral and let $r_a$, $r_b$,
$r_c$, $r_d$ be the radii of the circles inscribed in the triangles
$BCD$, $ACD$, $ABD$, $ABC$ respectively.  Prove that 
$r_a+r_c=r_b+r_d$.



\prob   %NO4 (Found on sci.math) 
Let  $a,b,c,d$ be positive integers such that $ab=cd$. Prove that
$a+b+c+d$ is not prime.





\prob %LA1
A sequence of integers $a_1,a_2\3,$ is such that $a_1=1$,
$a_2=2$ and for $n\ge 1$
   $$a_{n+2}=\cases{
      5a_{n+1}-3a_n,&if $a_n\cdot a_{n+1}$ is even\cr
      a_{n+1}-a_n,&if $a_n\cdot a_{n+1}$ is odd.\cr
}$$
Prove that $a_n\neq 0$ for all $n$.



\prob  %PI1
Consider the sequence
$$
\eqalign{x_1=19,\quad x_2=95,\cr
      x_{n+2}=\LCM(x_{n+1}, x_n)+x_n,\cr}
$$
for $n>1$, where $\LCM(a,b)$ means the least common multiple of $a$ and $b$.
Find the greatest common divisor of $x_{1995}$ and $x_{1996}$.
%(R. Breslav)




\prob  %SU2
%Let $n$ and~$k$ be integers, $1 < k \leq n$. Find integers
%$a_1, a_2, \ldots, a_n$ and~$b$ satisfying the following conditions: 
%\item{1.} If $1 \leq i_1, i_2, \ldots, i_{k-1} \leq n$, then
%      $b \eijaa a_{i_1}a_{i_2}\cdots a_{i_{k-1}}$.
%\item{2.} If $1 \leq i_1, i_2, \ldots, i_{k} \leq n$, then
%      $b \jakaa a_{i_1}a_{i_2}\cdots a_{i_{k}}$.
%\item{3.} If $1 \leq i < j \leq n$, $a_i \eijaa a_j$.
Let $n$ and~$k$ be integers, $1 < k \leq n$. Find an integer $b$ and a set
$A$
of $n$ integers satisfying the following conditions: 
\item{(i)} No product of $k-1$ distinct elements of $A$ is divisible by $b$.
\item{(ii)} Every product of $k$ distinct elements of $A$ is divisible by
$b$.
\item{(iii)} For all distinct $a,a'$ in $A$, $a$ does not divide $a'$.




\prob %VI2
Denote by $d(n)$ the number of distinct positive divisors of a positive
integer~$n$ (including $1$ and~$n$). Let $a>1$ and $n>0$ be integers such
that $a^n+1$ is a prime. Prove that 
   $$d(a^n-1)\geq n\;.$$






\prob  %PU4
The real numbers $x_{1},x_{2},\ldots ,x_{1996}$ have the
following property:
for any polynomial $W$ of degree 2 at least three of the numbers
$W(x_{1}),W(x_{2}),\ldots ,W(x_{1996})$ are equal. Prove that at least three
of the numbers $x_{1},x_{2},\ldots ,x_{1996}$ are equal.


\prob  %PU1
Let $S$ be a set of integers 
%such that $0 \in S$ and $1996 \in S$. 
containing the numbers $0$ and $1996$.
Suppose further that any integer root of any
non-zero polynomial with 
coefficients  in $S$ also belongs to $S$. Prove that $-2$ belongs to $S.$





\prob %LI2 
%The functional equation $f(x)=f(x^2+x+1)$ on the set ${\Bbb Z}$ of
%integers is given. Find a) all even functions, b) all odd functions of this
%kind.
Consider the functions $f$ defined on the set of integers such that
   $$f(x)=f(x^2+x+1),$$
for all integers $x$. Find a) all even functions, 
b) all odd functions of this kind.



\prob  %RU1
The graph of the function $f(x)=x^n+a_{n-1}x^{n-1}+\cdots+a_1x+a_0$ (where
$n>1$), intersects the
line~$y=b$ at the points~$B_1$, $B_2$, \dots,~$B_n$ (from left to right),
and the line~$y=c$ ($c\ne b$) at the points~$C_1$,~$C_2$, \dots,~$C_n$
(from left to right).
%with $B_1<B_2<\dots<B_n$ and $C_1<C_2<\dots<C_n$. 
%Let~$\alpha_i$ be the angle between the line~$B_iC_i$
%and the $x$-axis ($i=1$,~2, \dots,~$n$).
Let $P$ be a point on the line $y=c$, to the right to the point $C_n$.
Find the sum $\cot(\ang B_1C_1P)\3+\cot(\ang B_nC_nP)$.




\prob   %NO2
For which positive real numbers $a,b$ does the inequality
  $$x_1\cdot x_2+x_2\cdot x_3\3+x_{n-1}\cdot x_n+x_n\cdot x_1
    \ge x_1^a\cdot x_2^b\cdot x_3^a+x_2^a\cdot x_3^b\cdot 
        x_4^a\3+x_n^a\cdot x_1^b\cdot x_2^a$$
hold for all integers $n>2$ and positive real numbers $x_1\3,x_n$?



\prob %LA4
On an infinite checkerboard, two players alternately mark one 
unmarked cell.
One of them uses $\times$, the other $\circ$.  The first who fills
a $2\times 2$ square with his symbols wins.  Can the player who starts
always win?


\prob %RU5 
Using each of the eight digits~1,~3, 4, 5, 6, 7, 8 and~9 exactly
once, a three-digit number~$A$, two two-digit numbers
$B$~and~$C$, $B<C$, and a one-digit number~$D$ are formed. 
The numbers are such
that $A+D=B+C=143$. In how many ways can this be done?


\prob %PI2
The jury of an olympiad has 30 members in the beginning.  
Each member of the jury
thinks that some of his colleagues are competent, while all the others are
not, and these opinions do not change.
At the beginning of every session a voting takes place, and
those  members who are not competent
in the opinion of more than one half of the voters
are excluded from the jury for the rest
of the olympiad.  Prove that after at most 15 sessions
there will be no more exclusions.
(Note that nobody votes about his own competence.)




\prob %PI4
Four heaps contain 38, 45, 61, and 70 matches respectively. Two players
take turns choosing any two of the heaps and take some
non-zero number of matches from one heap and some
non-zero number of matches from the other heap.
The player who cannot make a move, loses. Which one of the players
has a winning strategy?




\prob %LA2
Is it possible to partition all positive integers into disjoint sets $A$
and $B$ such that

\item{(i)} no three numbers of $A$ form arithmetic progression,

\item{(ii)} no infinite non-constant
arithmetic progression can be formed by numbers
of $B$?


\bye
