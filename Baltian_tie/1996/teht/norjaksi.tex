\input bw96

\logoandtitle{Baltic Way 1996}

\kieli{Norwegian}

\prob %VI5
La $\a$ v{\ae}re vinkelen mellom to linjer som inneholder to
ikke-parallelle diagonaler til en regul{\ae}r $1996$-kant. La $\b$
v{\ae}re en annen slik vinkel.
Vis at $\a/\b$ er et rasjonalt tall. 
(En diagonal forbinder to hj\o{}rner som ikke er naboer.)




\prob %RU2
%The shaded region~$D$ in the figure below is enclosed between half-circles.
%Moreover it is divided by the circle~$C$. 
Figuren nedenfor viser tre halvsirkler med diametere $AP$, $PB$, og $AB$
henholdsvis
Sirkelen~$C$ tangerer to av halvsirklene og linjen~$PQ$ som st{\aa}r
normalt p{\aa} $AB$.
Arealet av det skraverte omr{\aa}det er~$39\pi$, 
og arealet av sirkelen~$C$ er~$9\pi$.
Finn lengden av diameteren~$AB$.
%\vskip 5cm
  $$\beginpicture
\setcoordinatesystem units <1truemm,1truemm> point at 0 23
\circulararc  180  degrees  from  23  0  center  at  0 0
\circulararc  180  degrees  from  23  0  center  at  17 0
\circulararc  180  degrees  from  11  0  center  at  -6 0
\circulararc  360  degrees  from  11  10.3 center  at  15.4  10.3  
% 24r=y^2
% r=403/92
\putrule from -23 0 to 23 0
\putrule from  11 0 to 11 20.2
\put {A} [t] at -23 -1      \put {B} [t] at  23 -1 
\put {P} [t] at  11 -1      \put {Q} [b] at  11 21
\put {C}     at  15 10
%\put {D}     at   0 19
\setshadegrid span <1.5pt>
\setquadratic
\vshade -21 9 11  -5.5 17 22  11 5.5 20 /
\vshade  11 11 20  15 15 17  18 15 15 /
\vshade  11  0 10.3   13  4.5 6.6   16  6  6 /
\vshade  16  6  6     17.5  5.5  6.3     19  5.5  11.5  /
\vshade  19  5.5 12.7  20 5  11    23  0  0 /
  \endpicture$$





\prob %VI6
La $ABCD$ v{\ae}re et kvadrat med sidelengde $1$, og la~$P$ og~$Q$ v{\ae}re
punkter i planet slik at 
$Q$ er sentrum i den omskrevne sirkelen til trekanten $BPC$ og $D$ 
er sentrum i den omskrevne sirkelen til trekanten $PQA$. 
Finn alle mulige verdier for lengden av linjestykket $PQ$. 



\prob  %PI3
$ABCD$ er et trapes ($AD\parallel BC$). $P$ er et punkt p{\aa} linjen $AB$
slik at $\ang CPD$ har st\o{}rst mulig verdi. $Q$ er et punkt p{\aa} linjen
$CD$ slik at $\ang BQA$ has st\o{}rst mulig verdi. Gitt at 
$P$ ligger p{\aa} linjestykket $AB$, vis at $\ang CPD=\ang BQA$.





\prob %LA6
La $ABCD$ v{\ae}re en konveks firkant innskrevet i en sirkel, og la
$r_a$, $r_b$,
$r_c$, $r_d$ v{\ae}re radiene i de innskrevne sirklene til trekantene
$BCD$, $ACD$, $ABD$ og $ABC$ henholdsvis. Vis at $r_a+r_c=r_b+r_d$.



\prob   %NO4 (Found on sci.math) 
La $a,b,c,d$ v{\ae}re positive heltall slik at $ab=cd$.
Vis at $a+b+c+d$ ikke er et primtall.





\prob %LA1
En f\o{}lge $\{a_1,a_2\3,\}$ av hele tall er slik at $a_1=1$,
$a_2=2$, og for $n\ge 1$ gjelder
   $$a_{n+2}=\cases{
      5a_{n+1}-3a_n,&hvis $a_n\cdot a_{n+1}$ er et partall\cr
      a_{n+1}-a_n,&hvis $a_n\cdot a_{n+1}$ er et oddetall.\cr
}$$
Vis at for alle $n$ er $a_n\neq 0$.



\prob  %PI1
Betrakt f\o{}lgen
$$
\eqalign{x_1=19,\quad x_2=95,\cr
      x_{n+2}=\LCM(x_{n+1}, x_n)+x_n,\cr}
$$
for $n>1$, der $\LCM(a,b)$ er minste felles multiplum til $a$ og $b$.
Finn st\o{}rste felles divisor til $x_{1995}$ og $x_{1996}$.
%(R. Breslav)




\prob  %SU2
La $n$ og~$k$ v{\ae}re hele tall, $1 < k \leq n$.
%$a_1, a_2, \ldots, a_n$ and~$b$ satisfying the following conditions: 
%\item{1.} If $1 \leq i_1, i_2, \ldots, i_{k-1} \leq n$, then
%      $b \eijaa a_{i_1}a_{i_2}\cdots a_{i_{k-1}}$.
%\item{2.} If $1 \leq i_1, i_2, \ldots, i_{k} \leq n$, then
%      $b \jakaa a_{i_1}a_{i_2}\cdots a_{i_{k}}$.
%\item{3.} If $1 \leq i < j \leq n$, $a_i \eijaa a_j$.
Finn et heltall $b$ og en mengde $A$
med $n$ distinkte heltall slik at f\o{}lgende betingelser
er oppfylt: 
\item{(i)} Ingen produkter av $k-1$ distinkte elementer fra $A$ 
er delelig med $b$.
\item{(ii)} Alle produkter av $k$ distinkte elementer fra $A$
er delelig med $b$.
\item{(iii)} For alle distinkte $a,a'$ i $A$ vil $a$ ikke dele $a'$.
                             


\prob %VI2
La $d(n)$ v{\ae}re antall distinkte positive divisorer til et positivt
heltall~$n$ (inkludert $1$ og~$n$). La $a>1$ og $n>0$ v{\ae}re hele tall
slik at $a^n+1$ er et primtall. Vis at 
   $$d(a^n-1)\geq n\;.$$






\prob  %PU4
Reelle tall $x_{1},x_{2},\ldots ,x_{1996}$ tilfredsstiller f\o{}lgende
egenskap:
for hvert polynom $W$ av grad 2 er minst tre av tallene
$W(x_{1}),W(x_{2}),\ldots ,W(x_{1996})$ like. Vis at minst tre av tallene 
$x_{1},x_{2},\ldots ,x_{1996}$ er like.


\prob  %PU1
La $S$ v{\ae}re en mengde heltall
%such that $0 \in S$ and $1996 \in S$. 
som inneholder tallene $0$ og $1996$.
Anta videre at enhver heltallig rot i et polynom (forskjellig fra 0) med
koeffienter fra $S$ ogs{\aa} er med i $S$. Vis at $-2$ er med i $S.$





\prob %LI2 
%The functional equation $f(x)=f(x^2+x+1)$ on the set ${\Bbb Z}$ of
%integers is given. Find a) all even functions, b) all odd functions of such 
%kind.
Betrakt funksjonene $f$ definert p{\aa} mengden av de hele tall som er slik
at 
$$f(x)=f(x^2+x+1),$$
for alle heltallige $x$. Finn a) alle jevne funksjoner,
b) alle odde funksjoner av denne typen.



\prob  %RU1
Grafen til fuksjonen $f(x)=x^n+a_{n-1}x^{n-1}+\cdots+a_1x+a_0$ ($n>1$), 
snitter linjen~$y=b$ i punktene~$B_1$, $B_2$, \dots,~$B_n$ (fra h\o{}yre mot
venstre),
og linjen~$y=c$ ($c\ne b$) i punktene~$C_1$,~$C_2$, \dots,~$C_n$
(fra h\o{}yre mot venstre).
%with $B_1<B_2<\dots<B_n$ and $C_1<C_2<\dots<C_n$. 
%Let~$\alpha_i$ be the angle between the line~$B_iC_i$
%and the $x$-axis ($i=1$,~2, \dots,~$n$).
La $P$ v{\ae}re et punkt p{\aa} linjen $y=c$ til h\o{}yre for punktet $C_n$.
Beregn summen $$\cot(\ang B_1C_1P)\3+\cot(\ang B_nC_nP).$$



\prob   %NO2
For hvilke positive hele tall $a,b$ vil ulikheten
$$x_1\cdot x_2+x_2\cdot x_3\3+x_{n-1}\cdot x_n+x_n\cdot x_1
    \ge x_1^a\cdot x_2^b\cdot x_3^a+x_2^a\cdot x_3^b\cdot 
        x_4^a\3+x_n^a\cdot x_1^b\cdot x_2^a$$
holde for alle heltall $n>2$ og alle positive reelle tall $x_1\3,x_n$?



\prob %LA4
To spillere markerer etter tur en umarkert rute p{\aa} et uendelig rutenett.
Den ene bruker $\times$, den andre $\circ$. Den som f\o{}rst fyller
et $2\times 2$ kvadrat med egne symboler vinner. Kan den spilleren som  
begynner alltid vinne?


\prob %RU5 
Ved {\aa} bruke hver av de {\aa}tte sifrene~1,~3, 4, 5, 6, 7, 8 and~9 
n\o{}yaktig en gang kan man danne et tresifret tall~$A$, to tosifrede tall
$B$~og~$C$, $B<C$, og et ensifret tall~$D$. 
Tallene er slik at $A+D=B+C=143$. P{\aa} hvor mange m{\aa}ter kan dette
gj\o{}res?


\prob %PI2
Juryen i en olympiade har opprinnelig 30 medlemmer.              
Hvert medlem av juryen mener at minst en av kollegaene er kompetent, mens
alle de andre ikke er det, og denne oppfatningen endrer seg ikke.
Ved begynnelsen av hver sesjon er det en avstemning, og de medlemmene
som av mer enn halvparten av de \o{}vrige ikke anses som kompetente
blir ekskludert fra juryen.
Vis at det etter h\o{}yst 15 sesjoner ikke bli fler eksklusjoner.





\prob %PI4
Fire hauger inneholder henholdsvis 38, 45, 61 og 70 pinner. To spillere
velger etter tur to av haugene og fjerner et positivt antall pinner
fra den ene haugen, og et positivt antall pinner fra den andre haugen.
Den spilleren som ikke kan utf\o{}re et trekk taper.
Hvem av spillerne har en vinnende strategi?




\prob %LA2
Er det mulig {\aa} dele inn de positive heltall i to disjunkte delmengder $A$
og $B$ slik at

\item{(i)} tre tall fra $A$ aldri danner en aritmetisk f\o{}lge

\item{(ii)} ingen uendelig ikke-konstant aritmetisk f\o{}lge kan formes
av tall i $B$?


\bye
