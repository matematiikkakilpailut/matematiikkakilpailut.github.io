\begin{bwField}{Geometria}

\begin{bwProblem}{G}{1}{04}
Olkoot $AB$ ja $CD$ ympyrän $\mathcal{C}$ kaksi halkaisijaa ja $P$
$\mathcal{C}$:n mielivaltainen piste. Olkoot $R$ ja $S$ pisteestä~$P$
$AB$:lle ja $CD$:lle piirrettyjen kohtisuorien kantapisteet. Osoita,
että janan $RS$ pituus ei riipu pisteen $P$ valinnasta.
\end{bwProblem}


\begin{bwProblem}{G}{2}{02}
Olkoon $P$ sellainen neliön  $ABCD$ sisäpiste, että $PA:PB:PC$ on $1:2:3$. Määritä $\angle BPA$.
\end{bwProblem}


\begin{bwProblem}{G}{3}{01}
Olkoon $E$ kuperan nelikulmion $ABCD$ sisäpiste.
Piirretään nelikulmion ulkopuolelle kolmiot $ABF$, $ BCG$, $ CDH$ ja $
DAI$ siten, että $\triangle ABF \sim \triangle DCE$, $\triangle BCG
\sim \triangle ADE$, $ \triangle CDH \sim \triangle BAE$ ja $\triangle
DAI \sim \triangle CBE$. Olkoot $P$, $Q$, $R$ ja $S$ pisteen $E$
projektiot suorilla $AB$, $BC$, $CD$ ja $DA$, tässä järjestyksessä.
Todista, että jos $PQRS$ on jännenelikulmio, niin
\begin{equation*}
  EF \cdot CD = EG \cdot DA = EH \cdot AB = EI \cdot BC \,.
\end{equation*}

\end{bwProblem}


\begin{bwProblem}{G}{4}{08}
Kolmion $ABC$ sisään piirretty ympyrä sivuaa kolmion sivuja $BC$,
$CA$ ja $AB$ pisteissä $D$, $E$ ja $F$, tässä järjestyksessä. Olkoon
$G$ se sisään piirretyn ympyrän piste, jolle $FG$ on ympyrän
halkaisija. Suorat $EG$ ja $FD$ leikkaavat pisteessä $H$. Todista,
että $CH\parallel AB$.
\end{bwProblem}

\begin{bwProblem}{G}{5}{09}

Olkoon $ABCD$ kupera nelikulmio, jossa $\angle ADB=\angle BDC$.
Oletetaan, että sivun $AD$ piste $E$ toteuttaa yhtälön 
$$AE\cdot ED+BE^2=CD\cdot AE.$$
Osoita, että $\angle EBA=\angle DCB$.

\end{bwProblem}
\end{bwField}