\begin{bwField}{Lukuteoria}

\begin{bwProblem}{N}{1}{01}
Olkoon $a$ kokonaisluku.
Määritellään jono $x_0,x_1,\ldots$ asettamalla $x_0=a$, $x_1=3$
ja
   $$
x_n=2x_{n-1}-4x_{n-2}+3,
   $$
kun $n>1$. Määritä suurin kokonaisluku~$k_a$, jolla on olemassa
sellainen alkuluku~$p$, että $p^{k_a}$ jakaa luvun~$x_{2011}-1$.
\end{bwProblem}

\begin{bwProblem}{N}{2}{02}
Määritä kaikki sellaiset positiviset kokonaisluvut~$d$, että jos $d$
jakaa kokonaisluvun~$n$, niin $d$ jakaa myös jokaisen kokonaisluvun~$m$,
jonka numerot ovat jossain järjestyksessä samat kuin luvun~$n$
numerot.
\end{bwProblem}

\begin{bwProblem}{N}{3}{07}
Määritä kaikki alkulukuparit~$(p,q)$, joille sekä $p^2+q^3$ että $q^2+p^3$
ovat kokonaisluvun neliöitä.
\end{bwProblem}

\begin{bwProblem}{N}{4}{15}
Olkoon $p\neq 3$ alkuluku.  Osoita, että on olemassa toistoton
positiivisten kokonaislukujen \break $x_1,x_2,\ldots,x_p$ 
aritmeettinen jono, jonka jäsenten tulo on kokonaisluvun kuutio.
\end{bwProblem}

\begin{bwProblem}{N}{5}{09}
Kokonaislukua $n\ge 1$ kutsutaan \emph{tasapainoiseksi}, jos sillä on
parillinen määrä eri alkutekijöitä.  Todista, että on olemassa äärettömän
monta sellaista positiivista kokonaislukua~$n$, että täsmälleen
kaksi luvuista $n$, $n+1$, $n+2$ ja $n+3$ on tasapainoisia.
\end{bwProblem}

\end{bwField}

