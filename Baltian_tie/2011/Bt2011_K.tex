\begin{bwField}{Kombinatoriikka}

\begin{bwProblem}{C}{1}{05}
Olkoon $n$ positiivinen kokonaisluku.  Todista, että
niiden suorien lukumäärä, jotka kulkevat origon ja täsmälleen
yhden toisen pisteen $(x,y)$ kautta, missä $x$ ja $y$ ovat
kokonaislukuja, $0\le x\le n$ ja $0\le y\le n$, on
vähintään $n^2/4$.
\end{bwProblem}

\begin{bwProblem}{C}{2}{06}
Tarkastellaan $15$-alkioista joukkoa 
$T=\{10a+b \,|\, a,b\in\ZZ,\,1\le a<b\le 6\}$.  
Olkoon $S$ joukon~$T$ osajoukko, jossa kaikki
kuusi numeroa $1,2,\ldots,6$ esiintyvät mutta joka ei sisällä
kolmikkoa, jossa esiintyisivät kaikki nämä $6$~numeroa.  Määritä
joukon~$S$ suurin mahdollinen koko.
\end{bwProblem}

\begin{bwProblem}{C}{3}{07}
Greifswaldissa on koulut $A$, $B$ ja $C$, joista kutakin käy
ainakin yksi oppilas.  Kustakin oppilaskolmikosta, joista yksi
käy koulua~$A$, toinen koulua~$B$ ja kolmas koulua~$C$,
jotkin kaksi tuntevat toisensa ja jotkin kaksi eivät tunne
toisiaan.  Todista, että ainakin yksi seuraavista pätee:
\begin{itemize}
\item[--]
Jokin koulun $A$ oppilas tuntee kaikki koulun~$B$ oppilaat.

\item[--]
Jokin koulun $B$ oppilas tuntee kaikki koulun~$C$ oppilaat.

\item[--]
Jokin koulun $C$ oppilas tuntee kaikki koulun~$A$ oppilaat.
\end{itemize}

\end{bwProblem}

\begin{bwProblem}{C}{4}{08}
Väritetään $m\times n$-ruudukon ruudut mustiksi ja valkoisiksi.
Värityksen sanotaan olevan \emph{pätevä}, jos se täyttää
seuraavat ehdot:
\begin{itemize}
\item[--] 
Kaikki reunaruudut ovat mustia.

\item[--]
Mitkään neljä $2\times 2$-ruudukon muodostavaa ruutua eivät ole
samanvärisiä.

\item[--]
Mitkään neljä $2\times 2$-ruudukon muodostavaa ruutua eivät ole
niin väritetyt, että vain kulmittain toisiaan koskettavat ruudut
ovat samanvärisiä.
\end{itemize}

Millä $m\times n$-ruudukoilla, jossa $m,n\ge 3$, on olemassa
pätevä väritys?
\end{bwProblem}

\goodbreak

\begin{bwProblem}{C}{5}{12}
Kaksi pelaajaa pelaa seuraavaa kokonaislukupeliä.
Aluksi luku on $2011^{2011}$, ja pelaajat siirtävät vuorotellen.
Jokaisella siirrolla lukua voi vähentää kokonaisluvulla,
joka on vähintään 1 ja korkeintaan 2010, tai luvun voi jakaa 2011:llä
ja pyöristää alaspäin lähimpään kokonaislukuun. Pelaaja, joka
ensimmäisenä päätyy epäpositiiviseen kokonaislukuun, voittaa. 
Kummalla pelaajista on voittostrategia?
\end{bwProblem}
  
\end{bwField}
