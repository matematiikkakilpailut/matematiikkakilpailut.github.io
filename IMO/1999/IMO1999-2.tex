\documentstyle[twoside,12pt]{article}
\input ghmac.tex
\textheight 9in
\textwidth 6in
\hoffset -0.4in
\voffset -1in
\setlength{\headsep}{2cm}
\pagestyle{empty}

\begin{document}
%\vskip-6cm
%\bigskip
%\input epsf
%\epsfxsize=3cm
%\epsfbox{a5f1.eps}


\hrule
\medskip
Finnish version \hspace{8.2cm} Finnish version
\bigskip
\bigskip


\centerline{\large \bf TOINEN P\"AIV\"A}
\medskip
\centerline{\large Bukarest, 17.\thinspace 7.\thinspace 1999}
\bigskip
\bigskip

\hspace{8.5cm}\begin{tabular}{l}
Kesto: 4 tuntia 30 minuuttia.\\
Jokainen teht\"av\"a on  7 pisteen arvoinen.
\end{tabular}
\vspace{2cm}

\noindent {\bf Teht\"av\"a 4.}
\bigskip

\noindent M\"a\"arit\"a kaikki sellaiset positiivisten kokonaislukujen
parit $(n,p)$, ett\"a

\smallskip

\indent \indent $p$ on alkuluku,

\smallskip

\indent\indent $n\leq 2p$ \ ja

\smallskip

\indent\indent $(p-1)^n + 1$ on jaollinen luvulla $n^{p-1}$.

\bigskip
\bigskip
\noindent{\bf Teht\"av\"a 5.}
\bigskip

\noindent Ympyr\"at $\Gamma_1$ ja $\Gamma_2$ sis\"altyv\"at
ympyr\"a\"an $\Gamma$ ja sivuavat ympyr\"a\"a $\Gamma$
eri pisteiss\"a $M$ ja $N$. Ympyr\"a
$\Gamma_1$ kulkee ympyr\"an $\Gamma_2$ keskipisteen
kautta. Ympyr\"oiden $\Gamma_1$ ja
$\Gamma_2$ leikkauspisteiden kautta kulkeva suora leikkaa 
ympyr\"an $\Gamma$ pisteiss\"a $A$ ja $B$. 
Suorat $MA$ ja $MB$ leikkaavat ympyr\"an $\Gamma_1$
pisteiss\"a $C$ ja $D$.

\smallskip
\noindent Todista, ett\"a suora $CD$ sivuaa ympyr\"a\"a $\Gamma_2$.

\bigskip
\bigskip

\noindent{\bf Teht\"av\"a 6.}
\bigskip

\noindent M\"a\"arit\"a kaikki sellaiset kuvaukset \
$f:\RR \to \RR$, ett\"a jokaisella
$x,y\in\RR$ on voimassa yht\"al\"o
\[
f(x-f(y))=f(f(y)) + x\,f(y)+f(x)-1.
\]

\end{document}
