\documentstyle[twoside,12pt]{article}
\input ghmac.tex
\textheight 9in
\textwidth 6in
\hoffset -0.4in
\voffset -1in
\setlength{\headsep}{2cm}
\pagestyle{empty}

\begin{document}
%\vskip-6cm
%\bigskip
%\input epsf
%\epsfxsize=3cm
%\epsfbox{a5f1.eps}

\hrule
\medskip
Finnish version \hspace{8.2cm} Finnish version
\bigskip
\bigskip

\centerline{\large \bf ENSIMM\"AINEN P\"AIV\"A}
\medskip
\centerline{\large Bukarest, 16.\thinspace 7.\thinspace 1999}
\bigskip
\bigskip

\hspace{8.5cm}\begin{tabular}{l}
Kesto: 4 tuntia 30 minuuttia.\\
Jokainen teht\"av\"a on  7 pisteen arvoinen.
\end{tabular}
\vspace{2cm}



\noindent{\bf Teht\"av\"a 1.}

\bigskip

\noindent M\"a\"arit\"a kaikki \"a\"arelliset tasojoukot $S$,
joissa on v\"ahint\"a\"an kolme pistett\"a ja jotka t\"aytt\"av\"at
seuraavan ehdon:

\medskip
{\advance\leftskip by 1cm \advance \rightskip by 1cm \parindent=0pt
kun $A$ ja $B$ ovat joukon $S$ kaksi eri pistett\"a,
joukko $S$ on symmetrinen janan $AB$
keskinormaalin suhteen.\par}

\bigskip
\bigskip
\noindent{\bf Teht\"av\"a 2.}

\bigskip

\noindent Olkoon $n$ kiinte\"a kokonaisluku, jolle $n\geq 2$.
\medskip
\begin{description}

\item{(a)} M\"a\"arit\"a  pienin sellainen vakio $C$,
ett\"a kaikilla 
reaalisilla $x_1,\ldots,x_n\geq 0$ p\"atee ep\"ayht\"al\"o
\[
\sum_{1\leq i < j \leq n} x_i x_j(x_i^2 + x_j^2) \leq C\Big(
\sum_{1\leq i \leq n} x_i\Big)^4.
\]

\medskip

\item{(b)} 
M\"a\"arit\"a, milloin yht\"asuuruus on voimassa, kun $C$ on kuten yll\"a.
\end{description}
\bigskip
\bigskip

\noindent{\bf Teht\"av\"a 3.}

\bigskip

\noindent Tarkastellaan $n\times n$-lautaa, miss\"a $n$ on kiinte\"a \
positiivinen parillinen kokonaisluku. \
Lauta koostuu $n^2$ yksikk\"oruudusta. Kahden eri ruudun sanotaan\
olevan \textit{vierekk\"aiset}, jos niill\"a on yhteinen sivu.

\smallskip
\noindent Laudan $N$ ruutua merkit\"a\"an niin, ett\"a jokaisen
laudan (merkityn tai merkitsem\"att\"om\"an) ruudun
vieress\"a on v\"ahint\"a\"an yksi  merkitty ruutu.

\smallskip
\noindent M\"a\"arit\"a luvun $N$ pienin mahdollinen arvo.

 \end{document}
