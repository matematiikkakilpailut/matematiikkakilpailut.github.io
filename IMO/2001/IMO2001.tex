\documentclass[12pt]{article}

\leftmargin=0pt
\topmargin=0pt
\headheight=0in
\headsep=0in
\oddsidemargin=0pt
\textwidth=6.5in
\textheight=8.5in

\begin{document}

\pagestyle{empty}


\begin{center}
{Finnish version \hfill Suomenkielinen versio} \\
\bigskip
{\Large 42. Kansainv\"aliset matematiikkaolympialaiset} \\
{\large Washington, DC, USA} \\
\medskip
{\large
\textbf{Ensimm\"ainen p\"aiv\"a -- 8.\ hein\"akuuta 2001}\\

\textbf{9.00 -- 13.30}\\

}
\end{center}

\vspace{0.5in}
\begin{enumerate}
\item[\textbf{1.}]
Ter\"av\"akulmaisen kolmion $ABC$  ymp\"ari piirretyn ympyr\"an
keskipiste on $O$. $A$:sta $BC$:lle piirretyn korkeusjanan
kantapiste on $P$.

\smallskip
Oletetaan, ett\"a $\angle BCA\ge \angle ABC+30^{\circ}$.

\smallskip
Todista, ett\"a $\angle CAB+\angle COP<90^{\circ}$.

\vspace{0.5in}

\item[\textbf{2.}]
Todista, ett\"a
\[
\frac{a}{\sqrt{a^2+8bc}} + \frac{b}{\sqrt{b^2+8ca}} +
\frac{c}{\sqrt{c^2+8ab}}
\geq 1
\]
kaikilla positiivisilla reaaliluvuilla $a, b$ ja $c$.

\vspace{0.5in}
\item[\textbf{3.}]
Matematiikkakilpailuun osallistui 21 tytt\"o\"a ja 21 poikaa.
\begin{itemize}
\item
Jokainen kilpailija ratkaisi enint\"a\"an kuusi teht\"av\"a\"a.
\item
Jokaista tytt\"o\"a ja poikaa kohden oli ainakin yksi teht\"av\"a, jonka
molemmat ratkaisivat.
\end{itemize}
Todista, ett\"a oli ainakin yksi teht\"av\"a, jonka ratkaisi ainakin kolme
tytt\"o\"a ja ainakin kolme poikaa.
\end{enumerate}
\vspace{1in}\noindent \textit{Jokaisen teht\"av\"an maksimipistem\"a\"ar\"a
on seitsem\"an.}
\pagebreak


\begin{center}
{Finnish version \hfill Suomenkielinen versio}  \\
\bigskip
{\Large 42. Kansainv\"aliset matematiikkaolympialaiset} \\
{\large Washington, DC, USA} \\
\medskip
{\large
\textbf{Toinen p\"aiv\"a -- 9.\ hein\"akuuta 2001}\\

\textbf{9.00 -- 13.30}\\

}
\end{center}

\vspace{0.5in}
\begin{enumerate}
\item[\textbf{4.}]
Olkoon $n$ pariton kokonaisluku, $n>1$, ja olkoot  $k_1$, $k_2$, $\dots$,
$k_n$ annettuja kokonaislukuja. Olkoon jokaiselle joukon
$\{1,\,2,\,\dots,\,n\}$ $n!$:lle permutaatiolle $a=(a_1,\, a_2,\, \dots,\,
a_n)$
\[
S(a) = \sum_{i=1}^n k_i a_i.
\]
Osoita, ett\"a on olemassa ainakin kaksi permutaatiota $b$ ja  $c$,
$b \neq c$, siten, ett\"a  $S(b) - S(c)$ on jaollinen $n!$:lla.

\vspace{0.5in}
\item[\textbf{5.}]
Olkoon kolmiossa $ABC$  kulman $BAC$ puolittaja $AP$, miss\"a
$P$ on sivulla  $BC$, ja olkoon  kulman $ABC$ puolittaja $BQ$, miss\"a $Q$ on
sivulla $CA$.

Tiedet\"a\"an, ett\"a  $\angle BAC = 60^{\circ}$ ja ett\"a
$AB+BP = AQ+QB$.

Mitk\"a ovat kolmion $ABC$ mahdolliset kulmat?

\vspace{0.5in}
\item[\textbf{6.}]
Olkoot  $a$, $b$, $c$ ja $d$  kokonaislukuja ja  $a>b>c>d>0$. Oletetaan,
ett\"a \[
ac+bd = (b+d+a-c)(b+d-a+c).
\]
Osoita, ett\"a $ab+cd$ ei ole alkuluku.

\end{enumerate}
\vspace{1in}
\noindent \textit{Jokaisen teht\"av\"an maksimipistem\"a\"ar\"a on
seitsem\"an.}
\end{document}

