\documentclass[12pt]{article}

\leftmargin=0pt
\topmargin=0pt
\headheight=0in
\headsep=0in
\oddsidemargin=0pt
\textwidth=6.5in
\textheight=8.5in

\begin{document}

\pagestyle{empty}


\begin{center}
% The left side must be in English. The right side should be in
% the language into which the paper is translated.
{ English version \hfill English version } \\
\bigskip
{\Large 42nd International Mathematical Olympiad} \\
{\large Washington, DC, United States of America} \\
\medskip
{\large
\textbf{First Day --- July 8, 2001} \\
\textbf{9:00 AM -- 1:30 PM}\\
}
\end{center}


\vspace{0.5in}

\begin{enumerate}
\item[\textbf{1.}] Let $ABC$ be an acute-angled triangle with
circumcentre $O$. Let $P$ on $BC$ be the foot of the altitude from
$A$.

\smallskip
Suppose that $\angle BCA \geq \angle ABC + 30^{\circ}$.

\smallskip
Prove that $\angle CAB + \angle COP < 90^{\circ}.$

\vspace{0.5in}

\item[\textbf{2.}] Prove that
\[
\frac{a}{\sqrt{a^2+8bc}} + \frac{b}{\sqrt{b^2+8ca}} + \frac{c}{\sqrt{c^2+8ab}}
\geq 1
\]
for all positive real numbers $a, b$ and $c$.

\vspace{0.5in}

\item[\textbf{3.}] Twenty-one girls and twenty-one boys
took part in a mathematical contest.
\begin{itemize}
\item
Each contestant solved at most six problems.
\item
For each girl and each boy, at least one problem was solved by both of them.
\end{itemize}
Prove that there was a problem that was solved by at least three
girls and at least three boys.
\end{enumerate}

\vspace{1in} \noindent \textit{Each problem is worth seven
points.}

 \pagebreak

\begin{center}
% The left side must be in English. The right side should be in
% the language into which the paper is translated.
{ English version \hfill English version } \\
\bigskip
{\Large 42nd International Mathematical Olympiad} \\
{\large Washington, DC, United States of America} \\
\medskip
{\large
\textbf{Second Day --- July 9, 2001} \\
\textbf{9:00 AM -- 1:30 PM}\\
}
\end{center}

\vspace{0.5in}

\begin{enumerate}
\item[\textbf{4.}] Let $n$ be an odd integer greater than 1,
and let $k_1, k_2, \dots, k_n$ be given integers. For each of the
$n!$ permutations $a=(a_1, a_2, \dots, a_n)$ of $1, 2, \dots, n$,
let
\[
S(a) = \sum_{i=1}^n k_i a_i.
\]
Prove that there are two permutations $b$ and $c$, with $b \neq
c$, such that $n!$ is a divisor of $S(b) - S(c)$.

\vspace{0.5in}

\item[\textbf{5.}] In a triangle $ABC$, let $AP$
bisect $\angle BAC$, with $P$ on $BC$, and let $BQ$ bisect $\angle
ABC$, with $Q$ on $CA$.

It is known that $\angle BAC = 60^{\circ}$ and that $AB+BP =
AQ+QB$.

What are the possible angles of triangle $ABC$?

\vspace{0.5in}

\item[\textbf{6.}]
Let $a, b, c, d$ be integers
with $a>b>c>d>0$. Suppose that
\[
ac+bd = (b+d+a-c)(b+d-a+c).
\]
Prove that $ab+cd$ is not prime.

\end{enumerate}

\vspace{1in} \noindent \textit{Each problem is worth seven
points.}

\end{document}
