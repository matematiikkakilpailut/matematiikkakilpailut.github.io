\documentclass[a4paper,12pt,fleqn]{article}
\usepackage[finnish]{babel}
\usepackage{amssymb}

\pagestyle{empty}

\parskip=6pt plus0pt minus0pt
\parindent=0pt
\mathsurround=1pt
\mathindent=4em
\setlength{\textheight}{22cm}
\setlength{\textwidth}{15cm}
\setlength{\oddsidemargin}{0cm}
\setlength{\evensidemargin}{0cm}


\begin{document}

\vspace*{4.5cm}

\begin{center}
Ensimm\"ainen p\"aiv\"a: 24.\thinspace 7.\thinspace 2002
\end{center}

\vspace*{1cm}
Kesto: 4 tuntia 30 minuuttia.

\vspace*{-4.5\parskip}
\begin{flushright}
Version: Finnish
\end{flushright}

\vspace*{-1.9\parskip}
Kustakin teht\"av\"ast\"a voi saada korkeintaan  7 pistett\"a.


\vspace*{7mm}
{\bf Teht\"av\"a 1}

Olkoon $n$ positiivinen kokonaisluku.  Olkoon $T$ niiden tason
pisteiden $(x,y)$ joukko, 
joille $x$ ja $y$ ovat ep\"anegatiivisia kokonaislukuja
ja $x+y<n$.  Jokainen joukon $T$ piste v\"aritet\"a\"an punaiseksi tai
siniseksi.  Jos piste $(x,y)$ on punainen, niin on my\"os jokainen
piste $(x',y')\in T$, miss\"a $x'\le x$ ja $y'\le y$. Kutsutaan 
$X$-joukoksi joukkoa, joka koostuu $n$ sinisest\"a pisteest\"a, 
joilla on eri $x$-koordinaatit, sek\"a
$Y$-joukoksi joukkoa, joka koostuu $n$ sinisest\"a pisteest\"a, 
joilla on eri $y$-koordinaatit.  Todista, ett\"a $X$-joukkoja on yht\"a monta
kuin $Y$-joukkoja.

\vspace*{7mm}
{\bf Teht\"av\"a 2}

Olkoon $BC$ \quad $O$-keskisen ympyr\"an $\Gamma$ halkaisija.
Olkoon $A$ ympyr\"an $\Gamma$ piirin 
piste, jolle $0^\circ<\measuredangle AOB<120^\circ$.
Olkoon $D$ pistett\"a $C$ sis\"alt\"am\"att\"om\"an kaaren $AB$ keskipiste.
Pisteen $O$ kautta kulkeva, janan $DA$ kanssa yhdensuuntainen suora
leikkaa suoran $AC$ pisteess\"a $J$.  Janan $OA$ keskinormaali
leikkaa ympyr\"an $\Gamma$ piirin pisteiss\"a $E$ ja $F$.  Todista, ett\"a $J$
on kolmion $CEF$ sis\"a\"anpiirretyn ympyr\"an keskipiste. 

\vspace*{7mm}
{\bf Teht\"av\"a 3}

Etsi kaikki  kokonaislukuparit $(m,n)$, $m,n\ge 3$, joille on
olemassa \"a\"arett\"om\"an monta sellaista positiivista
kokonaislukua $a$, ett\"a
   $${a^m+a-1\over a^n+a^2-1}$$
on kokonaisluku.

\newpage

\vspace*{4.5cm}

\begin{center}
Toinen p\"aiv\"a: 25.\thinspace 7.\thinspace 2002
\end{center}

\vspace*{1cm}
Kesto: 4 tuntia 30 minuuttia.

\vspace*{-4\parskip}
\begin{flushright}
Version: Finnish
\end{flushright}

\vspace*{-2\parskip}
Kustakin teht\"av\"ast\"a voi saada korkeintaan  7 pistett\"a.

\vspace*{7mm}
{\bf Teht\"av\"a 4}

Olkoon $n$ lukua $1$ suurempi kokonaisluku.
Luvun $n$ positiiviset tekij\"at ovat $d_1,d_2,\ldots,d_k$, miss\"a
	$$1=d_1<d_2<\cdots<d_k=n.$$  
Merkit\"a\"an $D=d_1d_2+d_2d_3+\,\cdots\,+d_{k-1}d_k$.

\begin{itemize}

\item[a)] Todista, ett\"a $D<n^2$.

\item[b)] M\"a\"arit\"a kaikki luvut $n$, joille $D$ on luvun $n^2$ tekij\"a.

\end{itemize}

\vspace*{7mm}
{\bf Teht\"av\"a 5}

Etsi kaikki funktiot $f:{\Bbb R}\to{\Bbb R}$, joille p\"atee
kaikilla $x,y,z,t\in{\Bbb R}$
   $$(f(x)+f(z))(f(y)+f(t))=f(xy-zt)+f(xt+yz).$$


\vspace*{7mm}
{\bf Teht\"av\"a 6}

Olkoot $\Gamma_1,\Gamma_2,\ldots,\Gamma_n$ tason 
yksikk\"os\"ateisi\"a ympyr\"oit\"a, miss\"a $n\ge 3$. 
Olkoot n\"aiden keskipisteet $O_1,\ldots,O_n$.  
Oletetaan, ett\"a mik\"a\"an suora 
ei kohtaa useampaa kuin kahta ympyr\"oist\"a.
Todista, ett\"a
   $$\sum_{1\le i<j\le n}{1\over |O_iO_j|}\le {(n-1)\pi\over 4}.$$


\end{document}



