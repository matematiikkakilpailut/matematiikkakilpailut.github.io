\input makrot
\input def
\input laskut
\input pictex

\font\isosser=cmss12
\centerline{\isosser 9. pohjoismainen kilpailu 15.\:3.\:1995}\bigskip

\line{
   \vtop{\hsize=0.63\hsize
\harj  Olkoon $AB$ $O$-keskisen ympyr�n halkai\-sija. 
Valitaan ympyr�n keh�lt� piste $C$ siten,
ett� $OC$ ja $AB$ ovat kohtisuorassa toisiaan vastaan. Olkoon $P$
mielivaltainen (lyhemm�n) kaaren $BC$ piste ja leikatkoot suorat $CP$ ja $AB$
pisteess� $Q$. Valitaan $R$  $AP$:lt� niin, ett�  $RQ$ ja $AB$ ovat
kohtisuorassa toisiaan vastaan. Osoita, ett� $|BQ|=|QR|$.}
   \qquad
   \vtop{\hsize=0.35\hsize
\beginpicture
\setcoordinatesystem units <1.5truecm,1.5truecm> point at 0 2.7
\setplotarea x from -1.2 to 1.2, y from -0.2 to 1.2
\circulararc 200 degrees from 0.985 -0.174 center at 0 0
\putrule from -1.2 0 to 2.3 0
\putrule from 2 0 to 2 1
\setlinear 
\plot  -1 0 2.3 1.1 /
\plot  2 0 0 1 /
%\put {$\bullet$} at 0.8 0.6
\put {$P$} [lb] <1truemm,2truemm> at 0.8 0.6
\put {$\bullet$} at 0 0
\put {$O$} [t] <0truemm,-2truemm> at 0 0
%\put {$\bullet$} at -1 0
\put {$A$} [tr] <-1truemm,-1truemm> at -1 0
%\put {$\bullet$} at 1 0
\put {$B$} [tl] <1truemm,-1truemm> at 1 0
%\put {$\bullet$} at 0 1
\put {$C$} [b] <0truemm,2truemm> at 0 1
%\put {$\bullet$} at 2 0
\put {$Q$} [tl] <1truemm,-1truemm> at 2 0
%\put {$\bullet$} at 2 1
\put {$R$} [bl] <1truemm,1truemm> at 2 1
\endpicture}
   \hfill}

\medskip
\hrjno=1
\harj Viestit koodataan k�ytt�en vain nollista ja
ykk�sist� koostuvia jonoja. Vain sellaisia jonoja, joissa esiintyy enint��n
kaksi per�kk�ist� ykk�st� tai nollaa saa k�ytt��.  (Esimerkiksi jono 011001
on sallittu, mutta 011101 ei ole.) M��rit� kaikkien tasan 12 merkist�
koostuvien jonojen lukum��r�.



\harj  Olkoon $n\ge 2$ ja olkoot $x_1$, $x_2$, \dots $x_n$
reaalilukuja, joille on voimassa $x_1+x_2+\dots + x_n\ge 0$ ja
$x_1^2+x_2^2+\dots +x_n^2=1$. Olkoon $M=\max\{x_1,\,x_2,\,\dots,\,x_n\}$.
Osoita, ett� $$M\ge {1\over \sqrt {n(n-1)}}.\eqno(1)$$
Selvit�, milloin (1):ss� vallitsee yht�suuruus.



\harj  Osoita, ett� on olemassa ��rett�m�n monta kesken��n
ep�yhtenev�� kolmiota $T$, joille p�tee 

\smallbreak

\item{(i)} Kolmion $T$ sivujen pituudet ovat per�kk�isi� kokonaislukuja.

\smallbreak

\item{(ii)} $T$:n pinta-ala on kokonaisluku.



\bye
