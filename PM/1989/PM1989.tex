\input makrot
\input def
\input laskut

\font\isosser=cmss12
\centerline{\isosser 3. pohjoismainen kilpailu ??.??.1989}\bigskip

\harj  M��rit� alinta mahdollista astetta oleva polynomi
$P$ jolla on seuraavat ominaisuudet:

\smallbreak

\item{(a)} $P$:n kertoimet ovat kokonaislukuja,
\item{(b)} $P$:n kaikki nollakohdat ovat kokonaislukuja,
\item{(c)} $P(0)=-1$,
\item{(d)} $P(3)=128$.

 

\harj  Tetraedrin kolmella sivutahkolla on kaikilla suora
kulma niiden yhteisess� k�rjess�. N�iden sivutahkojen alat ovat $A$, $B$ ja
$C$. Laske tetraedrin kokonaispinta-ala.

 

\harj  Olkoon $S$ kaikkien niiden suljetun v�lin $[-1,\,1]$
pisteiden $t$ joukko, joilla on se ominaisuus, ett� yht�l�ill� $x_0=t$,
$x_{n+1}=2x_n^2-1$ m��ritellylle lukujonolle $x_0$, $x_1$, $x_2$, \dots{}
l�ytyy positiivinen kokonaisluku $N$ siten, ett� $x_n=1$ kaikilla $n\ge N$.
Osoita, ett� joukossa $S$ on ��rett�m�n monta alkiota. 

 

\harj  Mille positiivisille kokonaisluvuille $n$ p�tee
seuraava v�ite: jos $a_1$, $a_2$, \dots , $a_n$ ovat positiivisia
kokonaislukuja, $a_k\le n$ kaikilla $k$ ja $\sum_{k=1}^na_k=2n$, niin on aina
mahdollista valita $a_{i_1}$, $a_{i_2}$, \dots, $a_{i_j}$ siten, ett�
indeksit $i_1$, $i_2$, \dots , $i_j$ ovat eri lukuja ja
$\sum_{k=1}^ja_{i_k}=n$?



\bye
