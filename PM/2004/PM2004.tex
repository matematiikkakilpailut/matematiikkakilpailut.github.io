\input makrot
\input laskut

\def\frac#1#2{{#1\over#2}}
\font\isosser=cmss12

\centerline{\isosser 18.~Pohjoismainen matematiikkakilpailu, 1.\:4.\:2004}

\vfill
\harj 
27 palloa, jotka on numeroitu 1:st� 27:��n, on sijoitettu punaiseen,
siniseen ja keltaiseen maljaan. Mitk� ovat punaisessa maljassa olevien
pallojen mahdolliset lukum��r�t, kun tiedet��n, ett� punaisessa,
sinisess� ja keltaisessa maljassa olevien pallojen numeroiden
keskiarvot ovat 15, 3 ja 18, t�ss� j�rjestyksess�?

\harj 
Olkoon $f_1=0$, $f_2=1$, ja $f_{n+2}=f_{n+1}+f_{n}$, kun $n=1$, 2,
$\dots$, Fibonaccin lukujono. Osoita, ett� on olemassa aidosti kasvava
p��ttym�t�n aritmeettinen kokonaislukujono, jonka yksik��n luku ei
kuulu Fibonaccin jonoon.

\smallskip\noindent
[Lukujono on {\it aritmeettinen\/}, jos sen per�kk�isten j�senten
erotus on vakio.]

\harj
Olkoon $x_{11}$, $x_{21}$, \dots, $x_{n1}$, $n> 2$,
kokonaislukujono. Oletetaan, ett� luvut $x_{i1}$ eiv�t kaikki ole
samoja. Jos luvut $x_{1k}$, $x_{2k}$, \dots, $x_{nk}$ on m��ritelty,
niin asetetaan
$$x_{i,k+1}=\frac12(x_{ik}+x_{i+1,k}), \; i=1, 2, \dots, n-1, \quad
x_{n,k+1}=\frac12(x_{nk}+x_{1k}).$$
Osoita, ett� jos  $n$ on pariton, niin jollakin $j$, $k$, $x_{jk}$ ei ole kokonaisluku. 
P�teek� t�m� my�s silloin, kun $n$ on parillinen? 

\harj
Olkoot $a$, $b$ ja $c$ kolmion sivujen pituudet ja olkoon $R$ kolmion
ymp�ri piirretyn ympyr�n s�de. Osoita, ett�
$$\frac {1}{ab}+\frac {1}{bc}+\frac {1}{ca}\ge \frac {1}{R^2}.$$

\vfill
\bye
