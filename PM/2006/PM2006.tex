\input makrot
\input laskut
\suomi
\overfullrule=0pt

\font\isosser=cmss12


\centerline{\isosser 20. Pohjoismainen matematiikkakilpailu, 30.\:3.\:2006}


\vfill
\harj  
Pisteet $B$ ja $C$ sijaitsevat kahdella pisteest�~$A$ l�htev�ll�
puolis�teell� niin, ett� $AB+AC$ on vakio. Osoita, ett� on olemassa
piste~$D\ne A$, niin ett� kolmion~$ABC$ ymp�ri piirretty ympyr� kulkee
$D$:n kautta kaikilla pisteiden~$B$ ja $C$ valinnoilla.

\harj  
Reaaliluvut $x$, $y$ ja $z$ eiv�t kaikki ole samoja ja ne toteuttavat
yht�l�t $$x+{1\over y}=y+{1\over z}=z+{1\over x}=k.$$ M��rit� kaikki
mahdolliset $k$:n arvot.


\harj 
Positiivisten kokonaislukujen jonon $\{ a_n\}$ m��rittelev�t ehdot
   $$
a_0=m\qquad {\rm ja} \qquad 
a_{n+1}=a_n^5 +487 \quad 
{\rm kaikilla}\quad n\geq 0.
   $$ 
M��rit� kaikki sellaiset $m$:n arvot, joilla jonoon kuuluu
mahdollisimman monta neli�lukua.


\harj 
$100\times 100$-{\v s}akkilaudan neli�t v�ritet��n 100:lla eri
v�rill�. Kuhunkin ruutuun k�ytet��n vain yht� v�ri� ja joka v�ri�
k�ytet��n tasan sataan ruutuun. Osoita, ett� laudalla on jokin vaaka-
tai pystyrivi, jonka ruutuihin on k�ytetty ainakin kymment� v�ri�.



\bye
