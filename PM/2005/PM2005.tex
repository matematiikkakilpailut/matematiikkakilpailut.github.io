\input makrot
\input laskut

\font\isosser=cmss12

\centerline{\isosser 19. Pohjoismainen matematiikkakilpailu, 5.\:4.\:2005} 

\vfill
\harj
M��rit� kaikki ne positiiviset kokonaisluvut $k$, joiden
kymmenj�rjestelm�esityksen numeroiden tulo on
   $$
{25\over 8} k - 211.
   $$

\harj
Olkoot $a$, $b$ ja $c$ positiivisia reaalilukuja. Todista, ett�
   $$
{2a^2 \over b + c}+{2b^2 \over c + a}+{2c^2 \over a + b}
\ge a + b + c.
   $$

\harj
2005~nuorta istuu suuren py�re�n p�yd�n ymparill�. Nuorista enint��n
668 on poikia. Sanomme, ett� tyt�n~$G$ asema on vahva, jos
tarkasteltaessa $G$:st� alkaen kuinka monen hyv�ns� vierekk�in istuvan
nuoren joukkoa kumpaan tahansa suuntaan, niin on n�iss� joukoissa on
aina aidosti enemm�n tytt�j� kuin poikia ($G$ on itse mukana laskussa).
Osoita, ett� olivat tyt�t ja pojat miss� j�rjestyksess� tahansa, joku
tytt� on aina vahvassa asemassa.

\harj
Ympyr� $\C_1$ on ympyr�n $\C_2$ sis�puolella, ja ympyr�t sivuavat toisiaan
pisteess�~$A$. $A$:n kautta kulkeva suora leikkaa $\C_1$:n my�s pisteess�~$B$
ja $\C_2$:n my�s pisteess�~$C$. Ympyr�n $\C_1$ pisteeseen~$B$ 
piirretty tangentti
leikkaa $\C_2$:n pisteiss�~$D$ ja $E$. Pisteen~$C$ 
kautta kulkevat ympyr�n~$\C_1$ tangentit sivuavat $\C_1$:t� 
pisteiss�~$F$ ja $G$. Osoita, ett� pisteet $D$, $E$,
$F$ ja $G$ ovat samalla ympyr�ll�.

\vfill
\bye
