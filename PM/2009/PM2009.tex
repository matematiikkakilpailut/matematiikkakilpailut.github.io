\input makrot
\input laskut

\font\isosser=cmss12

\centerline{\isosser 23.\ Pohjoismainen matematiikkakilpailu, 2.\:4.\:2009}

\vskip 3truecm
\harj
Kolmion sis�lt� valitaan piste $P$. $P$:n kautta piirret��n kolme
kolmion sivujen suuntaista suoraa. Ne jakavat kolmion kolmeksi
pienemm�ksi kolmioksi ja kolmeksi suunnikkaaksi. Olkoon $f$ kolmen
pienen kolmion yhteenlasketun alan ja koko kolmion alan suhde. Osoita,
ett� $f\ge \displaystyle{1\over 3}$, ja m��rit� ne pisteet $P$, joille
$f=\displaystyle{1\over 3}$.

\harj
Haalistuneelta paperinpalalta voidaan vaivoin lukea seuraavat merkinn�t:
   $$
(x^2 + x + a)(x^{15} - \dots) = x^{17} + x^{13} + x^5 - 90x^4 + x - 90. 
   $$
Jotkin osat ovat h�ipyneet n�kyvist�, erityisesti vasemman puolen
ensimm�isen tekij�n vakiotermi ja toisen tekij�n loppuosa. Olisi
mahdollista selvitt�� kokonaan toinen tekij�, mutta kysyt��n vain,
mik� on vakiotermi $a$. Oletetaan, ett� kaikki teht�v�ss� esiintyv�t
polynomit ovat kokonaislukukertoimisia.

\harj
Taululle on kirjoitettu kokonaisluvut 1, 2, 3, 4 ja 5. Lukuja voidaan
muuttaa niin, ett� pyyhit��n pois luvut $a$ ja $b$ ja kirjoitetaan
niiden sijaan luvut $a+b$ ja $ab$. Onko mahdollista toistamalla t�t�
operaatiota p��st� tilanteeseen, jossa kolme viidest� taululla
olevasta luvusta on 2009?

\harj
Turnaukseen osallistuu 32 kilpailijaa. Kaikki ovat pelikyvyilt��n
erilaisia ja kaksinkamppailussa parempi aina voittaa. Osoita, ett�
kulta-, hopea- ja pronssimitalien voittajat voidaan ratkaista 39
ottelun perusteella.

\bye 
