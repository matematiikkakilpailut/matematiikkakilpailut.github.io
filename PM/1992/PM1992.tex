\input makrot
\input def
\input laskut

\font\isosser=cmss12
\centerline{\isosser 6. pohjoismainen kilpailu ??.??.1992}\bigskip

\harj  M��rit� kaikki ne yht� suuremmat reaaliluvut $x$,
$y$ ja $z$, jotka toteuttavat yht�l�n $$x+y+z+{3\over x-1}+{3\over
y-1}+{3\over z-1}=2\(\sqrt{x+2}+\sqrt{y+2}+\sqrt{z+2}\).$$



\harj  Olkoon $n>1$ kokonaisluku ja olkoot $a_1$, $a_2$,
\dots, $a_n$ $n$ eri kokonaislukua. Todista, ett� polynomi $$f(x)=(x-a_1)(x-
a_2)\cdot\dots\cdot(x-a_n)-1$$ {\it ei ole jaollinen\/} mill��n
kokonaislukukertoimisella polynomilla, jonka aste on suurempi kuin nolla
mutta pienempi kuin $n$ ja jonka korkeimman $x$:n potenssin kerroin on 1. 



\harj  Todista, ett� kaikista kolmioista, joiden sis��n
piirretyn ympyr�n s�de on 1, pienin {\it piiri\/} on tasasivuisella
kolmiolla.



\harj  Peterill� on paljon samankokoisia neli�it�, joista
osa on mustia, osa valkeita. Peter haluaa koota neli�ist��n ison neli�n,
jonka sivun pituus on $n$ pikkuneli�n sivua, siten, isossa neli�ss� ei ole
{\it yht��n\/} sellaista pikkuneli�ist� muodostuvaa suorakaidetta, jonka
kaikki k�rkineli�t olisivat samanv�risi�. Kuinka suuren neli�n Peter pystyy
tekem��n?



\bye
