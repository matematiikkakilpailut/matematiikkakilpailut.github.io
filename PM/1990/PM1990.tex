\input makrot
\input def
\input laskut

\font\isosser=cmss12
\centerline{\isosser 4. pohjoismainen kilpailu ??.??.1990}\bigskip

\harj  Olkoot $m$, $n$ ja $p$  parittomia positiivisia
kokonaislukuja. Osoita, ett� luku  $$\sum_{k=1}^{(n-1)^p}k^m$$ on jaollinen
$n$:ll�. 
 
 
 
\harj  Olkoot $a_1$, $a_2$, \dots, $a_n$ reaalilukuja.
Osoita, ett� $$\root 3\of{a_1^3+a_2^3+\dots+a_n^3}\le \sqrt{a_1^2+a_2^2+
\dots+a_n^2}.\eqno{(1)}$$ Milloin (1):ss� vallitsee yht�suuruus? 
 

 
\harj   Olkoon $ABC$ kolmio ja $P$ piste $ABC$:n sis�ll�.
Oletetaan, ett� suora $l$, joka kulkee pisteen $P$ kautta, mutta ei
pisteen $A$ kautta, leikkaa $AB$:n ja $AC$:n  (tai niiden $B$:n ja $C$:n yli
ulottuvat jatkeet) pisteiss� $Q$ ja $R$. Etsi sellainen suora $l$, ett�
kolmion $AQR$ piiri on mahdollisimman pieni. 
 

 
\harj  Positiivisille kokonaisluvuille on sallittu kolme
operaatiota $f$, $g$ and $h$: $f(n)=10n$, $g(n)=10n+4$ and $h(2n)=n$, ts.\
luvun loppuun saa kirjoittaa nollan tai nelosen ja parillisen luvun saa
jakaa kahdella. Todista: jokaisen positiivisen kokonaisluvun voi konstruoida
aloittamalla luvusta 4 ja suorittamalla ��rellinen m��r� operaatioita $f$,
$g$ ja $h$ jossakin j�rjestyksess�. 



\bye
