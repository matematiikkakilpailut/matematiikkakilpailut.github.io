\input makrot
\input def
\input laskut

\font\isosser=cmss12
\centerline{\isosser 7. pohjoismainen kilpailu ??.??.1993}\bigskip

\harj  Olkoon $F$ kaikilla $x$, $0\le x\le 1$, m��ritelty
kasvava reaalilukuarvoinen funktio, joka toteuttaa ehdot

\smallbreak

\item{(i)} \qquad$\displaystyle F\({x\over 3}\)={F(x)\over 2}$

\smallbreak

\item{(ii)} \qquad$F(1-x)=1-F(x)$.

\smallbreak

\noindent{}M��rit� $\displaystyle F\({173\over 1993}\)$ ja $\displaystyle
F\({1\over 13}\)$.



\harj  $r$-s�teisen ympyr�n sis��n on piirretty
kuusikulmio. Kuusikulmion sivuista kaksi on pituudeltaan 1, kaksi
pituudeltaan 2 ja viimeiset kaksi pituudeltaan 3. Osoita, ett� $r$ toteuttaa
yht�l�n $$2r^3-7r-3=0.$$



\harj  Etsi kaikki yht�l�ryhm�n $$\left\{\eqalign{s(x)
+s(y)&=x\cr x+y+s(z)&=z\cr s(x)+s(y)+s(z)&=y-4\cr}\right.$$ ratkaisut, kun
$x$, $y$ ja $z$ ovat positiivisia kokonaislukuja ja $s(x)$, $s(y)$ ja $s(z)$
ovat $x$:n, $y$:n ja $z$:n kymmenj�rjestelm�esityksien {\it numeroiden
lukum��r�t\/}.



\harj  Merkit��n $T(n)$:ll� positiivisen kokonaisluvun $n$
kymmenj�rjestelm�esityksen {\it numeroiden summaa\/}. 

\smallbreak

\noindent{}a) Etsi positiiviluku $N$, jolle $T(k\cdot N)$ on parillinen
kaikilla $k$, $1\le k\le 1992$, mutta $T(1993\cdot N)$ on pariton.

\smallbreak

\noindent{}b) Osoita, ett� ei ole olemassa positiivista kokonaislukua $N$,
jolle $T(k\cdot N)$ olisi parillinen kaikilla positiivisilla
kokonaisluvuilla $k$.



\bye
