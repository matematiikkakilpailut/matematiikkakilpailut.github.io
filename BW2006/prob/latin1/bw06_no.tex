\documentclass{bw06}
%\usepackage[norwegian]{babel}
\version{Norwegian} \maintitle{Baltic Way 2006} \linetwo{Turku, 3.
november 2006}

\begin{document}
\maketitle

\begin{problems}

\item
  En f\o lge av reelle tall ~$a_1, a_2, a_3, \dots$ tilfredsstiller
    \[a_n=a_{n-1}+a_{n+2} \quad \text{for } n=2,3,4,\dots.\]
  Hva er st\o rste mulige antall positive p�f\o lgende elementer i f\o lgen?

\item
  Anta at de reelle tallene
$a_i\in[-2, 17]\; (i = 1, 2, \dots, 59)$ tilfredsstiller
$a_1+a_2+\dots+a_{59}=0$. Vis at
\[a_1^2 + a_2^2 + \dots + a_{59}^2 \le 2006.\]

\item
Vis at det for ethvert polynom $P(x)$ med reelle koeffisienter
finnes et positivt heltall~$m$ og polynomer $P_1(x)$, $P_2(x)$,
$\ldots$, $P_m(x)$ med reelle koeffisienter slik at
\[P(x)=(P_1(x))^3+(P_2(x))^3+\ldots+(P_m(x))^3.\]

\item
La $a$, $b$, $c$, $d$, $e$, $f$ v\ae re ikkenegative reelle tall som
tilfredsstiller $a+b+c+d+e+f=6$. Finn den st\o rste mulige verdien
til
\[abc+bcd+cde+def+efa+fab\] og bestem alle $6$-tupler
$(a,b,c,d,e,f)$ for hvilke denne maksimumsverdien n�s.

\item En til tider up�litelig professors siste bok omhandler en spesiell bin\ae r operasjon~$*$.
N�r denne operasjonen anvendes p� to heltall, blir resultatet igjen
et heltall.

\noindent Vi vet at operasjonen oppfyller f\o lgende to aksiomer:
  \begin{itemize}
    \item[a) ] $x*(x*y)=y$ for alle~$x,y \in \mathbb{Z}$;
    \item[b) ] $(x*y)*y=x$ for alle~$x,y \in \mathbb{Z}$.
  \end{itemize}

Professoren hevder i sin bok at:
  \begin{enumerate}
  \item Operasjonen~$*$ er kommutativ:
    $x*y=y*x$ for alle~$x,y \in \mathbb{Z}$.
  \item Operasjonen~$*$ er assosiativ:
    $(x*y)*z=x*(y*z)$ for alle~$x,y,z \in \mathbb{Z}$.
  \end{enumerate}
Hvilke av disse p�standene f\o lger fra de to aksiomene?

\item Bestem det st\o rste mulige antall elementer i et sett av positive heltall som oppfyller f\o lgende krav:
\begin{enumerate}
\item Heltallene best�r kun av siffre fra mengden $\{1,2,3,4,5,6\}$.
\item Intet siffer opptrer mer enn \'en gang i hvert tall.
\item Siffrene i hvert av tallene kommer i stigende rekkef\o lge.
\item Ethvert par av tall har minst ett siffer til felles (muligens i forskjellige posisjoner).
\item Intet siffer dukker opp i alle tallene.
\end{enumerate}

\item En fotograf tok bilder p� en fest med 10 deltagere. For hvert av de 45 mulige parene finnes det n\o yaktig ett bilde der begge er
med, og p� hvert av bildene er det enten to eller tre personer. Hva
er det minste mulige antallet bilder som ble tatt?

\item  Direkt\o ren har funnet ut at de ansatte har formet seks konspirasjoner p�
avdelingen hans, med n\o yaktig tre personer involvert i hver
konspirasjon. Vis at direkt\o ren kan splitte avdelingen i to
laboratorier slik at ingen av konspirasjonene har alle sine
involverte i samme laboratorium.

\item  %PU4
Til ethvert hj\o rne i en regul\ae r femkant tilegnes et reelt tall.
Det er tillatt � utf\o re  f\o lgende operasjon gjentatte ganger: Vi
velger to nabohj\o rner, og erstatter begge de tilegnede tallene med
deres gjennomsnitt. Er det alltid mulig � oppn� at alle tallene blir
lik null, gitt at summen til de fem tilegnede tallene opprinnelig
var lik null?

\item 162 plusstegn og 144 minustegn er plassert p� et $30\times 30$
brett slik at hver rad og hver kolonne inneholder h\o yst 17 tegn
(intet felt inneholder mer enn ett tegn). For hvert plusstegn teller
vi antall minustegn i samme rad, og for hvert minustegn teller vi
antall plusstegn i samme kolonne. Finn den maksimale verdien til
summen av disse tallene.
%suggested by   Konstantin Kokhas

\item H\o ydene i en trekant er 12, 15, og 20. Hva er trekantens
areal?

\item  La $ABC$ v\ae re en trekant, $B_1$ v\ae re midtpunktet til siden $AB$
og $C_1$ midpunktet til siden $AC$. La videre $P$ v\ae re det andre
skj\ae ringspunktet mellom $ABC_1$'s og $AB_1C$'s omskrevne sirkler,
samt $P_1$ v\ae re det andre skj\ae ringspunktet mellom linjen $AP$
og $AB_1C_1$'s omskrevne sirkel. Vis at $2AP = 3AP_1$.

\item  %PU5
I trekanten $ABC$ ligger punktene $D$ og $E$ henholdsvis p� sidene
$AB$ og $AC$. Linjene $BE$ og $CD$ skj\ae rer hverandre i $F$. Vis
at dersom
\[BC^2=BD\cdot BA+CE\cdot CA,\]
m� punktene $A$, $D$, $F$ og $E$ ligge p� en sirkel.

\item  2006 punkter er markert p� overflaten til en sf\ae re. Vis
at overflaten kan kuttes i 2006 kongruente deler slik at hver del
inneholder n\o yaktig ett av punktene.

\item  Medianene til trekanten $ABC$ m\o tes i punktet $M$. En
linje $t$ gjennom $M$ skj\ae rer $ABC$'s omsirkel i $X$ og $Y$ slik
at $A$ og $C$ ligger p� samme side av $t$. Vis at $BX\cdot
BY=AX\cdot AY + CX\cdot CY$.

\item Finnes det 4 forskjellige positive heltall slik at ethvert
produkt av to av dem \o ket med 2006 gir et kvadrattall?

\item  %PU7
Bestem alle positive heltall $n$ slik at $3^n+1$ blir delelig med
$n^2$.

\item For ethvert positivt heltall $n$ la $a_n$ betegne siste siffer i
$n^{(n^n)}$. Vis at f\o lgen $(a_n)$ er periodisk, og bestem lengden
til den minimale perioden.

\item Finnes det en f\o lge $a_1,a_2,a_3,\ldots$ av positive heltall slik at det for ethvert positivt heltall $n$
holder at alle summer av  $n$ p�f\o lgende elementer er delelige med
$n^2$?

\item  Et 12-siffret positivt heltall best�ende utelukkende av
siffrene 1, 5 og 9 er delelig med 37. Vis at summen av dens siffre
ikke kan v\ae re 76.


\end{problems}

Tid til disposisjon: $4$ og en halv time. 5 poeng per oppgave.

\end{document}
