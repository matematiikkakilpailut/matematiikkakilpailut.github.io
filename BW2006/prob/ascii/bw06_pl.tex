\documentclass{bw06}

\usepackage{polski}
\prefixing

\version{Polish}
\maintitle{Baltic Way 2006}
\linetwo{Turku, 3 listopada 2006}

\begin{document}
\maketitle

\begin{problems}

\item
  Ci/ag liczb rzeczywistych $a_1, a_2, a_3, \dots$ spe/lnia warunek
  \[a_n=a_{n-1}+a_{n+2} \quad \text{dla } n=2,3,4,\dots.\]
Ile najwi/ecej kolejnych wyraz/ow tego ci/agu mo/ze by/c dodatnich?
 
\item
  Liczby rzeczywiste
$a_i\in[-2, 17];\; i = 1, 2, \dots, 59$
spe/lniaj/a $a_1+a_2+\dots+a_{59}=0$.
Udowodni/c, /ze
\[a_1^2 + a_2^2 + \dots + a_{59}^2 \le 2006.\]

\item
Wykaza/c, /ze dla ka/zdego wielomianu 
 $P(x)$ o wsp/o/lczynnikach rzeczywistych
istniej/a liczba ca/lkowita dodatnia~$m$ oraz takie wielomiany 
$P_1(x)$, $P_2(x)$, $\ldots$, $P_m(x)$ o wsp/o/lczynnikach rzeczywistych,
/ze
\[P(x)=(P_1(x))^3+(P_2(x))^3+\ldots+(P_m(x))^3.\]

\item
Niech  $a$, $b$, $c$, $d$, $e$, $f$ b/ed/a nieujemnymi liczbami rzeczywistymi 
spe/lniaj/acymi warunek
 $a+b+c+d+e+f=6$. 
Znale/x/c najwi/eksz/a mo/zliw/a warto/s/c wyra/zenia
\[abc+bcd+cde+def+efa+fab\] 
i wyznaczy/c wszystkie 6-tki
$(a,b,c,d,e,f)$, dla kt/orych ta najwi/eksza warto/s/c jest osi/agana.

\item Nie zawsze wiarygodny profesor 
po/swi/eci/l swoj/a ostatni/a ksi/a/zk/e pewnemu 2-argumentowemu dzia/laniu~$*$. 
Dzia/lanie to zastosowane do dw/och liczb ca/lkowitych daje w wyniku liczb/e ca/lkowit/a.
Ma ono ponadto nast/epuj/ace w/lasno/sci:

  \begin{itemize}
    \item[a) ] $x*(x*y)=y$ dla dowolnych~$x,y \in \mathbb{Z}$;
    \item[b) ] $(x*y)*y=x$ dla dowolnych~$x,y \in \mathbb{Z}$.
  \end{itemize}
Profesor twierdzi w swojej ksi/a/zce, /ze 
  \begin{enumerate}
  \item dzia/lanie~$*$ jest przemienne:
    $x*y=y*x$ dla dowolnych~$x,y \in \mathbb{Z}$.
  \item dzia/lanie~$*$ jest /l/aczne:
    $(x*y)*z=x*(y*z)$ dla dowolnych~$x,y,z \in \mathbb{Z}$.
  \end{enumerate}
Kt/ore z tych twierdze/n wynikaj/a z w/lasno/sci a) i b) ?

\item 
Wyznaczy/c najwi/eksz/a moc zbioru liczb ca/lkowitych dodatnich o nast/epuj/acych w/lasno/sciach:
\begin{enumerate}
\item Cyfry ka/zdej z liczb nale/z/a do zbioru $\{1,2,3,4,5,6\}$.
\item /Zadna cyfra nie wyst/epuje w jednej liczbie wi/ecej ni/z raz.
\item W ka/zdej liczbie cyfry wyst/epuj/a w porz/adku rosn/acym.
\item Ka/zde dwie liczby maj/a co najmniej jedn/a wsp/oln/a cyfr/e (by/c mo/ze na r/o/znych pozycjach).
\item /Zadna cyfra nie wyst/epuje we wszystkich liczbach.
\end{enumerate}

\item 
Na 10-osobowym przyj/eciu fotograf robi/l zdj/ecia. 
Ka/zda spo/sr/od 45 mo/zliwych par uczestnik/ow wyst/epuje razem na dok/ladnie jednym zdj/eciu, a ka/zde zdj/ecie
przedstawia dwie lub trzy osoby. Jaka jest najmniejsza mo/zliwa liczba zrobionych zdj/e/c?

\item
Dyrektor wykry/l w swoim wydziale 6 spisk/ow. Ka/zdy z nich zosta/l zawi/azany przez dok/ladnie 3 osoby. 
Dowie/s/c, /ze dyrektor mo/ze podzieli/c wydzia/l na dwa laboratoria w taki spos/ob, 
aby /zadna spiskuj/aca grupa nie znalaz/la si/e ca/la w jednym laboratorium.

\item
Ka/zdemu wierzcho/lkowi pi/eciok/ata foremnego przypisano liczb/e rzeczywist/a. Wykonujemy wielokrotnie
nast/epuj/ac/a operacj/e: Wybieramy dwa s/asiednie wierzcho/lki pi/eciok/ata i zast/epujemy ka/zd/a
z przypisanych im liczb przez /sredni/a arytmetyczn/a tych liczb. Rozstrzygn/a/c, czy zawsze mo/zna
uzyska/c zera we wszystkich wierzcho/lkach, je/zeli w pocz/atkowej pozycji suma wszystkich pi/eciu liczb
jest r/owna zeru?

\item W tabeli $30\times 30$ rozmieszczono 162 plusy i 144 minusy, przy czym ka/zdy wiersz i ka/zda kolumna
zawiera co najwy/zej 17 znak/ow. (Ka/zde pole tabeli zawiera co najwy/zej jeden znak.)
Dla ka/zdego plusa obliczamy liczb/e minus/ow stoj/acych w tym samym wierszu,
a dla ka/zdego minusa obliczamy liczb/e plus/ow stoj/acych w tej samej kolumnie.
Znale/x/c najwi/eksz/a mo/zliw/a sum/e wszystkich obliczonych liczb.

\item Wysoko/sci tr/ojk/ata maj/a d/lugo/sci 12, 15 i 20. Ile wynosi pole tego tr/ojk/ata?

\item 
W tr/ojk/acie $ABC$ punkt $B_1$ jest /srodkiem boku $AB$, a punkt
$C_1$ jest /srodkiem boku $AC$. Niech $P$ b/edzie r/o/znym od $A$ punktem przeci/ecia
okr/eg/ow opisanych na tr/ojk/atach $ABC_1$ i $AB_1C$.
Niech $P_1$ b/edzie r/o/znym od $A$ punktem przeci/ecia prostej $AP$
z okr/egiem opisanym na tr/ojk/acie $AB_1C_1$. Dowie/s/c, /ze $2AP = 3AP_1$.

\item
W tr/ojk/acie $ABC$ punkty $D$, $E$ le/z/a odpowiednio na bokach $AB$, $AC$.
Proste $BE$ i $CD$ przecinaj/a si/e w punkcie $F$. Wykaza/c, /ze je/zeli
\[BC^2=BD\cdot BA+CE\cdot CA,\]
to punkty $A$, $D$, $F$, $E$ le/z/a na jednym okr/egu.

\item
Na powierzchni sfery zaznaczono 2006 punkt/ow.
Udowodni/c, /ze powierzchni/e mo/zna rozci/a/c na 2006 przystaj/acych cz/e/sci tak,
aby ka/zda cz/e/s/c zawiera/la w swoim wn/etrzu dok/ladnie jeden spo/sr/od zaznaczonych punkt/ow.

\item
/Srodkowe tr/ojk/ata $ABC$ przecinaj/a si/e w punkcie $M$.
Prosta $t$ przechodz/aca przez punkt $M$ przecina okr/ag opisany
na tr/ojk/acie $ABC$ w punktach $X$ i $Y$, przy czym
punkty $A$ i $C$ le/z/a po tej samej stronie prostej $t$.
Dowie/s/c, /ze $BX\cdot BY=AX\cdot AY + CX\cdot CY$.

\item
Czy istniej/a takie 4 r/o/zne liczby ca/lkowite dodatnie, /ze
iloczyn dowolnych dw/och z nich, powi/ekszony o 2006, jest
kwadratem liczby ca/lkowitej?

\item
Znale/x/c wszystkie takie liczby ca/lkowite dodatnie $n$,
/ze liczba $3^n+1$ dzieli si/e przez $n^2$.

\item
Dla ka/zdej liczby ca/lkowitej dodatniej $n$ niech $a_n$
oznacza ostatni/a cyfr/e liczby $n^{(n^n)}$. Udowodni/c, /ze ci/ag
$(a_n)$ jest okresowy i wyznaczy/c d/lugo/s/c najkr/otszego okresu.

\item
Czy istnieje taki ci/ag liczb ca/lkowitych dodatnich $a_1,a_2,a_3,\ldots$,
/ze dla dowolnej liczby ca/lkowitej dodatniej $n$ suma ka/zdych kolejnych $n$ wyraz/ow tego ci/agu
dzieli si/e przez $n^2$ ?

\item
Cyframi pewnej 12-cyfrowej liczby ca/lkowitej dodatniej, podzielnej przez 37,
mog/a by/c tylko 1, 5 i 9. Dowie/s/c, /ze suma cyfr tej liczby jest r/o/zna od 76.

\end{problems}

Czas pracy: $4\frac{1}{2}$ godz. Za ka/zde zadanie mo/zna otrzyma/c 5 punkt/ow.

\end{document}
