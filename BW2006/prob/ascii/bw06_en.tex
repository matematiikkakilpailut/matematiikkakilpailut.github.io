\documentclass{bw06}

\version{English}
\maintitle{Baltic Way 2006}
\linetwo{Turku, November 3, 2006}

\begin{document}
\maketitle

\begin{problems}

\item
  For a sequence~$a_1, a_2, a_3, \dots$ of real numbers it is known
  that
  \[a_n=a_{n-1}+a_{n+2} \quad \text{for } n=2,3,4,\dots.\]
  What is the largest
  number of its consecutive elements that can all be positive?

\item
  Suppose that the real numbers
$a_i\in[-2, 17];\; i = 1, 2, \dots, 59$
satisfy $a_1+a_2+\dots+a_{59}=0$.
Prove that
\[a_1^2 + a_2^2 + \dots + a_{59}^2 \le 2006.\]

\item
Prove that for every polynomial $P(x)$ with real coefficients
there exist a positive integer~$m$ and polynomials
$P_1(x)$, $P_2(x)$, $\ldots$, $P_m(x)$ with real coefficients
such that
\[P(x)=(P_1(x))^3+(P_2(x))^3+\ldots+(P_m(x))^3.\]

\item
Let $a$, $b$, $c$, $d$, $e$, $f$ be non-negative real
numbers satisfying $a+b+c+d+e+f=6$. Find the maximal possible value of
\[abc+bcd+cde+def+efa+fab\] and determine all $6$-tuples
$(a,b,c,d,e,f)$ for which this maximal value is achieved.

\item An occasionally unreliable professor has devoted his last book to 
a certain binary operation~$*$. When this operation is applied to 
any two integers, the result is again an integer.
The operation is known to satisfy the following axioms:
  \begin{itemize}
    \item[a) ] $x*(x*y)=y$ for all~$x,y \in \mathbb{Z}$;
    \item[b) ] $(x*y)*y=x$ for all~$x,y \in \mathbb{Z}$.
  \end{itemize}
The professor claims in his book that
  \begin{enumerate}
  \item the operation~$*$ is commutative:
    $x*y=y*x$ for all~$x,y \in \mathbb{Z}$.
  \item the operation~$*$ is associative:
    $(x*y)*z=x*(y*z)$ for all~$x,y,z \in \mathbb{Z}$.
  \end{enumerate}
Which of these claims follow from the stated axioms?

\item Determine the maximal size of a set of positive integers with the following
properties:
\begin{enumerate}
\item The integers consist of digits from the set $\{1,2,3,4,5,6\}$.
\item No digit occurs more than once in the same integer.
\item The digits in each integer are  in increasing order.
\item Any two integers have at least one digit in common (possibly at different
  positions).
\item There is no digit which appears in all the integers.
\end{enumerate}

\item A photographer took some pictures at a party with 10 people. Each of
the 45 possible pairs of people appears together on exactly one
photo, and each photo depicts two or three people. What is the
smallest possible number of photos taken?

\item  The director has found out that six conspiracies have been set  
up in his department, each of them involving exactly 3 persons. Prove  
that the director can split the department in two laboratories so that  
none of the conspirative groups is entirely in the same laboratory.

\item  %PU4
To  every vertex of a regular pentagon
a real number is assigned. We may
perform the following operation repeatedly: 
we choose two adjacent vertices of
the pentagon and replace each of the two numbers assigned to these
vertices by their arithmetic mean.  
Is it always possible to obtain the
position in which all five numbers are zeroes,
given that in the initial position the sum
of all five numbers is equal to zero?

\item 162 pluses and 144 minuses are placed in a $30\times 30$ 
table in such
a way that each row and each column contains at most 17 signs.
(No cell contains more than one sign.)
For every plus we count the number of minuses in its row and for every minus
we count the number of pluses in its column. Find the maximum of the sum
of these numbers.
%suggested by   Konstantin Kokhas

\item The altitudes of a triangle are 12, 15, and 20. What is the area
of the triangle?

\item  Let $ABC$ be a triangle, let $B_1$ be the midpoint of the side $AB$ and 
$C_1$ the midpoint of the side $AC$. Let $P$ be the point of intersection, other than
$A$, of the circumscribed circles around the triangles $ABC_1$ and $AB_1C$. Let 
$P_1$ be the point of intersection, other than $A$, of the line $AP$ with the 
circumscribed circle around the triangle $AB_1C_1$. Prove that $2AP = 3AP_1$.

\item  %PU5
In a triangle $ABC$, points $D$, $E$ lie on sides $AB$, $AC$ respectively.
The lines $BE$ and $CD$ intersect at $F$. Prove that if
\[BC^2=BD\cdot BA+CE\cdot CA,\]
then the points $A$, $D$, $F$, $E$ lie on a circle.

\item  There are 2006 points marked on the surface of a sphere. Prove
  that the surface can be cut into 2006 congruent  pieces so that
  each piece contains exactly one of these points inside it.

\item  Let the medians of the triangle $ABC$ intersect at point $M$. A  
line $t$ through $M$ intersects the circumcircle of $ABC$ at $X$ and $Y$ so  
that $A$ and $C$ lie on the same side of $t$. Prove that $BX\cdot  
BY=AX\cdot AY + CX\cdot CY$.

\item Are there 4 distinct positive integers such that
adding the product of
any two of them to 2006 yields a perfect square?

\item  %PU7
Determine all positive integers $n$ such that $3^n+1$ is divisible by $n^2$.

\item For a positive integer $n$ let $a_n$ denote the last digit of
$n^{(n^n)}$. Prove that the sequence $(a_n)$ is periodic and
determine the length of the minimal period.

\item Does there exist a sequence $a_1,a_2,a_3,\ldots$ of positive integers such that
the sum of every $n$ consecutive elements is divisible by $n^2$ for every
positive integer $n$?

\item  A 12-digit positive integer consisting only of digits
  1, 5 and 9 is divisible by 37.
  Prove that the sum of its digits is not equal to 76.


\end{problems}

Working time $4\frac{1}{2}$ hours. 5 points per problem.

\end{document}
