\documentclass{bw06}

\version{Dansk}
\maintitle{Baltic Way 2006}
\linetwo{Turku, 3. november 2006}

\begin{document}
\maketitle

\begin{problems}

\item
  For en f\o lge~$a_1, a_2, a_3, \dots$ af reelle tal vides det at
  \[a_n=a_{n-1}+a_{n+2} \quad \text{for } n=2,3,4,\dots.\]
  Hvad er det st\o rste antal p\aa{} hinanden f\o lgende elementer 
som kan v\ae re positive?

\item
  Antag at de reelle tal
$a_i\in[-2, 17];\; i = 1, 2, \dots, 59$
opfylder $a_1+a_2+\dots+a_{59}=0$.
Bevis at
\[a_1^2 + a_2^2 + \dots + a_{59}^2 \le 2006.\]

\item
Bevis at for ethvert polynomium $P(x)$ med reelle koefficienter findes 
der et positivt heltal $m$ og polynomier
$P_1(x), P_2(x), \ldots, P_m(x)$ med reelle koefficienter
s\aa ledes at
\[P(x)=(P_1(x))^3+(P_2(x))^3+\dots+(P_m(x))^3.\]

\item
Lad $a$, $b$, $c$, $d$, $e$, $f$ v\ae re ikke-negative reelle
tal som opfylder $a+b+c+d+e+f=6$. Find den st\o rst mulige v\ae rdi af
\[abc+bcd+cde+def+efa+fab,\] og bestem alle $6$-tupler
$(a,b,c,d,e,f)$ for hvilke denne maksimale v\ae rdi opn\aa s.

\item En til tider up\aa lidelig professor har viet sin sidste bog
til en bestemt bin\ae r operation $*$. N\aa r denne operation anvendes
 p\aa{} to vilk\aa rlige heltal, er resultatet igen et heltal. Operationen
 vides at opfylde disse aksiomer:
  \begin{itemize}
    \item[a) ] $x*(x*y)=y$ for alle~$x,y \in \mathbb{Z}$;
    \item[b) ] $(x*y)*y=x$ for alle~$x,y \in \mathbb{Z}$.
  \end{itemize}
Professoren p\aa st\aa r i sin bog at
  \begin{enumerate}
  \item operationen~$*$ er kommutativ:
    $x*y=y*x$ for alle~$x,y \in \mathbb{Z}$.
  \item operationen~$*$ er associativ:
    $(x*y)*z=x*(y*z)$ for alle~$x,y,z \in \mathbb{Z}$.
  \end{enumerate}
Hvilke af disse p\aa stande f\o lger fra de givne aksiomer?

\item Bestem den maksimale st\o rrelse af en m\ae ngde af positive
 heltal med f\o lgende egenskaber:
\begin{enumerate}
\item Heltallene best\aa r af cifre fra m\ae ngden $\{1,2,3,4,5,6\}$.
\item Intet ciffer optr\ae der mere end \'en gang i samme heltal.
\item Cifrene i hvert heltal er i voksende orden.
\item Hvert par af heltal har mindst et ciffer til f\ae lles (eventuelt 
i forskellige positioner).
\item Der er ikke noget ciffer som forekommer i samtlige heltal.
\end{enumerate}

\item En fotograf tog nogle billeder til en fest med 10 personer. Hvert
 af de 45 mulige par af personer optr\ae der sammen p\aa{} pr\ae cis et
 billede, og p\aa{} hvert billede er der to
 eller tre personer. Hvad er det mindst mulige antal billeder fotografen 
kan have taget?

\item Direkt\o ren har fundet ud af at der er opst\aa et seks
 sammensv\ae rgelser i hans afdeling. Hver af sammensv\ae rgelserne
 involverer pr\ae cis tre personer. Bevis at direkt\o ren
 kan dele afdelingen i to underafdelinger, s\aa ledes at ingen af de
 konspirerende grupper er i samme underafdeling.

\item Til hvert hj\o rne i en regul\ae r femkant knyttes et reelt tal. 
Vi m\aa{} udf\o re f\o lgende gentagne gange: V\ae lg to nabohj\o rner i
 femkanten, og erstat hvert af de to tal knyttet til 
disse hj\o rner med deres aritmetiske gennemsnit. Er det altid muligt at 
opn\aa{} stillingen hvor alle fem tal er nul, 
givet at summen af tallene i udgangsstillingen er nul?

\item $162$ plusser og 144 minusser er placeret i en $30\times 30$-tabel 
p\aa{} en s\aa dan m\aa de at hver r\ae kke og hver s\o jle indeholder h\o jst 17 fortegn.
(Ingen celle indeholder mere end et fortegn.)
For hvert plus t\ae ller vi antallet af minusser i dets r\ae kke, og for
 hvert minus t\ae ller vi antallet af plusser i dets s\o jle. Find den maksimale sum af disse tal.
%suggested by   Konstantin Kokhas

\item H\o jderne i en trekant er~12, 15 og 20. Hvad er arealet af trekanten?

\item  Lad $ABC$ v\ae re en trekant, lad $B_1$ v\ae re midtpunktet af 
siden $AB$ og 
$C_1$ midtpunktet af siden $AC$. Lad $P$ v\ae re sk\ae ringspunktet, 
forskelligt fra $A$, mellem de omskrevne cirkler for trekanterne $ABC_1$ og $AB_1C$. Lad $P_1$ v\ae re sk\ae ringspunktet, 
forskelligt fra $A$, mellem linjen $AP$ og den omskrevne cirkel for trekant $AB_1C_1$. Bevis at $2\lvert AP\rvert =
 3\lvert AP_1\rvert$.

\item  %PU5
I en trekant $ABC$ ligger punkterne $D$ og $E$ p\aa{} henholdsvis siden $AB$ og siden $AC$. Linjerne $BE$ og $CD$ sk\ae rer i $F$. Bevis at hvis
\[\lvert BC\rvert^2=\lvert BD\rvert\cdot \lvert BA\rvert+\lvert CE\rvert\cdot \lvert CA\rvert,\]
s\aa{} ligger punkterne~$A$, $D$, $F$, $E$ p\aa{} en cirkel.

\item  Der er markeret 2006 punkter p\aa{} en sf\ae re. Bevis at sf\ae ren kan inddeles i 2006 indbyrdes kongruente stykker, s\aa ledes at hvert stykke indeholder pr\ae cis et af disse punkter i sit indre.

\item  Lad medianerne i trekant $ABC$ sk\ae re i punktet $M$. En linje $t$ gennem $M$ sk\ae rer den omskrevne cirkel for trekant $ABC$ i $X$ og $Y$ p\aa{} en s\aa dan m\aa de at $A$ og $C$ ligger p\aa{} samme side af $t$. Bevis at $\lvert BX\rvert\cdot\lvert BY\rvert=\lvert AX\rvert\cdot \lvert AY\rvert + \lvert CX\rvert \cdot \lvert CY\rvert$.

\item Findes der fire forskellige positive heltal med den egenskab at produktet af vilk\aa rlige to lagt til 2006, giver et kvadrattal?

\item  %PU7
Bestem alle positive heltal $n$, s\aa ledes at $3^n+1$ er delelig med $n^2$.

\item For et positivt heltal $n$ betegner $a_n$ det sidste ciffer i $n^{(n^n)}$. Bevis at f\o lgen $(a_n)$ er periodisk, og bestem l\ae ngden af perioden.

\item Findes der en f\o lge $a_1,a_2,a_3,\ldots$ af positive heltal, s\aa ledes at summen af vilk\aa rlige $n$ p\aa{} hinanden f\o lgende elementer er delelig med $n^2$ for ethvert positivt heltal $n$.

\item Et 12-cifret positivt heltal best\aa r kun af cifrene 1, 5 og 9 og er deleligt med 37. Bevis at summen af dets cifre ikke er lig 76.

\end{problems}

Varighed: $4\frac{1}{2}$ time. 5 point pr. opgave.

\end{document}

