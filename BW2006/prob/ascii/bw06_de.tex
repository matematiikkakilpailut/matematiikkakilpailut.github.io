\documentclass{bw06}

\version{German} \maintitle{Baltic Way 2006} \linetwo{Turku, 3.
November 2006}

\begin{document}
\maketitle

\begin{problems}

\item
  Von einer Folge~$a_1, a_2, a_3, \dots$ reeller Zahlen ist folgendes bekannt:
  \[a_n=a_{n-1}+a_{n+2} \quad \text{f{\"u}r } n=2,3,4,\dots.\]
  Welches ist die gr{\"o}{\ss}te Zahl aufeinander folgender Elemente, die alle positiv sind?

\item
  Die reellen Zahlen
$a_i\in[-2, 17],\; i = 1, 2, \dots, 59$, erf{\"u}llen
$a_1+a_2+\dots+a_{59}=0$. \\ Man beweise:
\[a_1^2 + a_2^2 + \dots + a_{59}^2 \le 2006\]

\item Man beweise: F{\"u}r jedes Polynom $P(x)$ mit reellen
Koeffizienten existieren eine positive ganze Zahl~$m$ und Polynome
$P_1(x)$, $P_2(x)$, $\ldots$, $P_m(x)$ mit reellen Koeffizienten
mit
\[P(x)=(P_1(x))^3+(P_2(x))^3+\ldots+(P_m(x))^3.\]

\item Es seien $a$, $b$, $c$, $d$, $e$, $f$ nicht-negative reelle
Zahlen mit $a+b+c+d+e+f=6$. Man bestimme den maximalen Wert von
\[abc+bcd+cde+def+efa+fab\] und ermittle alle $6$-Tupel
$(a,b,c,d,e,f)$, f{\"u}r die dieser maximale Wert angenommen wird.

\item Ein manchmal zerstreuter Professor hat sein letztes Buch
einer gewissen bin{\"a}ren Operation~$*$\,\, gewidmet. Angewandt auf
zwei beliebige ganze Zahlen ergibt diese Operation wieder eine
ganze Zahl. Diese Operation gen{\"u}gt den folgenden Axiomen:
  \begin{itemize}
    \item[a) ] $x*(x*y)=y$ f{\"u}r alle~$x,y \in \mathbb{Z}$;
    \item[b) ] $(x*y)*y=x$ f{\"u}r alle~$x,y \in \mathbb{Z}$.
  \end{itemize}
Der Professor behauptet in seinem Buch:
  \begin{enumerate}
  \item Die Operation~$*$ ist kommutativ, d.h.
    $x*y=y*x$ f{\"u}r alle~$x,y \in \mathbb{Z}$.
  \item Die Operation~$*$ ist assoziativ, d.h.
    $(x*y)*z=x*(y*z)$ f{\"u}r alle~$x,y,z \in \mathbb{Z}$.
  \end{enumerate}
Welche dieser Behauptungen folgt aus den gegebenen Axiomen?

\item Man bestimme die gr{\"o}{\ss}tm{\"o}gliche Anzahl von Elementen einer
Menge von positiven ganzen Zahlen, die folgende Eigenschaften
haben:
\begin{enumerate}
\item Die Zahlen bestehen aus Ziffern aus der Menge
$\{1,2,3,4,5,6\}$. \item In jeder Zahl kommt keine Ziffer mehr als
einmal vor. \item In jeder Zahl sind die Ziffern aufsteigend
angeordnet. \item Je zwei Zahlen haben mindestens eine Ziffer
gemeinsam (eventuell an verschiedenen Stellen). \item Es gibt
keine Ziffer, die in allen Zahlen vorkommt.
\end{enumerate}

\item Auf einer Party mit 10 G{\"a}sten macht ein Photograph Bilder.
Jedes der 45 m{\"o}glichen Paare von G{\"a}sten kommt genau auf einem Bild
vor, und jedes Photo zeigt zwei oder drei G{\"a}ste. Man bestimme die
kleinstm{\"o}gliche Anzahl von Photos.

\item Der Dekan hat herausgefunden, dass in seiner Fakult{\"a}t sechs
Verschw{\"o}rungen laufen. Jede Verschw{\"o}rung wird von genau 3 Personen
getragen. Man beweise, dass der Dekan seine Fakult{\"a}t so in zwei
Institute zerteilen kann, dass keine der Verschw{\"o}rungsgruppen zur
G{\"a}nze in einem Institut sitzt.

\item Jeder Ecke eines regelm{\"a}{\ss}igen F{\"u}nfecks wird eine reelle Zahl
zugeordnet. Die Summe dieser f{\"u}nf Zahlen ist Null. Man kann die
folgende Operation anwenden: An zwei benachbarten Ecken werden
die Zahlen beide durch ihr arithmetisches Mittel ersetzt. \\
Gelingt es immer, durch wiederholte Anwendung dieser Operation
alle f{\"u}nf Zahlen auf Null zu bringen?

\item In einem $30\times 30$-Feld werden 162 Plus-Zeichen und 144
Minus-Zeichen so verteilt, dass in jeder Zeile und jeder Spalte
h{\"o}chstens 17 Symbole stehen. (Keine Zelle enth{\"a}lt mehr als ein
Zeichen.) F{\"u}r jedes Plus-Zeichen z{\"a}hlen wir die Anzahl der
Minus-Zeichen in seiner Zeile, und f{\"u}r jedes Minus-Zeichen z{\"a}hlen
wir die Anzahl der Plus-Zeichen in seiner Spalte. Bestimme das
Maximum der Summe dieser Anzahlen.

\item Die H{\"o}hen eines Dreiecks haben die L{\"a}ngen 12, 15 und 20. Wie
gro{\ss} ist der Fl{\"a}cheninhalt dieses Dreiecks?

\item  Es seien $ABC$ ein Dreieck, $B_1$ der Mittelpunkt der Seite
$\overline{AB}$ und $C_1$ der Mittelpunkt der Seite
$\overline{AC}$. Sei weiterhin $P$ der von $A$ verschiedene
Schnittpunkt der Umkreise der Dreiecke $ABC_1$ und $AB_1C$.
Schlie{\ss}lich sei $P_1$ der von $A$ verschiedene Schnittpunkt der
Geraden $AP$ mit dem Umkreis des Dreiecks $AB_1C_1$. Man beweise:
$ 2|\overline{AP}| = 3|\overline{AP_1}|$

\item In einem Dreieck $ABC$ liegen die Punkte $D$ und $E$ auf den
Seiten $\overline{AB}$ bzw. $\overline{AC}$. Die Geraden $BE$ und
$CD$ schneiden sich in $F$. Man beweise: \\Wenn $
|\overline{BC}|^2=|\overline{BD}|\cdot
|\overline{BA}|+|\overline{CE}|\cdot |\overline{CA}|,$ dann liegen
$A$, $D$, $F$ und $E$ auf einem Kreis.

\item Auf der Oberfl{\"a}che einer Kugel werden 2006 Punkte markiert.
Man beweise: Die Oberfl{\"a}che kann so in 2006 kongruente Teilfl{\"a}chen
zerlegt werden, dass innerhalb jeder Fl{\"a}che genau einer dieser
Punkte liegt.

\item  Die Seitenhalbierenden des Dreiecks $ABC$ schneiden sich im
Punkt $M$. Eine Gerade $t$ durch $M$ schneidet den Umkreis von
$ABC$ in $X$ und $Y$; dabei liegen $A$ und $C$ auf derselben Seite
von $t$. Man zeige: $|\overline{BX}|\cdot
|\overline{BY}|=|\overline{AX}|\cdot |\overline{AY}| +
|\overline{CX}|\cdot |\overline{CY}|$.

\item Gibt es vier verschiedene positive ganze Zahlen, die
folgende Bedingung erf{\"u}llen: F{\"u}r je zwei von ihnen ist die Summe
ihres Produktes mit 2006 eine Quadratzahl?

\item Man bestimme alle positiven ganzen Zahlen $n$ mit $n^2\,
|\,3^n+1$.

\item F{\"u}r eine positive ganze Zahl $n$ sei mit $a_n$ die letzte
Ziffer von $n^{(n^n)}$ bezeichnet. Man beweise, dass die Folge
$(a_n)$ periodisch ist, und bestimme die minimale Periodenl{\"a}nge.

\item Gibt es eine Folge $a_1,a_2,a_3,\ldots$ aus positiven ganzen
Zahlen, so dass f{\"u}r jede positive ganze Zahl $n$ die Summe von je
$n$ aufeinander folgenden Elementen durch $n^2$ teilbar ist?

\item  Eine 12-stellige positive ganze Zahl enth{\"a}lt nur die
Ziffern 1, 5 und 9. Weiterhin ist sie durch 37 teilbar. Man
beweise, dass ihre Quersumme nicht 76 sein kann.


\end{problems}

Arbeitszeit: $4\frac{1}{2}$ Stunden. 5 Punkte pro Aufgabe.

\end{document}
