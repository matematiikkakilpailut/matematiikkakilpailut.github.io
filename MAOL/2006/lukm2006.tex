\input MAOL
\input makrot
\input laskut
\input pstricks

\otsikko
Lukion matematiikkakilpailun \\
 \\
loppukilpailu, 3.\:2.\:2006  \\

\bigskip
\harj
M��rit� kaikki positiivisten kokonaislukujen parit $(x,y)$,
joille
   $$x+y+xy=2006.$$

\harj
Osoita, ett� kaikilla reaaliluvuilla~$a$ p�tee
   $$3(1+a^2+a^4)\ge (1+a+a^2)^2.$$

\harj
Luvut $p$, $4p^2+1$ ja $6p^2+1$ ovat alkulukuja.  M��rit�~$p$.

\harj
Kolmion kaksi keskijanaa ovat kohtisuorassa toisiaan vastaan.
Todista, ett� kolmion keskijanat ovat er��n suorakulmaisen
kolmion sivut.

\harj
Kuvan $16\times 16$-ruudukolla pelataan Nelipe-peli�
seuraavasti:  Kaksi pelaajaa kirjoittaa vuoroin ruudukon
eri ruutuihin kokonaislukuja $1,2,\ldots 16$.  Luku
pit�� valita niin, ettei mik��n luku toistu mill��n rivill�,
sarakkeella eik� miss��n kuvan 16~pikkuneli�ss�.
Pelin h�vi�� ensimm�inen, joka ei voi siirt��.  Kumpi pelaajista
voittaa, aloittaja vai toinen pelaaja,
kun pelaajat pelaavat parhaalla mahdollisella tavalla?

\input pstricks
\psset{unit=4mm}
   $$
\pspicture*(-0.5,-0.5)(16.5,18)

\psline[linewidth=2pt](0,0)(16,0) \psline(0,1)(16,1) 
\psline(0,2)(16,2) \psline(0,3)(16,3)
\psline[linewidth=2pt](0,4)(16,4) \psline(0,5)(16,5) 
\psline(0,6)(16,6) \psline(0,7)(16,7)
\psline[linewidth=2pt](0,8)(16,8) \psline(0,9)(16,9) 
\psline(0,10)(16,10) \psline(0,11)(16,11)
\psline[linewidth=2pt](0,12)(16,12) \psline(0,13)(16,13) 
\psline(0,14)(16,14) \psline(0,15)(16,15)
\psline[linewidth=2pt](0,16)(16,16)

\psline[linewidth=2pt](0,0)(0,16) \psline(1,0)(1,16)
\psline(2,0)(2,16) \psline(3,0)(3,16)
\psline[linewidth=2pt](4,0)(4,16) \psline(5,0)(5,16) 
\psline(6,0)(6,16) \psline(7,0)(7,16)
\psline[linewidth=2pt](8,0)(8,16) \psline(9,0)(9,16) 
\psline(10,0)(10,16) \psline(11,0)(11,16)
\psline[linewidth=2pt](12,0)(12,16) \psline(13,0)(13,16) 
\psline(14,0)(14,16) \psline(15,0)(15,16)
\psline[linewidth=2pt](16,0)(16,16)

\endpspicture 
   $$

\bye