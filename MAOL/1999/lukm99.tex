\input makrot
\input laskut
\let\prob=\harj

\noindent {\bf Lukion matematiikkakilpailu }

\noindent {\bf Loppukilpailu 22.\ tammikuuta 1999}

\bigbreak

\prob Osoita, ett\"a yht\"al\"oll\"a $$x^3+2y^2+4z=n$$ on
kokonaislukuratkaisu $(x,\,y,\,z)$ kaikilla kokonaisluvuilla $n$.

\prob Oletetaan, ett\"a positiiviset luvut $a_1$, $a_2$, \dots, $a_n$
muodostavat aritmeettisen lukujonon; siis $a_{k+1}-a_k=d$ kaikilla $k=1$, 2,
\dots, $n-1$. Osoita, ett\"a $${1\over a_1a_2}+{1\over a_2a_3}+\cdots+{1\over
a_{n-1}a_n}={n-1\over a_1a_n}.$$

\prob Selvit\"a, montako alkulukua on jonossa
$$101,\,10101,\,1010101\,\dots$$

\prob Kolmella 1-s\"ateisell\"a ympyr\"all\"a on yhteinen piste $O$.
Ympyr\"at leikkaavat lis\"aksi toisensa pareittain pisteiss\"a $A$, $B$ ja
$C$. Osoita, pisteet $A$, $B$ ja $C$ ovat saman 1-s\"ateisen ympyr\"an
keh\"all\"a. 

\prob Tavallista dominolaattaa voidaan pit\"a\"a lukuparina $(k,\,m)$,
miss\"a luvut $k$ ja $m$ voivat saada arvoja 0, 1, 2, 3, 4, 5 ja 6. Parit
$(k,\,m)$ ja $(m,\,k)$ m\"a\"arittelev\"at saman laatan. Erityisesti pari
$(k,\,k)$ m\"a\"arittelee dominolaatan. Sanomme, ett\"a kaksi dominolaattaa
sopii yhteen, jos niiss\"a esiintyy sama luku. {\it Yleistetyiss\"a 
$n$-dominolaatoissa\/} $m$ ja $k$ voivat saada arvoja 0, 1, \dots, $n$.
Kuinka suuri on todenn\"ak\"oisyys, ett\"a kaksi satunnaisesti valittua 
$n$-dominolaattaa sopii yhteen?

\vskip3cm

\noindent Aikaa 3 tuntia

\bigbreak

\noindent Kirjoita jokainen ratkaisu omalle paperilleen.

\noindent Muista kirjoittaa nimesi jokaiseen ratkaisupaperiin!

\noindent Laskimien ja taulukkokirjojen k\"aytt\"o on kielletty.

\bye



