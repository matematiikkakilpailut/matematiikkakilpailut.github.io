\input makrot
\input laskut
\parindent=0pt


\bf
\vbox{\hbox{Lukion matematiikkakilpailu}
      \smallskip
      \hbox{Loppukilpailu 30.\:1.\:1998}}
\rm
\vskip 2,5truecm

\harj
Osoita, ett\"a pisteet $A$, $B$, $C$ ja $D$ voidaan sijoittaa
tasoon niin, ett\"a nelikulmion $ABCD$ pinta-ala on kaksi kertaa
niin suuri kuin nelikulmion $ADBC$.

\harj
Kilpailutoimikunnassa on $11$ j\"asent\"a.  Kilpailuteht\"avi\"a
s\"ailytet\"a\"an hyv\"ass\"a tallessa lukkojen takana.  Avaimia on jaeltu
toimikunnan j\"asenille niin, ett\"a ketk\"a tahansa kuusi j\"asent\"a
voivat avata lukot, mutta ketk\"a\"an 
viisi  eiv\"at riit\"a niiden avaamiseen.  Kuinka
monta lukkoa tarvitaan v\"ahint\"a\"an, ja kuinka monta avainta t\"all\"oin
on kullakin toimikunnan j\"asenell\"a?


\harj
Voiko jonosta $1/2$, $1/4$, $1/8,\ldots$ valita geometrisen,
p\"a\"attyv\"an tai p\"a\"attym\"att\"om\"an, jonon, 
jonka per\"akk\"aisten j\"asenten
suhde ei ole $1$ ja jonka summa on $1/5$?

\harj
Neli\"oss\"a, jonka sivu on $1$, on $110$ pistett\"a.  Osoita, ett\"a
jotkin nelj\"a n\"aist\"a sijaitsevat ympyr\"ass\"a, jonka s\"ade on $1/8$.

\harj
$15\times 36$-ruudukkoa peitet\"a\"an neli\"olaatoilla.
Neli\"olaattoja on kahdenkokoisia; sivun pituus voi olla $7$ tai $5$.
Laattojen tulee peitt\"a\"a t\"aysi\"a
yksikk\"oneli\"oit\"a, eiv\"atk\"a ne saa menn\"a p\"a\"allekk\"ain.  Kuinka monta
ruutua laatat voivat peitt\"a\"a?

\vskip 3truecm
Aikaa: 3 tuntia

\bigskip
{\obeylines
Kirjoita kukin ratkaisu omalle paperilleen.
Muista kirjoittaa nimesi kuhunkin paperiin.
Laskimien ja taulukkokirjojen k\"aytt\"o on kielletty.
}

\medskip\parni
Mik\"ali sinulla on s\"ahk\"opostiosoite ja olet kiinnostunut
matematiikkavalmennuksesta, sinun kannattaa kirjoittaa
my\"os s\"ahk\"opostiosoitteesi yhdelle ratkaisupapereista.


\bye
