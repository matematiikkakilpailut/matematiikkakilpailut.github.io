\input makrot
\input laskut
\parindent=0pt
\let\0=\displaystyle
\overfullrule=0pt

\bf
\vbox{\hbox{Lukion matematiikkakilpailu}
      \smallskip
      \hbox{Loppukilpailu 2.\:2.\:2001}}
\rm
\vskip 2,5truecm

\harj
Suorakulmaisessa kolmiossa $ABC$ hypotenuusaa~$AB$
vastaan piirretty korkeusjana on $CF$.  $F$:n kautta kulkeva $B$-keskinen
ympyr� ja samans�teinen $A$-keskinen ympyr� leikkaavat toisensa
sivun~$CB$ pisteess�. M��rit� suhde $FB:BC$.


\harj
Toisiaan leikkaamattomien k�yrien yht�l�t ovat $y=ax^2+bx+c$ ja
$y=dx^2+ex+f$, miss� $ad<0$.  Todista, ett� on olemassa tason suora,
joka ei leikkaa kumpaakaan n�ist� k�yrist�.

\harj
Luvut $a$, $b$ ja $c$ ovat positiivisia kokonaislukuja ja
${\0 1\over \0 a}+{\0 1\over \0 b}+{\0 1\over \0 c}<1$.  Osoita, ett�
   $${1\over a}+{1\over b}+{1\over c}\le{41\over 42}.$$

\harj
Jokaviikkoisessa jokeriarvonnassa arvotaan seitsem�n numeron
jono.  Jokainen numero voi olla mik� tahansa luvuista
$0,1,2,3,4,5,6,7,8,9$. Kuinka suuri on todenn�k�isyys, ett�
jokeriarvonnan jonossa esiintyy korkeintaan viitt� eri numeroa?

\harj
M��rit� sellaiset $n\in\NN$, ett� $n^2+2$ on luvun $2+2001n$ tekij�.

\vskip 3truecm
Aikaa: 3 tuntia

\bigskip
{\obeylines
Kirjoita kukin ratkaisu omalle paperilleen.
Muista kirjoittaa nimesi kuhunkin paperiin.
Laskimien ja taulukkokirjojen k\"aytt\"o on kielletty.
\medskip
Kotisivuja:
{\tt http://www.maol.fi} (MAOL)
{\tt http://www.math.helsinki.fi/\~{}smy/olympia} (matematiikkavalmennus)
}




\bye






