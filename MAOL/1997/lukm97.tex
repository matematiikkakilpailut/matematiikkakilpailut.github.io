\input makrot
\input laskut
\overfullrule=0pt


\bf
\vbox{\hbox{Lukion matematiikkakilpailu}
      \smallskip
      \hbox{Loppukilpailu 25.\:1.\:1997}}
\rm
\vskip 2,5truecm

\harj
M\"a\"arit\"a ne luvut $a$, joille yht\"al\"oll\"a
   $$a3^x+3^{-x}=3$$
on tasan yksi ratkaisu $x$.

\harj
Ympyr\"at, joiden s\"ateet ovat $R$ ja $r$, miss\"a $R>r$,
sivuavat toisiaan ulkopuolisesti.  Ympyr\"oille piirret\"a\"an yhteinen
tangentti, joka ei kulje ympyr\"oiden sivuamispisteen kautta.
T\"am\"an tangentin ja ympyr\"oiden rajoittamaan alueeseen piirret\"a\"an
mahdollisimman suuri ympyr\"a.  Kuinka suuri on t\"am\"an ympyr\"an s\"ade?

\harj
Py\"ore\"an p\"oyd\"an \"a\"aress\"a on 12 ritaria.  Jokainen ritari on vihoissa
viereisten ritarien, mutta ei muiden ritarien, kanssa.  Viisi ritaria
on valittava pelastamaan prinsessaa.  Yht\"a\"an vihamiesparia ei haluta
mukaan. Kuinka monella eri tavalla valinta voidaan suorittaa?

\harj
Laske kaikkien sellaisten nelinumeroisten lukujen, joiden
kymmenj\"arjestelm\"aesityksess\"a on vain parittomia numeroita, summa.

\harj
Sijoita tasoon $n$ pistett\"a ($n\ge 3$) niin, ett\"a mink\"a\"an kahden pisteen
et\"aisyys ei ylit\"a yht\"a ja t\"asm\"alleen $n$ pisteparin v\"alinen
et\"aisyys on yksi.

\vskip 3truecm
Aikaa: 3 tuntia


\bye
