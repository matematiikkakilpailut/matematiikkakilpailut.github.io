\input makrot
\input laskut
\parindent=0pt
\let\0=\displaystyle
\overfullrule=0pt


\vbox{\sser\hbox{Lukion matematiikkakilpailu}
      \smallskip
      \hbox{Loppukilpailu 7.\:2.\:2003}}


\vskip 2truecm

\harj
Kolmion~$ABC$ sis��n piirretyn ympyr�n keskipiste on $I$.
Puolisuorat~$AI$, $BI$ ja $CI$ leikkaavat kolmion~$ABC$
ymp�ri piirretyn ympyr�n pisteiss�~$D$, $E$ ja $F$.
Todista, ett� $AD$ ja $EF$ ovat kohtisuorassa toisiaan
vastaan.

\harj
Mink� per�kk�isten kokonaislukujen v�lill� on lausekkeen
   $${1\over x_1+1}+{1\over x_2+1}+{1\over x_3+1}\3+ 
     {1\over x_{2001}+1}+{1\over x_{2002}+1}$$
arvo, kun $x_1=1/3$ ja $x_{n+1}={x_n}^2+x_n$?

\harj
P�yd�ll� on kuuden eri ihmisen tyhj�t kukkarot.  Kuinka monella
tavalla niihin voidaan sijoittaa $12$ kahden euron kolikkoa
niin, ett� korkeintaan yksi j�� tyhj�ksi.

\harj
Etsi ne positiivisten kokonaislukujen parit $(n,k)$, joille
   $$(n+1)^k-1=n!.$$

\harj
Pelaajat Aino ja Eino valitsevat vuorotellen eri lukuja joukosta
$\joukko{0\3,n}$, miss� $n\in\NN$ on ennalta kiinnitetty
luku.  Peli p��ttyy, kun jommankumman pelaajan luvuista
voi valita nelj�, jotka sopivassa j�rjestyksess� muodostavat
aritmeettisen jonon.  Pelin voittaa pelaaja, jonka luvuista t�llaisen
jonon voi muodostaa.  Osoita, ett� on olemassa sellainen $n$,
ett� aloittajalla on voittostrategia.  Etsi mahdollisimman pieni
t�llainen $n$.


\vfill
\sser 
Aikaa 3 tuntia

\bigskip
{\obeylines
Kirjoita kukin ratkaisu omalle paperilleen.
Muista kirjoittaa nimesi kuhunkin paperiin.
Laskimien ja taulukkokirjojen k\"aytt\"o on kielletty.}



\bigskip
\hrule
\smallskip
\line{
\sser Kotisivuja: \hfill
\raise -1ex\hbox{\vbox{
   \hbox{{\tt http://www.maol.fi} (MAOL)} 
   \hbox{{\tt http://www.math.helsinki.fi/\~{}smy/olympia}
          (mat.valmennus)} 
     }}
}





\bye






